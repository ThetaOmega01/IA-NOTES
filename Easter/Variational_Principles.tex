\documentclass[a4paper]{article}

\newcommand{\triposcourse}{Variational Principles}
\usepackage{fancyhdr,titlesec,geometry}
\usepackage[dvipsnames]{xcolor}
\usepackage[many]{tcolorbox}
\usepackage{xifthen}
\usepackage{import}
\usepackage{parskip}
\usepackage{transparent}
\usepackage{mathtools,amssymb,amsfonts,amsthm,bm}   % Math Presets
\usepackage{array,tabularx,booktabs}                % Table Presets
\usepackage{graphicx,wrapfig,float,caption}         % Figure Presets
\usepackage{setspace,multicol}                      % Text Presets
\usepackage{tikz,physics,cancel,tkz-euclide,pgfplots,tikz-3dplot}                    % Physics Presets
\usepackage{amsmath}
\usepackage{mathrsfs}
\usepackage{enumerate}
\usepackage[shortlabels]{enumitem}
\usepackage{hyperref}
\usepackage{lipsum}
\usepackage{IEEEtrantools}
\usepackage{xcomment}

\tdplotsetmaincoords{60}{120}

\usetikzlibrary{arrows.meta}
\usetikzlibrary{decorations.markings}
\usetikzlibrary{decorations.pathmorphing}
\usetikzlibrary{automata, positioning}
\usetikzlibrary{fadings}
\usetikzlibrary{intersections}
\usetikzlibrary{cd}
\usetikzlibrary{patterns}
\usetikzlibrary{shapes.arrows}
\usepgfplotslibrary{colormaps, external}
\pgfarrowsdeclarecombine{twolatex'}{twolatex'}{latex'}{latex'}{latex'}{latex'}
\tikzset{->/.style = {decoration={markings,
                                  mark=at position 1 with {\arrow[scale=1.6]{latex'}}},
                      postaction={decorate}}}
\tikzset{<-/.style = {decoration={markings,
                                  mark=at position 0 with {\arrowreversed[scale=1.6]{latex'}}},
                      postaction={decorate}}}
\tikzset{<->/.style = {decoration={markings,
                                   mark=at position 0 with {\arrowreversed[scale=1.6]{latex'}},
                                   mark=at position 1 with {\arrow[scale=1.6]{latex'}}},
                       postaction={decorate}}}
\tikzset{->-/.style = {decoration={markings,
                                   mark=at position #1 with {\arrow[scale=1.6]{latex'}}},
                       postaction={decorate}}}
\tikzset{-<-/.style = {decoration={markings,
                                   mark=at position #1 with {\arrowreversed[scale=1.6]{latex'}}},
                       postaction={decorate}}}
\tikzset{->>/.style = {decoration={markings,
                                  mark=at position 1 with {\arrow[scale=1.6]{twolatex'}}},
                      postaction={decorate}}}
\tikzset{<<-/.style = {decoration={markings,
                                  mark=at position 0 with {\arrowreversed[scale=1.6]{twolatex'}}},
                      postaction={decorate}}}
\tikzset{<<->>/.style = {decoration={markings,
                                   mark=at position 0 with {\arrowreversed[scale=1.6]{twolatex'}},
                                   mark=at position 1 with {\arrow[scale=1.6]{twolatex'}}},
                       postaction={decorate}}}
\tikzset{->>-/.style = {decoration={markings,
                                   mark=at position #1 with {\arrow[scale=1.6]{twolatex'}}},
                       postaction={decorate}}}
\tikzset{-<<-/.style = {decoration={markings,
                                   mark=at position #1 with {\arrowreversed[scale=1.6]{twolatex'}}},
                       postaction={decorate}}}

\tikzset{
set arrow inside/.code={\pgfqkeys{/tikz/arrow inside}{#1}},
set arrow inside={end/.initial=>, opt/.initial=},
/pgf/decoration/Mark/.style={
    mark/.expanded=at position #1 with
    {
        \noexpand\arrow[\pgfkeysvalueof{/tikz/arrow inside/opt}]{\pgfkeysvalueof{/tikz/arrow inside/end}}
    }
},
arrow inside/.style 2 args={
    set arrow inside={#1},
    postaction={
        decorate,decoration={
            markings,Mark/.list={#2}
        }
    }
},
}

\tikzstyle{circ}=[fill=black, draw=black, shape=circle]
\tikzset{
dot/.style = {circle, fill, minimum size=#1,
              inner sep=0pt, outer sep=0pt},
dot/.default = 5pt% size of the circle diameter 
}
\tikzset{mstate/.style={circle, draw, blue, text=black, minimum width=0.7cm}}

\def\centerarc[#1](#2)(#3:#4:#5)% Syntax: [draw options](center)(initial angle:final angle:radius)
    { \draw[#1] ($(#2)+({#5*cos(#3)},{#5*sin(#3)})$) arc (#3:#4:#5); }

\hypersetup{
    colorlinks=true,
    linkcolor=blue,
    filecolor=blue,
    citecolor = black,      
    urlcolor=cyan,
    }

%%%%%%%%%%% Snippets %%%%%%%%%%%%%%%%
\newcommand*\widefbox[1]{\fbox{\hspace{2em}#1\hspace{2em}}}
\newcommand{\xint}{\int_{x_1}^{x_2}}
\newcommand{\mw}{\sqrt{m\omega}}
\newcommand{\de}{\delta}
\newcommand{\dde}{\dot{\delta}}
\newcommand{\di}{\delta_i}
\newcommand{\ddi}{\dot{\delta_i}}
\newcommand{\dddi}{\ddot{\delta_i}}
\newcommand{\dipl}{\delta_{i+1}}
\newcommand{\dimi}{\delta_{i-1}}
\newcommand{\ddt}[1]{\frac{{d} #1}{dt}}
\newcommand{\ddtt}[1]{\frac{d^2 #1}{dt^2}}
\newcommand{\ddx}[1]{\frac{d #1}{dx}}
\newcommand{\ddxx}[1]{\frac{d^2 #1}{dx^2}}
\newcommand{\eps}{\epsilon}
\newcommand{\del}[2]{\frac{\partial #1}{\partial #2}}
\newcommand{\deltwo}[2]{\frac{\partial^2 #1}{\partial #2^2}}
\newcommand{\lam}{\lambda}
\newcommand{\Lam}{\Lambda}
\newcommand{\sig}{\sigma}
\newcommand{\Sig}{\Sigma}
\newcommand{\half}{\frac{1}{2}}
\newcommand{\munu}{{\mu\nu}}
\newcommand{\thalf}{\tfrac{1}{2}}
\renewcommand{\div}{\nabla\cdot}
\renewcommand{\curl}{\nabla\times}

\DeclareMathOperator{\orb}{Orb}
\DeclareMathOperator{\stab}{Stab}
\DeclareMathOperator{\adj}{adj}
\DeclareMathOperator{\ccl}{ccl}
\let\var\relax
\DeclareMathOperator{\var}{Var}
\DeclareMathOperator{\cov}{Cov}
\DeclareMathOperator{\corr}{Corr}
\DeclareMathOperator{\Markov}{Markov}
\DeclareMathOperator{\nullity}{nullity}

\newcommand{\bfA}{{\bf A}}
\newcommand{\bfB}{{\bf B}}
\newcommand{\bfC}{{\bf C}}
\newcommand{\bfD}{{\bf D}}
\newcommand{\bfE}{{\bf E}}
\newcommand{\bfF}{{\bf F}}
\newcommand{\bfG}{{\bf G}}
\newcommand{\bfH}{{\bf H}}
\newcommand{\bfI}{{\bf I}}
\newcommand{\bfJ}{{\bf J}}
\newcommand{\bfK}{{\bf K}}
\newcommand{\bfL}{{\bf L}}
\newcommand{\bfM}{{\bf M}}
\newcommand{\bfN}{{\bf N}}
\newcommand{\bfO}{{\bf O}}
\newcommand{\bfP}{{\bf P}}
\newcommand{\bfQ}{{\bf Q}}
\newcommand{\bfR}{{\bf R}}
\newcommand{\bfS}{{\bf S}}
\newcommand{\bfT}{{\bf T}}
\newcommand{\bfU}{{\bf U}}
\newcommand{\bfV}{{\bf V}}
\newcommand{\bfW}{{\bf W}}
\newcommand{\bfX}{{\bf X}}
\newcommand{\bfY}{{\bf Y}}
\newcommand{\bfZ}{{\bf Z}}

\newcommand{\bfa}{{\bf a}}
\newcommand{\bfb}{{\bf b}}
\newcommand{\bfc}{{\bf c}}
\newcommand{\bfd}{{\bf d}}
\newcommand{\bfe}{{\bf e}}
\newcommand{\bff}{{\bf f}}
\newcommand{\bfg}{{\bf g}}
\newcommand{\bfh}{{\bf h}}
\newcommand{\bfi}{{\bf i}}
\newcommand{\bfj}{{\bf j}}
\newcommand{\bfk}{{\bf k}}
\newcommand{\bfl}{{\bf l}}
\newcommand{\bfm}{{\bf m}}
\newcommand{\bfn}{{\bf n}}
\newcommand{\bfo}{{\bf o}}
\newcommand{\bfp}{{\bf p}}
\newcommand{\bfq}{{\bf q}}
\newcommand{\bfr}{{\bf r}}
\newcommand{\bfs}{{\bf s}}
\newcommand{\bft}{{\bf t}}
\newcommand{\bfu}{{\bf u}}
\newcommand{\bfv}{{\bf v}}
\newcommand{\bfw}{{\bf w}}
\newcommand{\bfx}{{\bf x}}
\newcommand{\bfy}{{\bf y}}
\newcommand{\bfz}{{\bf z}}

\newcommand{\mcA}{{\mathcal{A}}}
\newcommand{\mcB}{{\mathcal{B}}}
\newcommand{\mcC}{{\mathcal{C}}}
\newcommand{\mcD}{{\mathcal{D}}}
\newcommand{\mcE}{{\mathcal{E}}}
\newcommand{\mcF}{{\mathcal{F}}}
\newcommand{\mcG}{{\mathcal{G}}}
\newcommand{\mcH}{{\mathcal{H}}}
\newcommand{\mcI}{{\mathcal{I}}}
\newcommand{\mcJ}{{\mathcal{J}}}
\newcommand{\mcK}{{\mathcal{K}}}
\newcommand{\mcL}{{\mathcal{L}}}
\newcommand{\mcM}{{\mathcal{M}}}
\newcommand{\mcN}{{\mathcal{N}}}
\newcommand{\mcO}{{\mathcal{O}}}
\newcommand{\mcP}{{\mathcal{P}}}
\newcommand{\mcQ}{{\mathcal{Q}}}
\newcommand{\mcR}{{\mathcal{R}}}
\newcommand{\mcS}{{\mathcal{S}}}
\newcommand{\mcT}{{\mathcal{T}}}
\newcommand{\mcU}{{\mathcal{U}}}
\newcommand{\mcV}{{\mathcal{V}}}
\newcommand{\mcW}{{\mathcal{W}}}
\newcommand{\mcX}{{\mathcal{X}}}
\newcommand{\mcY}{{\mathcal{Y}}}
\newcommand{\mcZ}{{\mathcal{Z}}}

\newcommand{\bbA}{{\mathbb{A}}}
\newcommand{\bbB}{{\mathbb{B}}}
\newcommand{\bbC}{{\mathbb{C}}}
\newcommand{\bbD}{{\mathbb{D}}}
\newcommand{\bbE}{{\mathbb{E}}}
\newcommand{\bbF}{{\mathbb{F}}}
\newcommand{\bbG}{{\mathbb{G}}}
\newcommand{\bbH}{{\mathbb{H}}}
\newcommand{\bbI}{{\mathbb{I}}}
\newcommand{\bbJ}{{\mathbb{J}}}
\newcommand{\bbK}{{\mathbb{K}}}
\newcommand{\bbL}{{\mathbb{L}}}
\newcommand{\bbM}{{\mathbb{M}}}
\newcommand{\bbN}{{\mathbb{N}}}
\newcommand{\bbO}{{\mathbb{O}}}
\newcommand{\bbP}{{\mathbb{P}}}
\newcommand{\bbQ}{{\mathbb{Q}}}
\newcommand{\bbR}{{\mathbb{R}}}
\newcommand{\bbS}{{\mathbb{S}}}
\newcommand{\bbT}{{\mathbb{T}}}
\newcommand{\bbU}{{\mathbb{U}}}
\newcommand{\bbV}{{\mathbb{V}}}
\newcommand{\bbW}{{\mathbb{W}}}
\newcommand{\bbX}{{\mathbb{X}}}
\newcommand{\bbY}{{\mathbb{Y}}}
\newcommand{\bbZ}{{\mathbb{Z}}}

\newcommand{\mfa}{{\mathfrak{a}}}
\newcommand{\mfb}{{\mathfrak{b}}}
\newcommand{\mfc}{{\mathfrak{c}}}
\newcommand{\mfd}{{\mathfrak{d}}}
\newcommand{\mfe}{{\mathfrak{e}}}
\newcommand{\mff}{{\mathfrak{f}}}
\newcommand{\mfg}{{\mathfrak{g}}}
\newcommand{\mfh}{{\mathfrak{h}}}
\newcommand{\mfi}{{\mathfrak{i}}}
\newcommand{\mfj}{{\mathfrak{j}}}
\newcommand{\mfk}{{\mathfrak{k}}}
\newcommand{\mfl}{{\mathfrak{l}}}
\newcommand{\mfm}{{\mathfrak{m}}}
\newcommand{\mfn}{{\mathfrak{n}}}
\newcommand{\mfo}{{\mathfrak{o}}}
\newcommand{\mfp}{{\mathfrak{p}}}
\newcommand{\mfq}{{\mathfrak{q}}}
\newcommand{\mfr}{{\mathfrak{r}}}
\newcommand{\mfs}{{\mathfrak{s}}}
\newcommand{\mft}{{\mathfrak{t}}}
\newcommand{\mfu}{{\mathfrak{u}}}
\newcommand{\mfv}{{\mathfrak{v}}}
\newcommand{\mfw}{{\mathfrak{w}}}
\newcommand{\mfx}{{\mathfrak{x}}}
\newcommand{\mfy}{{\mathfrak{y}}}
\newcommand{\mfz}{{\mathfrak{z}}}

\newcommand{\mfA}{{\mathfrak{A}}}
\newcommand{\mfB}{{\mathfrak{B}}}
\newcommand{\mfC}{{\mathfrak{C}}}
\newcommand{\mfD}{{\mathfrak{D}}}
\newcommand{\mfE}{{\mathfrak{E}}}
\newcommand{\mfF}{{\mathfrak{F}}}
\newcommand{\mfG}{{\mathfrak{G}}}
\newcommand{\mfH}{{\mathfrak{H}}}
\newcommand{\mfI}{{\mathfrak{I}}}
\newcommand{\mfJ}{{\mathfrak{J}}}
\newcommand{\mfK}{{\mathfrak{K}}}
\newcommand{\mfL}{{\mathfrak{L}}}
\newcommand{\mfM}{{\mathfrak{M}}}
\newcommand{\mfN}{{\mathfrak{N}}}
\newcommand{\mfO}{{\mathfrak{O}}}
\newcommand{\mfP}{{\mathfrak{P}}}
\newcommand{\mfQ}{{\mathfrak{Q}}}
\newcommand{\mfR}{{\mathfrak{R}}}
\newcommand{\mfS}{{\mathfrak{S}}}
\newcommand{\mfT}{{\mathfrak{T}}}
\newcommand{\mfU}{{\mathfrak{U}}}
\newcommand{\mfV}{{\mathfrak{V}}}
\newcommand{\mfW}{{\mathfrak{W}}}
\newcommand{\mfX}{{\mathfrak{X}}}
\newcommand{\mfY}{{\mathfrak{Y}}}
\newcommand{\mfZ}{{\mathfrak{Z}}}

\newcommand{\rma}{\mathrm{a}}
\newcommand{\rmb}{\mathrm{b}}
\newcommand{\rmc}{\mathrm{c}}
\newcommand{\rmd}{\mathrm{d}}
\renewcommand{\dd}{\,\mathrm{d}}
\newcommand{\rme}{\mathrm{e}}
\newcommand{\rmf}{\mathrm{f}}
\newcommand{\rmg}{\mathrm{g}}
\newcommand{\rmh}{\mathrm{h}}
\newcommand{\rmi}{\mathrm{i}}
\newcommand{\rmj}{\mathrm{j}}
\newcommand{\rmk}{\mathrm{k}}
\newcommand{\rml}{\mathrm{l}}
\newcommand{\rmm}{\mathrm{m}}
\newcommand{\rmn}{\mathrm{n}}
\newcommand{\rmo}{\mathrm{o}}
\newcommand{\rmp}{\mathrm{p}}
\newcommand{\rmq}{\mathrm{q}}
\newcommand{\rmr}{\mathrm{r}}
\newcommand{\rms}{\mathrm{s}}
\newcommand{\rmt}{\mathrm{t}}
\newcommand{\rmu}{\mathrm{u}}
\newcommand{\rmv}{\mathrm{v}}
\newcommand{\rmw}{\mathrm{w}}
\newcommand{\rmx}{\mathrm{x}}
\newcommand{\rmy}{\mathrm{y}}
\newcommand{\rmz}{\mathrm{z}}
\newcommand{\rmA}{\mathrm{A}}
\newcommand{\rmB}{\mathrm{B}}
\newcommand{\rmC}{\mathrm{C}}
\newcommand{\rmD}{\mathrm{D}}
\newcommand{\rmE}{\mathrm{E}}
\newcommand{\rmF}{\mathrm{F}}
\newcommand{\rmG}{\mathrm{G}}
\newcommand{\rmH}{\mathrm{H}}
\newcommand{\rmI}{\mathrm{I}}
\newcommand{\rmJ}{\mathrm{J}}
\newcommand{\rmK}{\mathrm{K}}
\newcommand{\rmL}{\mathrm{L}}
\newcommand{\rmM}{\mathrm{M}}
\newcommand{\rmN}{\mathrm{N}}
\newcommand{\rmO}{\mathrm{O}}
\newcommand{\rmP}{\mathrm{P}}
\newcommand{\rmQ}{\mathrm{Q}}
\newcommand{\rmR}{\mathrm{R}}
\newcommand{\rmS}{\mathrm{S}}
\newcommand{\rmT}{\mathrm{T}}
\newcommand{\rmU}{\mathrm{U}}
\newcommand{\rmV}{\mathrm{V}}
\newcommand{\rmW}{\mathrm{W}}
\newcommand{\rmX}{\mathrm{X}}
\newcommand{\rmY}{\mathrm{Y}}
\newcommand{\rmZ}{\mathrm{Z}}

\newcommand{\GL}{\mathrm{GL}}
\newcommand{\Or}{\mathrm{O}}
\newcommand{\PGL}{\mathrm{PGL}}
\newcommand{\PSL}{\mathrm{PSL}}
\newcommand{\PSO}{\mathrm{PSO}}
\newcommand{\PSU}{\mathrm{PSU}}
\newcommand{\SL}{\mathrm{SL}}
\newcommand{\SO}{\mathrm{SO}}
\newcommand{\Spin}{\mathrm{Spin}}
\newcommand{\Sp}{\mathrm{Sp}}
\newcommand{\SU}{\mathrm{SU}}
\newcommand{\Mat}{\mathrm{Mat}}

% Some common notations

\renewcommand{\v}{\mathbf{v}}
\newcommand{\w}{\mathbf{w}}
\renewcommand{\u}{\mathbf{u}}

% Matrix algebras
\newcommand{\gl}{\mathfrak{gl}}
\newcommand{\ort}{\mathfrak{o}}
\newcommand{\so}{\mathfrak{so}}
\newcommand{\su}{\mathfrak{su}}
\newcommand{\uu}{\mathfrak{u}}
\renewcommand{\sl}{\mathfrak{sl}}
\DeclareMathOperator{\spn}{span}

\newcommand{\mobius}{{M\"{o}bius }}

\renewcommand{\ge}{\geqslant}
\renewcommand{\le}{\leqslant}
\renewcommand{\geq}{\geqslant}
\renewcommand{\leq}{\leqslant}

\newcommand\independent{\protect\mathpalette{\protect\independenT}{\perp}}
\def\independenT#1#2{\mathrel{\rlap{$#1#2$}\mkern2mu{#1#2}}}

\setlength{\parindent}{0pt}
% \setlength{\parskip}{\baselineskip}
\newcommand{\incfig}[1]{%
    \def\svgwidth{0.4\columnwidth}
    \import{./figures/}{#1.pdf_tex}
}
%%%%%%%%%%%%%%%%%%%%%%%%%%%%%%%%%%%%%

% \usepackage[T1]{fontenc}
\usepackage{lmodern,mathrsfs}

%%%%%%%boxed enviroment for final layout%%%%%%%%%%%%%

\newtheoremstyle{mystyle}%
  {}%
  {}%
  {}%
  {}%
  {\sffamily\bfseries}%
  {.}%
  { }%
  {}%

% \renewenvironment{proof}{{\sffamily\bfseries Proof. }}{\qed}

\theoremstyle{mystyle}{
  \newtheorem{theorem}{Theorem}[section]
  \newtheorem{lemma}[theorem]{Lemma}
  \newtheorem{proposition}[theorem]{Proposition}
  \newtheorem{corollary}[theorem]{Corollary}
  \newtheorem{problem}[theorem]{Problem}
  \newtheorem*{claim}{Claim}
  \newtheorem*{slemma}{Lemma}
  \newtheorem*{sprop}{Proposition}
  \newtheorem*{notation}{Notation}

  \newtheorem{inquestion}{Question}
  \newtheorem*{sque}{Question}

  \newtheorem{definition}{Definition}[section]
  \newtheorem{conjecture}{Conjecture}[section]
  \newtheorem{example}{Example}[section]
  \newtheorem*{law}{Law}

  \newtheorem*{remark}{Remark}
  \newtheorem*{note}{Note}
}

\newenvironment{question}[1]
{\renewcommand\theinquestion{#1}\inquestion}
{\endinquestion}

\theoremstyle{definition}{
    \newtheorem*{exercise}{Exercise}}

\tcolorboxenvironment{definition}{
  boxrule=0pt,
  boxsep=2pt,
  colback={White!90!Cerulean},
  enhanced jigsaw, 
  borderline west={2pt}{0pt}{Cerulean},
  sharp corners,
  before skip=10pt,
  after skip=10pt,
  breakable,
  parbox=false,
}

\tcolorboxenvironment{notation}{
  boxrule=0pt,
  boxsep=2pt,
  colback={White!90!Cerulean},
  enhanced jigsaw, 
  borderline west={2pt}{0pt}{Cerulean},
  sharp corners,
  before skip=10pt,
  after skip=10pt,
  breakable,
  parbox=false,
}

\tcolorboxenvironment{proposition}{
  boxrule=0pt,
  boxsep=2pt,
  colback={White!90!Yellow},
  enhanced jigsaw, 
  borderline west={2pt}{0pt}{Yellow},
  sharp corners,
  before skip=10pt,
  after skip=10pt,
  breakable,
  parbox=false,
}

\tcolorboxenvironment{sprop}{
  boxrule=0pt,
  boxsep=2pt,
  colback={White!90!Yellow},
  enhanced jigsaw, 
  borderline west={2pt}{0pt}{Yellow},
  sharp corners,
  before skip=10pt,
  after skip=10pt,
  breakable,
  parbox=false,
}

\tcolorboxenvironment{theorem}{
  boxrule=0pt,
  boxsep=2pt,
  colback={White!90!Dandelion},
  enhanced jigsaw, 
  borderline west={2pt}{0pt}{Dandelion},
  sharp corners,
  before skip=10pt,
  after skip=10pt,
  breakable,
  parbox=false,
}

\tcolorboxenvironment{lemma}{
  boxrule=0pt,
  boxsep=2pt,
  blanker,
  borderline west={2pt}{0pt}{Red},
  before skip=10pt,
  after skip=10pt,
  sharp corners,
  left=12pt,
  right=12pt,
  breakable,
  parbox=false,
}

\tcolorboxenvironment{corollary}{
  boxrule=0pt,
  boxsep=2pt,
  blanker,
  borderline west={2pt}{0pt}{ForestGreen},
  before skip=10pt,
  after skip=10pt,
  sharp corners,
  left=12pt,
  right=12pt,
  breakable,
  parbox=false,
}

\tcolorboxenvironment{proof}{
  boxrule=0pt,
  boxsep=2pt,
  blanker,
  borderline west={2pt}{0pt}{NavyBlue!80!white},
  before skip=10pt,
  after skip=10pt,
  left=12pt,
  right=12pt,
  breakable,
  parbox=false,
}

\tcolorboxenvironment{remark}{
  boxrule=0pt,
  boxsep=2pt,
  blanker,
  borderline west={2pt}{0pt}{Green},
  before skip=10pt,
  after skip=10pt,
  left=12pt,
  right=12pt,
  breakable,
  parbox=false,
}

\tcolorboxenvironment{note}{
  boxrule=0pt,
  boxsep=2pt,
  blanker,
  borderline west={2pt}{0pt}{PineGreen},
  before skip=10pt,
  after skip=10pt,
  left=12pt,
  right=12pt,
  breakable,
  parbox=false,
}

\tcolorboxenvironment{example}{
  boxrule=0pt,
  boxsep=2pt,
  blanker,
  borderline west={2pt}{0pt}{Black},
  sharp corners,
  before skip=10pt,
  after skip=10pt,
  left=12pt,
  right=12pt,
  breakable,
  parbox=false,
}

\titleformat*{\section}{\Large\bfseries\sffamily}
\titleformat*{\subsection}{\large\bfseries\sffamily}
\titleformat*{\subsubsection}{\bfseries\sffamily}
\titleformat*{\paragraph}{\bfseries\sffamily}

%%%%%%%%%%%%%%%%%%%%%%%%%%%%%%%%%%%%%%%%%%%%%%%%%%%%

\title{\textbf{\sffamily\triposcourse{} Notes}}
% \usepackage[T1]{fontenc}
\usepackage{crimson}

\theoremstyle{plain}

\theoremstyle{definition}
\newtheorem{theorem}{Theorem}[section]
\newtheorem{lemma}[theorem]{Lemma}
\newtheorem{proposition}[theorem]{Proposition}
\newtheorem{corollary}[theorem]{Corollary}
\newtheorem{problem}[theorem]{Problem}
\newtheorem*{claim}{Claim}
\newtheorem*{slemma}{Lemma}
\newtheorem*{sprop}{Proposition}
\newtheorem*{notation}{Notation}
\newtheorem*{exercise}{Exercise}

\newtheorem{inquestion}{Question}
\newtheorem*{sque}{Question}
\newenvironment{question}[1]
  {\renewcommand\theinquestion{#1}\inquestion}
  {\endinquestion}

\newtheorem{definition}{Definition}[section]
\newtheorem{conjecture}{Conjecture}[section]
\newtheorem{example}{Example}[section]
\newtheorem*{law}{Law}

\theoremstyle{remark}
\newtheorem*{remark}{Remark}
\newtheorem*{note}{Note}

\title{\textbf{\triposcourse{} Notes}}
\theoremstyle{plain}{
  \newtheorem{theorem}{Theorem}[section]
  \newtheorem{lemma}[theorem]{Lemma}
  \newtheorem{proposition}[theorem]{Proposition}
  \newtheorem{corollary}[theorem]{Corollary}
  \newtheorem*{claim}{Claim}
  \newtheorem*{slemma}{Lemma}
  \newtheorem*{sprop}{Proposition}
  \newtheorem{conjecture}{Conjecture}[section]
  \newtheorem*{law}{Law}
  \newtheorem{inquestion}{Question}
  \newtheorem*{sque}{Question}
}

\theoremstyle{definition}{
  \newtheorem{definition}{Definition}[section]
  \newtheorem{example}{Example}[section]
  \newtheorem*{notation}{Notation}
  \newtheorem*{exercise}{Exercise}
}

\theoremstyle{remark}{
  \newtheorem*{remark}{Remark}
  \newtheorem*{note}{Note}
}

\newenvironment{question}[1]
{\renewcommand\theinquestion{#1}\inquestion}
{\endinquestion}

\title{\textbf{\sffamily\triposcourse{} Notes}}

%layout full
% \geometry{%
%   a4paper,
%   lmargin=2cm,
%   rmargin=2.5cm,
%   tmargin=3.5cm,
%   bmargin=2.5cm,
%   footskip=12pt,
%   headheight=24pt}
% layout trim
% \geometry{
% papersize={379pt, 542pt},
% textwidth=345pt,
% textheight=443pt,
% left=17pt,
% top=54pt,
% right=17pt
% }
\pagestyle{fancy}
\rhead{\textsc{\triposcourse{}}}
\author{You}

\graphicspath{ {./images/} }
\pgfplotsset{compat=1.17}
\begin{document}
\maketitle
\tableofcontents
\setcounter{section}{-1}
\newpage
\part*{Lecture 1}
\section{Motivation}
The history of variational principles can be dated back to a pretty concrete classical problem known as the Brachistochrone problem:
\begin{example}[The Brachistochrone Problem]
    Consider a particle moving under the influence of gravity on a wire joining points $A$ and $B$.
    What shape of the curve gives the shortest travel time of the particle given that it starts from rest?\\
    Mathematically, we want to minimise the quantity
    $$\tau=\int_A^B\mathrm dt=\int_A^B\frac{\mathrm dL}{v(x,y)}$$
    This problem was first created by Johann Bernoulli in 1696, who posted the problem in a journal as a challenge to the world's mathematcians at the time.
    And the answer given by Newton starts the study of calculus of variations.
    We are not going to fully solve the problem in this section, but instead just give a taste of the whole picture.\\
    Assuming $A$ is the origin and $B=(x_2,y_2)$
    Using energy conservation,
    $$\frac{1}{2}mv^2+mgy=0\implies v=\sqrt{-2gy}$$
    Take $y$ as a function of $x$, then we are aiming at finding the minima of the functional
    $$\tau[y]=\frac{1}{\sqrt{2g}}\int_0^{x_2}\frac{\sqrt{1+(y^\prime)^2}}{\sqrt{-y}}\,\mathrm dx$$
    subject to
    $$\begin{cases}
        y(0)=0\\
        y(x_2)=y_2
    \end{cases}$$
\end{example}
Another very famous and very useful example is geodesics.
\begin{example}[Geodesics]
    On a surface $\Sigma$, a geodesics $\gamma$ is the path of least length joining two given points.
    If $\Sigma$ is the Euclidean plane, it is well-known that $\gamma$ has to be a straight line.
    Let $D[y]$ denote the length of the curve given by $y$ and assume that the curve traverse as a function $y(x)$ from $x_1,x_2$, then minimising the following integral 
    $$D[y]=\int_{x_1}^{x_2}\sqrt{1+(y^\prime)^2}\,\mathrm dx$$
    subject to $y(x_1)=y_1,y(x_2)=y_2$ would be our aim.
\end{example}
In general, the calculus of variations focus on solving optimisation problems of a functional (i.e. functions from a space of functions to the reals)
$$F[y]=\int_{x_1}^{x_2}f(x,y(x),y^\prime(x))\,\mathrm dx$$
among all (sufficiently smooth) functions $y$ subject to initial conditions.
\begin{example}[Examples of Functionals]
    1. For a function $y(x)$, we can define the functional calculating the area under the curve by setting $f(x,y,y^\prime)=y$.\\
    2. We can also define the functional calculating length by
    $$f(x,y,y^\prime)=\sqrt{1+(y^\prime)^2}$$
\end{example}
Notationally, we write $C(\mathbb R)$ as the space of continuous functions from $\mathbb R$ to $\mathbb R$.
We write $C^k(\mathbb R)$ to denote the space of $k$-time differentiable functions with continuous $k^{th}$ derivative.
And $C^k_{\alpha,\beta}(\mathbb R)$ is the space of functions $[\alpha,\beta]\to\mathbb R$ that are $C^k$ on $[\alpha,\beta]$ and vanishes at $\alpha,\beta$.
One should know that these are these are all (infinite-dimensional) vector spaces over $\mathbb R$, the detail of analysis of which will be covered in functional analysis contexts.\\
The reason why the course is called ``variational principles'' instead of ``calculus of variations'' is because what we are studying are principles in nature and laws of physics that follow from extremising functionals.
\begin{example}[Fermat's Principle]
    Light that travels between two points follow the path that extremise the travel time.
\end{example}
\begin{example}[Principle of Least Action]
    Let $T,V$ be the kinetic and potential energies.
    We define the functional on a path
    $$S[\gamma]=\int_{t_1}^{t_2}(T-V)\,\mathrm dt$$
    then the path that a particle travels from $t=t_1$ to $t=t_2$ is the one that extremises $S$.
    Once we have developed our theory further, one can show that Newton's Second Law follows from this principle.\\
    Leibniz's take on this principle is that we live in ``the best of all possible worlds''.
    In some sense, we go from science to something like the field of theology.
    Of course, this is not in the scope of this course.\\
    Richard Feynman's take on this principle is that it is ``wrong''.
    Indeed, in quantum physics, nothing takes a definitive path, but every paths are possible with some probability.
    The principle under this framework, instead, is that the stationary path of a particle is, in fact, the interference along all paths.
\end{example}
What we shall show in this course are the follows:
First, we will see some necessary conditions for a function to be an extrema, like the Euler-Lagrange equations.
Secondly, we will bring this principle on problems in geometry, physics, and problem with constraints (like the isoperimetric inequality).
We will also talk about what is called the ``second variation'' as an analog of the second derivative.
\newpage
\part*{Lecture 2}
\section{Calculus for functions on $ \mathbb{R} ^n $}
Let $C^2(\mathbb R^2)$ denote the collection of twice differentiable functions $\mathbb R^n\to\mathbb R$ with continuous second derivatives.
\begin{definition}
    Let $f\in C^2(\mathbb R^n)$.
    A point $\mathbf{a}\in\mathbb R^n$ is stationary (or is a stationary point of $f$) if $\nabla f(\mathbf{a})=\mathbf{0}$.
\end{definition}
If we expand $f$ near one of its stationary points $\mathbf{x}=\mathbf{a}$, then
\begin{align*}
    f(\mathbf{x})&=f(\mathbf{a})+(\mathbf{x}-\mathbf{a})\cdot\nabla f+\frac{1}{2}(x_i-a_i)(x_j-a_j)\frac{\partial^2f}{\partial x_i\partial x_j}(\mathbf{a})+o(|\mathbf{x}-\mathbf{a}|^2)\\
    &=f(\mathbf{a})+\frac{1}{2}(\mathbf{x}-\mathbf{a})^\top H(\mathbf{a})(\mathbf{x}-\mathbf{a}),
\end{align*}
Where $H_{ij}(\mathbf{a})=\partial^2f/\partial x_i\partial x_j(\mathbf{a})$ is the Hessian, which is a real symmetric matrix for every $\mathbf{a}$.

Shift the origin to set $\mathbf{a}=\mathbf{0}$ and write $H=H(\mathbf{0})$.
As $H$ is real symmetric, we know that in another orthonormal basis $H$ can be diagonalised to $H=\operatorname{diag}(\lambda_1,\ldots,\lambda_n)$, hence in this new coordinate system,
$$f(\mathbf{x})-f(\mathbf{0})=\frac{1}{2}\sum_{i=1}^n\lambda_ix_i^2+o(|x|^2).$$

\begin{theorem}
    Let $\lambda_i$ be as before, then 
    \begin{enumerate}[(i)]
        \item If $\forall i,\lambda_i>0$, then $\mathbf{0}$ is a local minimum.
        \item If instead $\forall i,\lambda_i<0$, then $\mathbf{0}$ is a local maximum.
        \item If some of the eigenvalues are positive and some are negative, then $ \mathbf{0} $ is a saddle point.
        \item If for some $i$ has $\lambda_i=0$, then we need higher order terms. 
    \end{enumerate}
\end{theorem}

In the special case where $n=2$, we have $\det H=\lambda_1\lambda_2$ and $\operatorname{tr} H=\lambda_1+\lambda_2$, both of which are easier to calculate than the actual eigenvalues, so we can formulate a corollary to make things easier.
\begin{corollary}
    \begin{enumerate}[(i)]
        \item If $\det H>0,\operatorname{tr}H>0$, then it is a local minimum.
        \item If $\det H>0,\operatorname{tr}H<0$, it is a local maximum.
        \item If $\det H<0$, it is a saddle point.
        \item If $ \det H=0 $, then we have to look at higher order terms.
    \end{enumerate}
\end{corollary}
Despite these results, one should note that the actual (global) maxima and minima might occur at boundaries where we may not have $\nabla f=0$.

Also, there are plenty of examples of functions who only have saddle points.
Consider a harmonic function $f$ on $D\subset\mathbb R^2$, i.e. $f_{xx}+f_{yy}=0$, then at a stationary point we necessarily have $\operatorname{tr}H=0$, hence either $\det H=0$ or it is a saddle point.

But of course, the theorem and corollary can help a lot in classifying stationary points.
\begin{example}
    Let $f(x,y)=x^3+y^3-3xy$, then $\nabla f=(3x^2-3y,3y^2-3x)^\top$, so the only stationary points are $(x,y)=(0,0),(1,1)$.
    $$H=\begin{pmatrix}
        6x&-3\\
        -3&6y
    \end{pmatrix}$$
    Hence, by the above corollary, $(0,0)$ is a saddle point while $(1,1)$ is a minimum. Near $ f=0, f \sim -3xy $, and $f$ decreases on $y=x$, increases on $y=-x$.
    \begin{center}\vspace{1ex}
        \includegraphics[scale=0.1]{phase_1.1.jpeg}
    \end{center}
\end{example}

\subsection{Constraints and Lagrange Multipliers}
The optimisation problems in $\mathbb R^n$ are most likely coming with some sort of constraints.
We first illustrate this by way of an example
\begin{example}
    Find the circle centered at $(0,0)$ with smallest radius which intersects the parabola $y=x^2-1$.

    \begin{center}
        \includegraphics[scale=0.14]{best_fit_circ.jpeg}
    \end{center}
    
    Naturally, there are many ways to do this.
    One can solve the problem directly by simply substitution
    $$x^2+y^2=x^2+(x^2-1)^2=x^4-x^2+1$$
    which minimum is $3/4$ by either calculus or completing square.
    So the smallest radius is $\sqrt{3}/2$.

    The second method to solve this probelm is called the \textit{Lagrange multipliers}.
    Define a new function
    $$h(x,y,\lambda)=f(x,y)-\lambda g(x,y)$$
    where $f(x,y)=x^2+y^2$ is the function we want to minimise and $g(x,y)=y-x^2+1$ which vanishes if $(x,y)$ is on the parabola, i.e. $ g(x,y)=0 $ is the constraint.
    $\lambda$ here is called the \textit{Lagrange multiplier}.
    We now try to extremise $h$ over $x,y,\lambda$.
    Naturally, we solve the partial derivatives
    $$\begin{cases}
        \partial h/\partial x=2x+2\lambda x=0\\
        \partial h/\partial y=2y-\lambda=0\\
        \partial h/\partial\lambda=y-x^2+1=0
    \end{cases}$$
    Hence the stationary point of $h$ are at $(x,y)=(0,-1),(\pm 1/\sqrt{2},-1/2)$.
    Note that $ f(0,1)=1(\lambda=-2),\ f(\pm 1/\sqrt{2},-1/2)=3/4(\lambda=-1) $, so the result follows.
\end{example}

Why does the method of Lagrange multiplier work?

Geometrically, if we want to minimise a function $f$ subject to $g=0$.
Now $\nabla g$ is always perpendicular to $g=0$.
Then, suppose the minimum of $f$ is $c$, then the graph of $f(x)=c$ would touch $g(x)=0$, therefore their gradients there are parallel, so $\nabla f=\lambda\nabla g$ for some $\lambda$ and hence $\nabla h=0$ (note that $\partial h/\partial\lambda=0$ iff $g=0$), which is exactly the system we want to solve in the last part.

\begin{center}
    \includegraphics[scale=0.15]{best_fit_circ2.jpeg}
\end{center}

If there are many constraints $g_\alpha(\mathbf{x})=0$ of the minimisation problem of $f:\mathbb R^k\to\mathbb R$ where $\alpha=1,\dots,k$, we can define an analogous
$$h(x_1,\ldots,x_k,\lambda_1,\ldots,\lambda_k)=f(\mathbf{x})-\lambda_\alpha g_\alpha(\mathbf{x})$$
with $k$ Lagrange multipliers. Then we can extremise $h$ by solving 
\[
    \frac{\partial h}{\partial x_i} = 0,\quad \frac{\partial h}{\partial \lambda_\alpha}=0.  
\]
It follows by substitution back to $f$.
\newpage
\part*{Lecture 3}
\section{The Euler-Lagrange Equation}
\subsection{Derivation of the Equation}
We now move on to the most important theorem of the course, which gives a necessary condition to extremise a functional in the form
$$F[y]=\int_\alpha^\beta f(x,y,y^\prime)\,\mathrm dx$$
where $f$ is given, and $'$ denotes differentiation.
In contexts of geometry and physics, $y$ is most likely to carry the meaning of the trajectory of some point.
\begin{center}
    \includegraphics[scale=0.12]{euler-lagrange1.jpeg}
\end{center}
We first assume that a extremum $y$ exists, then when we apply a small perturbation $y\mapsto y+\epsilon\eta(x)$ with $\eta(\alpha)=\eta(\beta)=0$ to keep the endpoint fixed.
Now we want to compute $F[y+\epsilon\eta]$, but first of all we will need a lemma.

\begin{lemma}\label{fund_lemma}
    If $g:[\alpha,\beta]\to\mathbb R$ is continuous on $[\alpha,\beta]$ and
    $$\int_\alpha^\beta g(x)\eta(x)\,\mathrm dx=0$$
    for all $\eta\in C([\alpha,\beta])$ with $\eta(\alpha)=\eta(\beta)=0$, then $\forall x\in [\alpha,\beta],g(x)=0$.
\end{lemma}
\begin{proof}
    Assume for sake of contradiction that there exists some $\bar{x}$ on $(\alpha,\beta)$ such that $g(\bar{x})\neq 0$.
    wlog $g(\bar{x})>0$, then by continuity there is an interval $[x_1,x_2]\subset[\alpha,\beta]$ such that $\exists c>0,\forall x\in [x_1,x_2],g(x)>c$.
    Set
    $$\eta(x)=\begin{cases}
        (x-x_1)(x_2-x)\text{, if $x\in [x_1,x_2]$}\\
        0\text{, otherwise}
    \end{cases}$$
    Then
    \begin{align*}
        \int_\alpha^\beta g(x)\eta(x)\,\mathrm dx
        &=\int_{x_1}^{x_2}g(x)(x-x_1)(x_2-x)\,\mathrm dx\\
        &\ge\int_{x_1}^{x_2}c(x-x_1)(x_2-x)\,\mathrm dx>0.
    \end{align*}
    Contradiction.
    So $g=0$ on $(\alpha,\beta)$, and it is also zero at $\alpha,\beta$ by continuity, hence $g=0$ on $[\alpha,\beta]$.
\end{proof}
\begin{remark}
    The $\eta$ we have used above is called a bump function, which is $C^2$ as one can verify.
    One can also make a $C^k$ bump function by considering
    \footnote{It is also easy to construct a $C^{\infty}$ bump function.}
    $$\eta(x)=\begin{cases}
    ((x-x_1)(x_2-x))^{k+1}\text{, if $x\in [x_1,x_2]$}\\
    0\text{, otherwise}
\end{cases}$$
\end{remark}
Now back at $F[y+\epsilon\eta]$, we have
\begin{align*}
    F[y+\epsilon\eta]&=\int_\alpha^\beta f(x,y+\epsilon\eta,y^\prime+\epsilon\eta^\prime)\,\mathrm dx\\
    &=F[y]+\epsilon\int_\alpha^\beta\left( \frac{\partial f}{\partial y}\eta+\frac{\partial f}{\partial y^\prime}\eta^\prime \right)\,\mathrm dx+O(\epsilon^2)
\end{align*}
We will analysis the $O(\epsilon^2)$ remainder later.

For now, we just observe that for $y$ to be an extremum, the first-order term shall vanish, so we want something like 
\[
    \frac{\partial F[y+\epsilon \eta]}{\partial \epsilon} =0.
\]
Integrate the vanishing first-order coefficient by parts,
\begin{align*}
    0&=\left.\frac{\partial f}{\partial y^\prime}\eta\right|_\alpha^\beta+\int_\alpha^\beta\left( \frac{\partial f}{\partial y}\eta-\frac{\mathrm d}{\mathrm dx}\left( \frac{\partial f}{\partial y^\prime} \right)\eta \right)\,\mathrm dx\\
    &=\int_\alpha^\beta\left( \frac{\partial f}{\partial y}-\frac{\mathrm d}{\mathrm dx}\frac{\partial f}{\partial y^\prime} \right)\eta\,\mathrm dx
\end{align*}
By the preceding lemma, we must have
$$\boxed{\frac{\mathrm d}{\mathrm dx}\frac{\partial f}{\partial y^\prime}-\frac{\partial f}{\partial y}=0}$$
This is known as the Euler-Lagrange equation, which is the \textit{necessary condition} for an extremum. This equation is first developed in 1745 in a letter from Lagrange to Euler.
\begin{remark}\ 
    \begin{enumerate}
        \item The Euler-Lagrange equation is a second-order ODE with initial conditions $y(\alpha)=y_1,y(\beta)=y_2$.
        \item Sometimes the LHS is denoted $\delta F[y]/\delta y(x)$ and is called the functional derivative.
        Some author also write $\epsilon\eta=\delta y$, allowing one to write $F[y+\delta y]=F[y]+\delta F[y]$ where
        $$\delta F[y]=\int_\alpha^\beta\frac{\delta F[y]}{\delta y(x)}\delta y(x)\,\mathrm dx$$
        \item Other kinds of boundary conditions are possible, for example $ \frac{\partial f}{\partial y'}\Big|_{\alpha,\beta}=0  $.
        \item Be careful with derivatives as the notation can be a bit confusing.
        The $x,y,y^\prime$ are independent variables when we are talking about partial derivatives of $f(x,y,y^\prime)$.
        \item For any $h(x,y,y^\prime)$, a somewhat useful formula is
        $$\frac{\mathrm dh(x,y(x),y^\prime(x))}{\,\mathrm dx}=\frac{\partial h}{\partial x}+\frac{\partial h}{\partial y}y^\prime+\frac{\partial h}{\partial y^\prime}y^{\prime\prime}$$
        For example, for $f(x,y,y^\prime)=x((y^\prime)^2-y^2)$, we have $\mathrm df/\mathrm dx=(y^\prime)^2-y^2-2xyy^\prime+2y^{\prime\prime}y^\prime x$.
    \end{enumerate}
\end{remark}
\subsection{First Integrals of the Euler-Lagrange Equation}
Now we move on to solve the Euler-Lagrange equations, which is a second order ODE.
In some special cases, this is a quite easy thing to do.
In particular, if $f$ does not explicitly depend on some of its variables, then we can simplify the equation to something that is easier to solve.
These simplifications are often in the form of something being constant.
Expressions like these are called the first integrals of the equations.
Not only are they tools we can use to solve the equation, we can also view them as a conserved quantity of something that is described by a variational problem.
We will discuss the former in this section.
The latter will be mentioned later, when we discuss Noether's Theorem.

Assume that $f$ does not explicitly depend on $y$, then $\partial f/\partial y=0$, so the Euler-Lagrange equation can be rewritten as
$$\frac{\mathrm d}{\mathrm dx}\frac{\partial f}{\partial y^\prime}=0\implies \frac{\partial f}{\partial y^\prime}=\text{const,}$$
which is a first order ODE.
\begin{example}[Geodesics on the Euclidean plane]
    Consider the geodesics on the Euclidean plane.
    We know that we want to extremise the functional
    $$F[y]=\int_\alpha^\beta\sqrt{1+(y^\prime)^2}\,\mathrm dx$$
    In this case, the apparent $f$ does not depend on $y$, hence we can obtain the solution by just solving the first integral
    $$\frac{y^\prime}{\sqrt{1+(y^\prime)^2}}=\text{const.}$$
    Therefore $ y'=m $ and thus $y=mx+c$ for some constants $m,c$, which is our familiar formulation of a straight line, the geodesics of the plane.
\end{example}
\end{document}