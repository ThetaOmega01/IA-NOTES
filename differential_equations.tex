\documentclass[10pt]{article}
\pdfoutput=1 
\usepackage{NotesTeX,lipsum}

%\usepackage{showframe}

\title{\begin{center}{\Huge \textit{Differential Equations Notes}}\\{{\itshape Based on Lectures and "An Introduction to ODEs"}}\end{center}}
\author{$\theta\omega\theta$}
\affiliation{
Not in University of Cambridge\\
skipped some talks irrelevant to contents\\
}

\emailAdd{not telling you}

\begin{document}
	\maketitle
	\flushbottom
	\newpage
    \pagestyle{fancynotes}
    %main doc
    \part{Basic Calculus}
    \section{Lecture 1}
    \subsection{Differentiation}
	\begin{definition}[Derivative]
        The derivative of a function $f(x)$ wrt its argument $x$ is the function
        \[
            \frac{\mathrm{d}f}{\mathrm{d}x} = \lim_{h \to 0} \frac{f(x+h)-f(x)}{h} 
        .\]
        We define higher derivatives recursively by 
        \[
            \frac{\mathrm{d}^nf}{\mathrm{d}x^n} = \frac{\mathrm{d}}{\mathrm{d}x}\left( \frac{\mathrm{d}^{n-1}f}{\mathrm{d}x^{n-1}}  \right)   
        .\]
    \end{definition}
    For the derivative to exist, we need
    \[
        \lim_{h \to 0-} \frac{f(x+h)-f(x)}{h} = \lim_{h \to 0+} \frac{f(x+h)-f(x)}{h} 
    .\]
    
    Rules for differentiation:
    \begin{enumerate}
        \item \textbf{Chain rule}: $ (f(g(x)))' = f'(g(x))g'(x) $.
        \item \textbf{Product rule}: $ (u\cdot v)' = u\cdot v'+u'\cdot v $.
        \item \textbf{Leibniz's rule}: generalisation of product rule.\sidenote{There are multiple ways to prove, e.g. by induction.}
        \[
            \frac{\mathrm{d}^n}{\mathrm{d}x^n}(u\cdot v) = \sum_{k=0}^{n}\binom{n}{k}u^{(k)}v^{(n-k)}
        .\]
    \end{enumerate}
    \subsection{Order of magnitude}
    The goal is to compare the sizes of functions, in the vicinity of specific points.
    \begin{definition}[Little and Big o]
        We say $ f(x) = o(g(x)) $ as $x\to x_0$ if $ \lim_{x \to x_0} \frac{f(x)}{g(x)} = 0 $.

        We say $ f(x) = O(g(x)) $ as $x\to x_0$ if $ \exists M, \delta>0, \left| x-x_0 \right| <\delta \Rightarrow \left| f(x) \right| \le M \left| g(x) \right| . $ The infinite case is defined similarly.
    \end{definition}

    To find the tangent line to $f$ at $x_0$, note that 
    \[
        \begin{aligned}
             & \frac{\mathrm{d}f}{\mathrm{d}x}\Big|_{x=x_0} = \frac{f(x_0+h)-f(x_0)}{h}+ \frac{o(h)}{h} & \text{when $ h \to 0$} \\
             \Longrightarrow & f(x_0+h) = f(x_0)+\frac{\mathrm{d}f}{\mathrm{d}x}\Big|_{x=x_0} h + o(h)& \text{when $ h \to 0$} \\
        \end{aligned}
    \]
\end{document}