\documentclass[11pt]{article}
\usepackage{amssymb}
\usepackage{amsmath}
\usepackage{graphicx}
\usepackage{color}
\usepackage{xcolor}
\usepackage{enumerate}
\usepackage{multicol}

\usepackage[flushleft]{paralist}



\usepackage{geometry}
\geometry{%
  a4paper,
  lmargin=2cm,
  rmargin=2.5cm,
  tmargin=3.5cm,
  bmargin=2.5cm,
  footskip=12pt,
  headheight=24pt}


\newcommand{\comment}[1]{{\bf Comment} {\it #1}}
%\renewcommand{\comment}[1]{}
\newcommand{\mobius}{{M\"{o}bius }}
\newcommand{\bct}[1]{{\color{blue}#1}}
%\renewcommand{\comment}[1]{}
\newcommand{\rct}[1]{{\color{red}#1}}
\newcommand{\mcM}{\mathcal{M}}
\newcommand{\bbC}{\mathbb{C}}
\newcommand{\bbR}{\mathbb{R}}
\newcommand{\bbF}{\mathbb{F}}
\newcommand{\bbZ}{\mathbb{Z}}
\newcommand{\GL}{\mathrm{GL}}
\newcommand{\Or}{\mathrm{O}}
\newcommand{\PGL}{\mathrm{PGL}}
\newcommand{\PSL}{\mathrm{PSL}}
\newcommand{\PSO}{\mathrm{PSO}}
\newcommand{\PSU}{\mathrm{PSU}}
\newcommand{\SL}{\mathrm{SL}}
\newcommand{\SO}{\mathrm{SO}}
\newcommand{\Spin}{\mathrm{Spin}}
\newcommand{\Sp}{\mathrm{Sp}}
\newcommand{\SU}{\mathrm{SU}}
\newcommand{\U}{\mathrm{U}}
\newcommand{\Mat}{\mathrm{Mat}}

% Matrix algebras
\newcommand{\gl}{\mathfrak{gl}}
\newcommand{\ort}{\mathfrak{o}}
\newcommand{\so}{\mathfrak{so}}
\newcommand{\su}{\mathfrak{su}}
\newcommand{\uu}{\mathfrak{u}}
\renewcommand{\sl}{\mathfrak{sl}}
\DeclareMathOperator{\spn}{span}


\setlength{\parskip}{10pt}
\setlength{\parindent}{0pt}

\newlength{\qspace}
\setlength{\qspace}{15pt}


\newcounter{qnumber}
\setcounter{qnumber}{0}

\newenvironment{question}%
 {\vspace{\qspace}
  \begin{enumerate}[\bfseries 1\quad][10]%
    \setcounter{enumi}{\value{qnumber}}%
    \item%
 }
{
  \end{enumerate}
  \filbreak
  \stepcounter{qnumber}
 }


\newenvironment{questionparts}[1][1]%
 {
  \begin{enumerate}[\bfseries (i)]%
    \setcounter{enumii}{#1}
    \addtocounter{enumii}{-1}
    % \setlength{\itemsep}{5mm}
    \setlength{\parskip}{3pt}
 }
 {
  \end{enumerate}
 }



\DeclareMathOperator{\cosec}{cosec}
\DeclareMathOperator{\Var}{Var}
\DeclareMathOperator{\ord}{ord}
\DeclareMathOperator{\sym}{Sym}

\def\le{\leqslant}
\def\ge{\geqslant}
\def\leq{\leqslant}
\def\geq{\geqslant}

\def\var{{\rm Var}\,}

\newcommand{\ds}{\displaystyle}
\newcommand{\ts}{\textstyle}


\title{\textbf{Vectors and Matrices Important Examples}}
\author{\texttt{jt775}}
\date{\null}
\begin{document}
\maketitle
\vspace{-1.5cm}
\begin{question}
    \begin{questionparts}
        \item \textit{Example sheet 1 Q4, trignometric identity.}
    
        Express
        \[
        I=\frac{z^{5}-1}{z-1}
        \]
        as a polynomial in $z$. By considering the complex fifth root of unity $\omega$, obtain the four factors of $I$ linear in $z$. Hence write $I$ as the product of two real quadratic factors. By considering the term in $z^{2}$ in the identity so obtained for $I$, show that
        \[
        4 \cos \frac{\pi}{5} \sin \frac{\pi}{10}=1.
        \]
        \item \textit{Example sheet 1 Q7 \& Q9, geometry.}
        
        Show by vector methods that the altitudes of a triangle are concurrent.

        In $\triangle A B C$, let $\overrightarrow{A B}=\mathbf{u}, \overrightarrow{B C}=\mathbf{v}$ and $\overrightarrow{C A}=\mathbf{w}$. Show that
        \[
        \mathbf{u} \times \mathbf{v}=\mathbf{v} \times \mathbf{w}=\mathbf{w} \times \mathbf{u},
        \]
        and hence obtain the sine rule for $\triangle A B C$.

        Given any three vectors $\mathbf{p}, \mathbf{q}, \mathbf{r}$ such that
        \[
        \mathbf{p} \times \mathbf{q}=\mathbf{q} \times \mathbf{r}=\mathbf{r} \times \mathbf{p},
        \]
        and $|\mathbf{p} \times \mathbf{q}| \neq 0$, show that
        \[
        \mathbf{p}+\mathbf{q}+\mathbf{r}=\mathbf{0}.
        \]
    \end{questionparts}
\end{question}

\begin{question}\textit{Example sheet 1 Q12, vector equations.}
    
    Let $\mathbf{a}, \mathbf{b}, \mathbf{c}, \mathbf{d}$ be fixed vectors in three dimensions. For each of the following equations, find all solutions for $\mathbf{r}$ :
    \[
    \text { (i) } \mathbf{r}+\mathbf{r} \times \mathbf{d}=\mathbf{c} ; \quad \text { (ii) } \mathbf{r}+(\mathbf{r} \cdot \mathbf{a}) \mathbf{b}=\mathbf{c}.
    \]
    [In (ii), consider separately the cases $\mathbf{a} \cdot \mathbf{b} \neq-1$ and $\mathbf{a} \cdot \mathbf{b}=-1 .$]
\end{question}
\end{document}