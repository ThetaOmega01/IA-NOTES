\documentclass[11pt]{article}
\usepackage{amssymb}
\usepackage{amsmath}
\usepackage{graphicx}
\usepackage{color}
\usepackage{xcolor}
\usepackage{enumerate}
\usepackage{multicol}
\usepackage{hyperref}
\usepackage[flushleft]{paralist}



\usepackage{geometry}
\geometry{%
  a4paper,
  lmargin=2cm,
  rmargin=2.5cm,
  tmargin=3.5cm,
  bmargin=2.5cm,
  footskip=12pt,
  headheight=24pt}


\newcommand{\comment}[1]{{\bf Comment} {\it #1}}
%\renewcommand{\comment}[1]{}
\newcommand{\mobius}{{M\"{o}bius }}
\newcommand{\bct}[1]{{\color{blue}#1}}
%\renewcommand{\comment}[1]{}
\newcommand{\rct}[1]{{\color{red}#1}}
\newcommand{\mcM}{\mathcal{M}}
\newcommand{\bbC}{\mathbb{C}}
\newcommand{\bbR}{\mathbb{R}}
\newcommand{\bbF}{\mathbb{F}}
\newcommand{\GL}{\mathrm{GL}}
\newcommand{\Or}{\mathrm{O}}
\newcommand{\PGL}{\mathrm{PGL}}
\newcommand{\PSL}{\mathrm{PSL}}
\newcommand{\PSO}{\mathrm{PSO}}
\newcommand{\PSU}{\mathrm{PSU}}
\newcommand{\SL}{\mathrm{SL}}
\newcommand{\SO}{\mathrm{SO}}
\newcommand{\Spin}{\mathrm{Spin}}
\newcommand{\Sp}{\mathrm{Sp}}
\newcommand{\SU}{\mathrm{SU}}
\newcommand{\U}{\mathrm{U}}
\newcommand{\Mat}{\mathrm{Mat}}

% Matrix algebras
\newcommand{\gl}{\mathfrak{gl}}
\newcommand{\ort}{\mathfrak{o}}
\newcommand{\so}{\mathfrak{so}}
\newcommand{\su}{\mathfrak{su}}
\newcommand{\uu}{\mathfrak{u}}
\renewcommand{\sl}{\mathfrak{sl}}
\DeclareMathOperator{\spn}{span}


\setlength{\parskip}{10pt}
\setlength{\parindent}{0pt}

\newlength{\qspace}
\setlength{\qspace}{15pt}


\newcounter{qnumber}
\setcounter{qnumber}{0}

\newenvironment{question}%
 {\vspace{\qspace}
  \begin{enumerate}[\bfseries 1\quad][10]%
    \setcounter{enumi}{\value{qnumber}}%
    \item%
 }
{
  \end{enumerate}
  \filbreak
  \stepcounter{qnumber}
 }


\newenvironment{questionparts}[1][1]%
 {
  \begin{enumerate}[\bfseries (i)]%
    \setcounter{enumii}{#1}
    \addtocounter{enumii}{-1}
    % \setlength{\itemsep}{5mm}
    \setlength{\parskip}{3pt}
 }
 {
  \end{enumerate}
 }



\DeclareMathOperator{\cosec}{cosec}
\DeclareMathOperator{\Var}{Var}
\DeclareMathOperator{\ord}{ord}
\DeclareMathOperator{\sym}{Sym}

\def\le{\leqslant}
\def\ge{\geqslant}


\def\var{{\rm Var}\,}

\newcommand{\ds}{\displaystyle}
\newcommand{\ts}{\textstyle}


\title{\textbf{Groups Important Examples}}
\author{\texttt{jt775}}
\date{\null}
\begin{document}
\maketitle
\vspace{-1.5cm}

\begin{question}\textit{Example sheet 1 Q4, nth root of unity.}
    
    Let $S$ be a finite non-empty set of non-zero complex numbers which is closed under multiplication. Show that $S$ is a subset of the set $\{z \in \mathbb{C}:|z|=1\}$. Show that $S$ is a group with respect to multiplication, and deduce that for some $n \in \mathbb{N}, S$ is the set of $n$ th roots of unity, that is, $S=\left\{e^{2 \pi i k / n}: k=0,1, \ldots, n-1\right\}$.
\end{question}

\begin{question}\textit{Example sheet 1 Q7 \& Q8, minimal generating set, analogy to Wilson's theorem.}

  Let $G$ be a group in which every element other than the identity has order two. Show that $G$ is abelian. Show that if $G$ is also finite, then the order of $G$ is a power of 2. Can such a group be infinite?

  Let $G$ be a group of even order. Show that $G$ contains an element of order two. Can a group have exactly two elements of order two? 
  
  Which (not necessarily finite) groups have a non-zero even number of elements of order two? \textit{[Hint: it must be infinite.]}
\end{question}

\begin{question}\textit{Example sheet 1 Q12, properties of homomorphisms.}

  \begin{questionparts}
    \item Find all homomorphisms $ D_{2n}\to C_n $.
  \end{questionparts}

  \textit{With some additional problems, Example sheet 1 Q15 \& Q16.}

  \begin{questionparts}
    \setcounter{enumii}{1}
    \item The Pr\"{u}fer $p$-group is defined by
    \[
      \mathbb{Z}_{p^\infty}=\bigcup_{k=0}^\infty \mathbb{Z}_{p^k} \text{ where } \mathbb{Z}_{p^k}= \{e^{\frac{2 i \pi m}{p^k}} \ | \ 0 \le m \le p^k-1\}.
    \]
    Show that $ \mathbb{Z}_{p^\infty} $ is a group.
    Show that all proper subgroups of $ \mathbb{Z}_{p^\infty} $ are finite.

    There are other examples of infinite groups whose proper subgroups are finite, see the \href{https://planetmath.org/tarskigroup}{Tarski monster group.}
    \item Let $n$ be a positive integer. Construct a group $G$ such that, for any group $H$ that can be generated by $n$ elements, there is a surjective homomorphism from $G$ to $H$.
  \end{questionparts}
\end{question}

\end{document}

