\documentclass[11pt]{article}
\usepackage{amssymb}
\usepackage{amsmath}
\usepackage{graphicx}
\usepackage{color}
\usepackage{xcolor}
\usepackage{enumerate}
\usepackage{multicol}

\usepackage[flushleft]{paralist}



\usepackage{geometry}
\geometry{%
  a4paper,
  lmargin=2cm,
  rmargin=2.5cm,
  tmargin=3.5cm,
  bmargin=2.5cm,
  footskip=12pt,
  headheight=24pt}


\newcommand{\comment}[1]{{\bf Comment} {\it #1}}
%\renewcommand{\comment}[1]{}
\newcommand{\mobius}{{M\"{o}bius }}
\newcommand{\bct}[1]{{\color{blue}#1}}
%\renewcommand{\comment}[1]{}
\newcommand{\rct}[1]{{\color{red}#1}}
\newcommand{\mcM}{\mathcal{M}}
\newcommand{\bbC}{\mathbb{C}}
\newcommand{\bbR}{\mathbb{R}}
\newcommand{\bbF}{\mathbb{F}}
\newcommand{\rmd}{\mathrm{d}}
\newcommand{\GL}{\mathrm{GL}}
\newcommand{\Or}{\mathrm{O}}
\newcommand{\PGL}{\mathrm{PGL}}
\newcommand{\PSL}{\mathrm{PSL}}
\newcommand{\PSO}{\mathrm{PSO}}
\newcommand{\PSU}{\mathrm{PSU}}
\newcommand{\SL}{\mathrm{SL}}
\newcommand{\SO}{\mathrm{SO}}
\newcommand{\Spin}{\mathrm{Spin}}
\newcommand{\Sp}{\mathrm{Sp}}
\newcommand{\SU}{\mathrm{SU}}
\newcommand{\U}{\mathrm{U}}
\newcommand{\Mat}{\mathrm{Mat}}

% Matrix algebras
\newcommand{\gl}{\mathfrak{gl}}
\newcommand{\ort}{\mathfrak{o}}
\newcommand{\so}{\mathfrak{so}}
\newcommand{\su}{\mathfrak{su}}
\newcommand{\uu}{\mathfrak{u}}
\renewcommand{\sl}{\mathfrak{sl}}
\DeclareMathOperator{\spn}{span}


\setlength{\parskip}{10pt}
\setlength{\parindent}{0pt}

\newlength{\qspace}
\setlength{\qspace}{15pt}


\newcounter{qnumber}
\setcounter{qnumber}{0}

\newenvironment{question}%
 {\vspace{\qspace}
  \begin{enumerate}[\bfseries 1\quad][10]%
    \setcounter{enumi}{\value{qnumber}}%
    \item%
 }
{
  \end{enumerate}
  \filbreak
  \stepcounter{qnumber}
 }


\newenvironment{questionparts}[1][1]%
 {
  \begin{enumerate}[\bfseries (i)]%
    \setcounter{enumii}{#1}
    \addtocounter{enumii}{-1}
    % \setlength{\itemsep}{5mm}
    \setlength{\parskip}{3pt}
 }
 {
  \end{enumerate}
 }



\DeclareMathOperator{\cosec}{cosec}
\DeclareMathOperator{\Var}{Var}
\DeclareMathOperator{\ord}{ord}
\DeclareMathOperator{\sym}{Sym}

\def\le{\leqslant}
\def\ge{\geqslant}


\def\var{{\rm Var}\,}

\newcommand{\ds}{\displaystyle}
\newcommand{\ts}{\textstyle}


\title{\textbf{Differential Equations Revision}}
\author{\texttt{jt775}}
\date{\null}
\begin{document}
\maketitle
\vspace{-1.5cm}
\begin{question}
    State and prove the Leibniz's rule. State Taylor's theorem. State and prove L'H\^{o}pital's rule.

    Show that for $ f=f(x,y) $ we have $ \rmd f=\partial f/\partial x\ \rmd x+\partial f/\partial y\ \rmd y  $. Use change of variables to get partial derivatives in polar coordinates.

    For a surface $ f(x,y,z)=c $, compute $ \partial z/\partial x  $.

    Let $ I(\alpha) = \int_{a(\alpha)}^{b(\alpha)} f(x,\alpha) \,\mathrm{d}x$. Compute $ \rmd I/ \rmd \alpha  $.
\end{question}
\begin{question}
  Describe how to define $e$ via a solution of a differential equation. What is an eigenfunction?

  State the 4 rules for linear ODEs.

  In a sample of rock, isotope A decays to isotope B at a rate proportional to $a$, the number of nuclei of A. B decays to C at a rate proportional to $b$, the number of nuclei of B. Find $b(t)/a(t)$.

  Deduce the integrating factor of first order ODEs of non-constant coefficients.

  ({\color{red}$\bigstar$}) Use discrete method and series method to solve $ 5y'-3y=0 $.

  What does it mean for an ODE to be exact? State the condition for an ODE to be exact. Hence solve $ 6y(y-x)y'+(2x-3y^2)=0 $.
\end{question}
\begin{question}
  Sketch the isoclines of $ \frac{\mathrm{d}y}{\mathrm{d}t}=t(1-y^2) $. Solve the equation.

  What is an fixed point? Use perturbation analysis to determine stabilities of the fixed points of $ \frac{\mathrm{d}y}{\mathrm{d}t}=t(1-y^2) $. For autonomous DEs?

  Use phase portrait to analyse $ \frac{\mathrm{d}c}{\mathrm{d}t}=\lambda(a_0-c)(b_0-c)  $ and $ \frac{\mathrm{d}y}{\mathrm{d}t} =(\alpha-\beta)y-\gamma y^2 $. What problems do they refer to?

  What is a fixed point in a discrete equation? State and prove the criterion of stability. The Logistic map is given by $ \frac{x_{n+1}-x_n}{\Delta t}=\lambda x_n-\gamma x_{n}^2 $. Consider the simplified version $ x_{n+1}=rx_n(1-x_n)=f(x_n) $. Find the fixed points and analyse their stabilities.
\end{question}

\begin{question}
  What does it mean for a set of functions to be \textit{linearly independent}?

  Consider the second order differential equation 
  \[
    \left( a\frac{\mathrm{d}^2}{\mathrm{d}x^2}+b \frac{\mathrm{d}}{\mathrm{d}x}+c \right)y=0,
  \]
  where $ a,b,c $ are constants. Explain how to solve it by the method of auxiliary equation. Explan the use of detuning when the roots are repeated.

  Consider the second order differential equation 
  \[
    \frac{\mathrm{d}^2y}{\mathrm{d}x^2}+p(x)\frac{\mathrm{d}y}{\mathrm{d}x} +q(x)y=0.
  \]
  Given one solution $y_1$, use reduction of order to find a second solution $y_2$.
\end{question}

\begin{question}
  What is a solution space? Consider $ y''+4y=0 $, explain why finding two independent solutions is enough for finding all solutions. Hence extend to $n$-dimensional cases.

  State the definition of a \textit{Wronskian}. Explain why solution vectors are linearly independent if their Wronskian is non-zero. Explain how to find the original differential equation given independent solutions.
\end{question}
\end{document}