\documentclass[11pt]{article}
\usepackage{amssymb}
\usepackage{amsmath}
\usepackage{graphicx}
\usepackage{color}
\usepackage{xcolor}
\usepackage{enumerate}
\usepackage{multicol}

\usepackage[flushleft]{paralist}



\usepackage{geometry}
\geometry{%
  a4paper,
  lmargin=2cm,
  rmargin=2.5cm,
  tmargin=3.5cm,
  bmargin=2.5cm,
  footskip=12pt,
  headheight=24pt}


\newcommand{\comment}[1]{{\bf Comment} {\it #1}}
%\renewcommand{\comment}[1]{}
\newcommand{\mobius}{{M\"{o}bius }}
\newcommand{\bct}[1]{{\color{blue}#1}}
%\renewcommand{\comment}[1]{}
\newcommand{\rct}[1]{{\color{red}#1}}
\newcommand{\mcM}{\mathcal{M}}
\newcommand{\bbC}{\mathbb{C}}
\newcommand{\bbR}{\mathbb{R}}
\newcommand{\bbF}{\mathbb{F}}
\newcommand{\GL}{\mathrm{GL}}
\newcommand{\Or}{\mathrm{O}}
\newcommand{\PGL}{\mathrm{PGL}}
\newcommand{\PSL}{\mathrm{PSL}}
\newcommand{\PSO}{\mathrm{PSO}}
\newcommand{\PSU}{\mathrm{PSU}}
\newcommand{\SL}{\mathrm{SL}}
\newcommand{\SO}{\mathrm{SO}}
\newcommand{\Spin}{\mathrm{Spin}}
\newcommand{\Sp}{\mathrm{Sp}}
\newcommand{\SU}{\mathrm{SU}}
\newcommand{\U}{\mathrm{U}}
\newcommand{\Mat}{\mathrm{Mat}}

% Matrix algebras
\newcommand{\gl}{\mathfrak{gl}}
\newcommand{\ort}{\mathfrak{o}}
\newcommand{\so}{\mathfrak{so}}
\newcommand{\su}{\mathfrak{su}}
\newcommand{\uu}{\mathfrak{u}}
\renewcommand{\sl}{\mathfrak{sl}}
\DeclareMathOperator{\spn}{span}
\DeclareMathOperator{\tr}{tr}


\setlength{\parskip}{10pt}
\setlength{\parindent}{0pt}

\newlength{\qspace}
\setlength{\qspace}{15pt}


\newcounter{qnumber}
\setcounter{qnumber}{0}

\newenvironment{question}%
 {\vspace{\qspace}
  \begin{enumerate}[\bfseries 1\quad][10]%
    \setcounter{enumi}{\value{qnumber}}%
    \item%
 }
{
  \end{enumerate}
  \filbreak
  \stepcounter{qnumber}
 }


\newenvironment{questionparts}[1][1]%
 {
  \begin{enumerate}[\bfseries (i)]%
    \setcounter{enumii}{#1}
    \addtocounter{enumii}{-1}
    % \setlength{\itemsep}{5mm}
    \setlength{\parskip}{3pt}
 }
 {
  \end{enumerate}
 }



\DeclareMathOperator{\cosec}{cosec}
\DeclareMathOperator{\Var}{Var}
\DeclareMathOperator{\ord}{ord}
\DeclareMathOperator{\sym}{Sym}

\def\le{\leqslant}
\def\ge{\geqslant}


\def\var{{\rm Var}\,}

\newcommand{\ds}{\displaystyle}
\newcommand{\ts}{\textstyle}


\title{\textbf{Vectors and Matrices Revision}}
\author{\texttt{jt775}}
\date{\null}
\begin{document}
\maketitle
\vspace{-1.5cm}
\begin{question}
    Give the definition of a field and show that $\bbC$ is a field. State and prove the triangular inequality and its equivalent forms.

    Write down the complex definitions of $ \exp, \sin, \cos $. Explain how to take log of a complex number. Give the definition of complex powers.

    What is an $n$th root of unity?

    Write down the (complex numbers and vector) equations of a generic line and a generic circle. Solve $ |\mathbf{r}|^2+\mathbf{r}\cdot \mathbf{a}=k $ and $ \mathbf{r}+\mathbf{a} \times (\mathbf{b}\times \mathbf{r}) =\mathbf{c} $. Deduce the cosine rule.

    Give definitions of the Kronecker delta and Levi-Civia epsilon. Show that $ \epsilon_{ijk}\epsilon_{pqk}= \delta_{ip}\delta_{jq}-\delta_{iq}\delta_{jp}$.
\end{question}
\begin{question}
  State and prove Cauchy-Schwarz inequality and hence the triangular inequality.

  State the definitions of a vector space and linear independence. Show that $ \{\mathbf{a},\mathbf{b},\mathbf{c}\} $ is linearly independent if $ [\mathbf{a},\mathbf{b},\mathbf{c}]\neq 0 $. By considering a suitable inner product, show that $ \left| \int_{0}^{1} f(x)g(x) \,\mathrm{d}x \right| \le \left( \int_{0}^{1} f(x) \,\mathrm{d}x \right)^{1/2}\left( \int_{0}^{1} g(x) \,\mathrm{d}x \right)^{1/2} $ for $f$ smooth and $ f(0)=f(1)=0 $.

  State the definitions of a basis and the dimension of a vector space. You should show that the dimension is well-defined. Show also that we can extend a linearly independent set to a basis, and contract a spanning set to a basis. 
  
  By considering $ s_n=\sqrt{2} \sin (n \pi x) $, Show that $ V=\left\{ f:[0,1] \to \mathbb{R} :f(0)=f(1)=0\right\} $ of smooth functions is infinite-dimensional.

  State the defintion and propoerties of the complex inner product.
\end{question}
\begin{question}
  Let $ V,W $ be vector spaces.
  \begin{questionparts}
    \item What is a linear map $ V \to W $? Show that the image and kernel of a linear map are subspaces. Show that a linear map is determined by its action on a basis. State and prove the Rank-Nullity Theorem.
    \item For $ V=W=\mathbb{R}^2 $, give general forms of a rotation and a reflection. For $ V=W=\mathbb{R}^{3} $, give general forms of a rotation and a reflection. Give also the general forms of a dilation and a shear.
    \item Let $ \{\mathbf{e}_i\}_{i=1}^n,\ \{\mathbf{f}_j\}_{j=1}^m $ be bases of $V,W$ respectively, and $T:V\to W$ be a linear map. Define the matrix of $T$. State the rules of matrix multiplication.
  \end{questionparts}
\end{question}
\newpage
\begin{question}
  Let $ A\in \mcM_{m \times n} $. What is a left inverse of $A$? A right inverse? Show that if $m=n$ then the left and right inverses are identical. Give the definition of $ A^{-1} $. Compute the inverse of a generic invertible $ 2 \times 2 $ matrix. Verify that the inverse of a rotation of angle $ \theta $ in $ \mathbb{R}^3 $ is the rotation of angle $ -\theta $.
\end{question}
\begin{question}
  Give the definition of the transpose of $A$. State the rules of transpose. Explain what it means for a matrix to be \textit{symmetric} or \textit{anti-symmetric}. Show that every matrix can be written as a combination of a symmetric matrix and an anti-symmetrix matrix.

  Give the definition of the Hermitian conjugate and trace of $A$. Explain what it means for a matrix to be \textit{Hermitian} or \textit{anti-Hermitian}. Show that:
  \begin{enumerate}[(1)]
    \item $ \tr(\alpha M+ \beta N)=\alpha \tr(M)+\beta \tr(N) $.
    \item $ \tr(MN)=\tr(NM) $ for $ M\in \mathcal{M}_{m\times n}, N\in \mathcal{M}_{n\times m} $.
    \item $ \tr(M^{\top})=\tr(M) $.
    \item $ \tr(I)=\delta_{ii}=n $ if $ I\in \mathcal{M}_{n\times n} $.
  \end{enumerate}
  Show that if $S\in \mcM_{n\times n}$ is symmetric, then $ S=T+\frac{1}{n}\tr(S) I $, where $T$ is traceless and $I$ is the identity.

  Explain what it means for a matrix to be orthogonal or unitary.. Show that the rows and columns of an orthogonal matrix are orthonormal. Show that orthogonal matrices preserve inner products. Find all $ 2\times 2 $ orthogonal matrices.
\end{question}
\end{document}