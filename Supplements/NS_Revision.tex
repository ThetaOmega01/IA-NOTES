\documentclass[11pt]{article}
\usepackage{amssymb}
\usepackage{amsmath}
\usepackage{graphicx}
\usepackage{color}
\usepackage{xcolor}
\usepackage{enumerate}
\usepackage{multicol}

\usepackage[flushleft]{paralist}



\usepackage{geometry}
\geometry{%
  a4paper,
  lmargin=2cm,
  rmargin=2.5cm,
  tmargin=3.5cm,
  bmargin=2.5cm,
  footskip=12pt,
  headheight=24pt}


\newcommand{\comment}[1]{{\bf Comment} {\it #1}}
%\renewcommand{\comment}[1]{}
\newcommand{\mobius}{{M\"{o}bius }}
\newcommand{\bct}[1]{{\color{blue}#1}}
%\renewcommand{\comment}[1]{}
\newcommand{\rct}[1]{{\color{red}#1}}
\newcommand{\mcM}{\mathcal{M}}
\newcommand{\bbC}{\mathbb{C}}
\newcommand{\bbR}{\mathbb{R}}
\newcommand{\bbF}{\mathbb{F}}
\newcommand{\bbZ}{\mathbb{Z}}
\newcommand{\GL}{\mathrm{GL}}
\newcommand{\Or}{\mathrm{O}}
\newcommand{\PGL}{\mathrm{PGL}}
\newcommand{\PSL}{\mathrm{PSL}}
\newcommand{\PSO}{\mathrm{PSO}}
\newcommand{\PSU}{\mathrm{PSU}}
\newcommand{\SL}{\mathrm{SL}}
\newcommand{\SO}{\mathrm{SO}}
\newcommand{\Spin}{\mathrm{Spin}}
\newcommand{\Sp}{\mathrm{Sp}}
\newcommand{\SU}{\mathrm{SU}}
\newcommand{\U}{\mathrm{U}}
\newcommand{\Mat}{\mathrm{Mat}}

% Matrix algebras
\newcommand{\gl}{\mathfrak{gl}}
\newcommand{\ort}{\mathfrak{o}}
\newcommand{\so}{\mathfrak{so}}
\newcommand{\su}{\mathfrak{su}}
\newcommand{\uu}{\mathfrak{u}}
\renewcommand{\sl}{\mathfrak{sl}}
\DeclareMathOperator{\spn}{span}


\setlength{\parskip}{10pt}
\setlength{\parindent}{0pt}

\newlength{\qspace}
\setlength{\qspace}{15pt}


\newcounter{qnumber}
\setcounter{qnumber}{0}

\newenvironment{question}%
 {\vspace{\qspace}
  \begin{enumerate}[\bfseries 1\quad][10]%
    \setcounter{enumi}{\value{qnumber}}%
    \item%
 }
{
  \end{enumerate}
  \filbreak
  \stepcounter{qnumber}
 }


\newenvironment{questionparts}[1][1]%
 {
  \begin{enumerate}[\bfseries (i)]%
    \setcounter{enumii}{#1}
    \addtocounter{enumii}{-1}
    % \setlength{\itemsep}{5mm}
    \setlength{\parskip}{3pt}
 }
 {
  \end{enumerate}
 }



\DeclareMathOperator{\cosec}{cosec}
\DeclareMathOperator{\Var}{Var}
\DeclareMathOperator{\ord}{ord}
\DeclareMathOperator{\sym}{Sym}

\def\le{\leqslant}
\def\ge{\geqslant}
\def\leq{\leqslant}
\def\geq{\geqslant}

\def\var{{\rm Var}\,}

\newcommand{\ds}{\displaystyle}
\newcommand{\ts}{\textstyle}


\title{\textbf{Numbers and Sets Revision}}
\author{\texttt{jt775}}
\date{\null}
\begin{document}
\maketitle
\vspace{-1.5cm}
\begin{question}
    State the Peano Axioms of $ \mathbb{N} $. Hence define addition and show that it is commutative, associative, and satisfies the cancellation law. Define the ordering of $ \mathbb{N} $ and show that it is reflective, transitive, anti-symmetric, and order-preserving. State and prove the law of trichotomy. Do the similar process to multiplications.

    Prove the law of strong induction by induction. State the well-ordering principle.

    Prove that $ a\times b=b \Leftrightarrow a=0 \lor b=0 $.
\end{question}
\begin{question}
  Show that every natural number can be expressed as a product of primes. 
  
  State and prove the Euclid Algorithm. State and prove B\'{e}zout Theorem.

  Prove that if $p$ is a prime then $ p|ab \Rightarrow p|a \lor p|b $. Hence state and prove the Fundamental Theorem of Arithmetic. Is this obvious in any number fields?
\end{question}
\begin{question}
  Prove that the modular inverse is unique in $ \bbZ_n $. Prove that every element has an inverse in $ \bbZ_p $, where $p$ is prime. Hence show that $a$ has an inverse in $ \bbZ_n $ if and only if $ \gcd(a,n)=1 $.

  State and prove Fermat's Little Theorem. State and prove Fermat-Euler Theorem.

  Show that if $p$ is a prime then $ x^2 \equiv 1 \pmod{p} \Rightarrow x \equiv \pm 1 \pmod p $. Hence state and prove Wilson's Theorem.

  Prove that if $p$ is an odd prime, then $ x^2 +1\equiv 0 \pmod p $ has a solution if and only if $ p \equiv 1 \pmod 4 $.

  State and prove Chinese Remainder Theorem. State Generalised CRT. Solve the following congruences:
  \begin{multicols}{3}
    \begin{enumerate}
      \item $ 7x \equiv 4\pmod{30} $;
      \item $ 3x \equiv 9 \pmod{28} $;
      \item $ 10x \equiv 12 \pmod{{34}} $;
      \item $ \begin{cases}
      x \equiv 6 \pmod{17} \\
      x \equiv 2 \pmod{19} \\
      \end{cases}$;
      \item $ \begin{cases}
        x \equiv 6 \pmod{34} \\
        x \equiv 2 \pmod{36} \\
        \end{cases} $.
    \end{enumerate}
  \end{multicols}

  Explain the procedures of RSA Encryption.
\end{question}

\begin{question}
  Prove that $ \sqrt{2} $ is irrational by considering the number $ cx+d $, where $ x^2=2 $ and $c,d\in\bbZ$.

  Give definitions of supremum and infimum.

  What does it mean for a set $ S \subseteq \mathbb{R} $ to be \textit{dense}? Prove that $ \mathbb{Q} $ and $ \mathbb{Q}^\complement $ are dense.
\end{question}
\end{document}