\documentclass[11pt]{article}
\usepackage{amssymb}
\usepackage{amsmath}
\usepackage{graphicx}
\usepackage{color}
\usepackage{xcolor}
\usepackage{enumerate}
\usepackage{multicol}

\usepackage[flushleft]{paralist}



\usepackage{geometry}
\geometry{%
  a4paper,
  lmargin=2cm,
  rmargin=2.5cm,
  tmargin=3.5cm,
  bmargin=2.5cm,
  footskip=12pt,
  headheight=24pt}


\newcommand{\comment}[1]{{\bf Comment} {\it #1}}
%\renewcommand{\comment}[1]{}
\newcommand{\mobius}{{M\"{o}bius }}
\newcommand{\bct}[1]{{\color{blue}#1}}
%\renewcommand{\comment}[1]{}
\newcommand{\rct}[1]{{\color{red}#1}}
\newcommand{\mcM}{\mathcal{M}}
\newcommand{\bbC}{\mathbb{C}}
\newcommand{\bbR}{\mathbb{R}}
\newcommand{\bbF}{\mathbb{F}}
\newcommand{\GL}{\mathrm{GL}}
\newcommand{\Or}{\mathrm{O}}
\newcommand{\PGL}{\mathrm{PGL}}
\newcommand{\PSL}{\mathrm{PSL}}
\newcommand{\PSO}{\mathrm{PSO}}
\newcommand{\PSU}{\mathrm{PSU}}
\newcommand{\SL}{\mathrm{SL}}
\newcommand{\SO}{\mathrm{SO}}
\newcommand{\Spin}{\mathrm{Spin}}
\newcommand{\Sp}{\mathrm{Sp}}
\newcommand{\SU}{\mathrm{SU}}
\newcommand{\U}{\mathrm{U}}
\newcommand{\Mat}{\mathrm{Mat}}

% Matrix algebras
\newcommand{\gl}{\mathfrak{gl}}
\newcommand{\ort}{\mathfrak{o}}
\newcommand{\so}{\mathfrak{so}}
\newcommand{\su}{\mathfrak{su}}
\newcommand{\uu}{\mathfrak{u}}
\renewcommand{\sl}{\mathfrak{sl}}
\DeclareMathOperator{\spn}{span}


\setlength{\parskip}{10pt}
\setlength{\parindent}{0pt}

\newlength{\qspace}
\setlength{\qspace}{15pt}


\newcounter{qnumber}
\setcounter{qnumber}{0}

\newenvironment{question}%
 {\vspace{\qspace}
  \begin{enumerate}[\bfseries 1\quad][10]%
    \setcounter{enumi}{\value{qnumber}}%
    \item%
 }
{
  \end{enumerate}
  \filbreak
  \stepcounter{qnumber}
 }


\newenvironment{questionparts}[1][1]%
 {
  \begin{enumerate}[\bfseries (i)]%
    \setcounter{enumii}{#1}
    \addtocounter{enumii}{-1}
    % \setlength{\itemsep}{5mm}
    \setlength{\parskip}{3pt}
 }
 {
  \end{enumerate}
 }



\DeclareMathOperator{\cosec}{cosec}
\DeclareMathOperator{\Var}{Var}
\DeclareMathOperator{\ord}{ord}
\DeclareMathOperator{\sym}{Sym}

\def\le{\leqslant}
\def\ge{\geqslant}


\def\var{{\rm Var}\,}

\newcommand{\ds}{\displaystyle}
\newcommand{\ts}{\textstyle}


\title{\textbf{Groups Revision}}
\author{\texttt{jt775}}
\date{\null}
\begin{document}
\maketitle
\vspace{-1.5cm}
\begin{question}
  Explain what a group is. Show that the followings are groups:

\begin{multicols}{2}
  \begin{enumerate}
    \item $G = \left\{ e\right\}$, the trivial group,
    \item $ G = \left\{ \text{symmetries of } \triangle \right\} $,
    \item $ (\mathbb{Z} , +) $,
    \item $ (\mathbb{R} ,+), (\mathbb{Q} , +), (\mathbb{C} , +) $,
    \item $ \mathbb{R}^* = \mathbb{R} \setminus \left\{ 0\right\}; (\mathbb{R}^*, \times) $,
    \item $ (\mathbb{Z}_n, + \pmod n), \mathbb{Z}_n = \left\{ 0,1,\dots, n-1\right\} $,
    \item Vector spaces with addition of vectors,
    \item $ (\rm{GL}_2(\mathbb{R}), \text{matrix multiplication}) $, set of invertible $2\times 2$ matrices,
\end{enumerate}
\end{multicols}

Show that the followings are not groups:

\begin{multicols}{2}
  \begin{enumerate}
    \item $ (\mathbb{Z}_n, +) $,
    \item $ (\mathbb{Z} , \times) $,
    \item $ (\mathbb{R} , *) $, where $ r*s = r^2 s $,
    \item $ (\mathbb{N}, *), n*m = |n-m| $.
  \end{enumerate}
\end{multicols}

Let $G$ be a group. Show that:

\begin{multicols}{3}
  \begin{enumerate}
    \item The identity is unique.
    \item The inverse is unique.
    \item $ gh=g \land hg=g \Rightarrow h=e $.
    \item $ gh=e \Rightarrow hg=e, h=g^{-1} $.
    \item $ (g^{-1})^{-1}=g $.
\end{enumerate}
\end{multicols}

Explain what an abelian group is. Define a subgroup of a group. Show that the identity is unique among all subgroups. State and prove the subgroup test.
Find all subgroups of $ (\mathbb{Z} ,+) $.

Explain what it means for a group $G$ to be generated by $X$. Give the general form of elements in $ \langle X \rangle  $. Hence define the generating set of $G$. Is it unique?
\end{question}

\begin{question}
  Let $ G,\ H $ be groups. Let $ \varphi: G\to H $ be a function.

  \begin{questionparts}
    \item Explain what it means for $ \varphi $ to be a homomorphism. Explain what it means for $ \varphi $ to be an isomorphism.
    \item Show that if $ \varphi $ is a homomorphism then $ \varphi $ fixes $e$ and $ \varphi (h^{-1})=\varphi (h)^{-1} $. Show that the composition of homomorphisms is also a homomorphism.
    \item Show that $ G \cong H \Leftrightarrow H \cong G $.
    \item Define the image and kernel of $ \varphi $ and show that they are subgroups. State and prove the conditions for $ \varphi $ to be surjective and injective, respectively.
    \item Define the direct product $ G \times H $. Show that everything in (the isomorphic copy of) $G$ commutes with everything in (the isomorphic copy of) $H$. State and prove the Direct Product Theorem.
  \end{questionparts}
\end{question}
\newpage
\begin{question}
  Let $G$ be a group. 
  
  Explain what it means for $G$ to be cyclic. Define the generator of a cyclic group. Show that the followings are cyclic and give the generators:
    \begin{enumerate}
      \item $ (\mathbb{Z} ,+) $,
      \item $ (\mathbb{Z}_n, +_n) $, where $ \gcd(k,n)=1 $.
    \end{enumerate}
  
  Show that a cyclic group $G$ is isomorphic to $ \mathbb{Z}  $ or $ C_n $ for some $ n\in \mathbb{N} $.

  Define the order of an element $ g\in G $. Show that $ \operatorname{ord} g |m$ where $ m\in \mathbb{N} $ such that $ g^m=e $. Prove that cyclic groups are abelian.

  Explain what it means for $G$ to be dihedral. Find all elements of $G$ and hence find $ |G| $. Find a generating set of $G$.

  Explain how to write group presentations. Show that $ \langle a,b,c| aba^{-1}b^{-1}=b, bcb^{-1}c^{-1}=c, cac^{-1}a^{-1}=a \rangle=\{e\} $.
\end{question}

\begin{question}
  Let $X$ be a set. Define a permutation of $X$. Define $ \operatorname{Sym}X $. Define $ S_n $. Explain what it means for a permutation to be a $k$-cycle or transposition. Explain what it means for two cycles to be disjoint. Show that disjoint cycles commute. Show that $ S_n $ is non-abelian for $ n\ge 3 $.

  Deduce the disjoint cycle decomposition of a permutation. Explain what the cycle type of a permutation is. Let $ \sigma\in S_n $. State and prove the relation between $ \operatorname{ord}\sigma $ and its disjoint cycle decomposition.

  Define the sign of a permutation and show that it is well-defined. Show that the sign is a surjective homomorphism. Define the alternating group. Show that $ \sigma\in S_n $ is even if and only if its disjoint cycle decomposition contains an even number of even cycles.
\end{question}

\begin{question}
  State and prove the Lagrange's Theorem, clearly explain any new concepts involved.

  Define the coset representation of a subgroup. Prove the followings:
  \begin{enumerate}
    \item $ g_1H=g_2H \Longleftrightarrow g_1^{-1}g_2\in H $.
    \item Let $G$ be a finite group and $g\in G$, then $ \ord(g)||G| $.
    \item Let $G$ be a finite group. If $g\in G$, then $ g^{|G|}=e $.
    \item If $ |G| $ is prime, then $G$ is cyclic, and is generated by any non-identity element.
  \end{enumerate}
  Define $\mathbb{Z}_n^*$ and the Euler's totient function. State and prove Fermat-Euler's theorem.

  Classify all groups of order 1, 2, 3, 4, 5.
\end{question}

\begin{question}
  Explain what it means for a subgroup to be normal. State and prove the equivalent definitions of a normal subgroup. Show that $ n \mathbb{Z} \trianglelefteq \mathbb{Z} $ and $ A_3 \trianglelefteq S_3 $.

  Show that every abelian group is normal and every subgroup of index 2 is normal. Classify all groups of order 6.

  Define the quotient group and quotient map, clearly explain any new concepts involved and show that the it is well-defined. Show that normal subgroups are exactly kernels of homomorphisms.

  Show the followings:
  \begin{enumerate}
    \item $ \mathbb{Z}/n\mathbb{Z} \cong \mathbb{Z}_{n} $.
    \item $ |G/N|=|G:N| $.
    \item If $ G=H\times K $, then $ H \trianglelefteq G \land K\trianglelefteq G $. We have $ G/H \cong K \land G/K \cong H $.
    \item $ N \trianglelefteq D_8 $ and $ D_8/N\cong C_2 \times C_2 $, where $ N=\langle r^2 \rangle  $. 
    \item $ H=\langle (12) \rangle \le S_3 $ cannot form a quotient.
  \end{enumerate}
\end{question}

\begin{question}
  State and prove the first isomorphism theorem.

  State and prove the correspondence theorem.

  State and prove the second and third isomorphism theorems.

  Draw the group lattice diagram of $ C_4 \times C_2 $.

  Define simple groups. Prove that $ C_p $ with $p$ prime are simple. Prove that $A_5$ is simple.
\end{question}

\begin{question}
  Let $ G $ be a group and $X$ be a set.
  \begin{questionparts}
    \item Explain what is an action of $G$ on $X$.
    \item Show that $ \alpha:G \times X \to X $ by $ \alpha(g,x)=\alpha_g(x) $ is an action if and only if the function $ \rho:G \to \sym(X) $ by $ \rho(g)=\alpha_g $ is a homomorphism. Define the kernel of an action.
    \item By considering unordered pairs of opposite faces and 1st isomorphism theorem, show that there are symmetries of a cube which do not move faces.
    \item Explain what it means for an action to be faithful.
  \end{questionparts}
\end{question}

\begin{question}
  Let $ G \curvearrowright X $. Define the orbit and stabliser of an action. Show that the stabliser is a subgroup and the orbits partition $X$.

  State and prove the Orbit-Stabliser theorem. Use this to show that $ |D_{2n}|=2n $.

  \begin{questionparts}
    \item Determine the symmetries of a tetrahedron. Determine the group of rotations of a tetrahedron.
    \item Determine the symmetries of a cube. Determine the group of rotations of a cube.
    \item List all platonic solids.
  \end{questionparts}
  \textit{[You should use the fact that symmetries are isometries.]}

  State and prove Cauchy's theorem.
\end{question}

\begin{question}
  \begin{questionparts}
    \item By using left multiplication action, prove Cayley's theorem. 
    
    Let $H\le G$, show that $G \curvearrowright G/H$ by left multiplication and it is transitive.
    \item Show that a group acts on itself by conjugation. 
    
    Define the centre, conjugacy class, and centraliser. 
    
    Show that the centre is the intersection of all centralisers.
    
    Define the conjugation of a subgroup and show that it is also a subgroup. 
    
    Prove that the conjugation of a subgroup is isomorphic to itself. 
    
    Show that group $G$ acts by conjugation on the set of its subgroups and the singleton orbits are the normal subgroups. 
    
    Show that normal subgroups are those subgroups that are union of conjugacy classes.
    \item Given a $k$-cycle $(a_1\ a_2\ \cdots\ a_k)$ and $ \sigma\in S_n $, show that $ \sigma(a_1\ \cdots \ a_k)\sigma^{-1}=(\sigma(a_1)\ \sigma(a_2)\ \cdots\ \sigma(a_k)). $
    \item Show that two elements of $S_n$ are conjugate if and only if they have the same cycle type. Hence find conjugacy classes of $S_4$ and $ S_5 $. Find also the normal subgroups of $ S_4,\ S_5 $.
    \item Let $ \sigma\in S_n $. Explain what it means for its conjugacy class to split in $A_n$. 
    
    Show that the conjugacy class of $ \sigma $ in $A_n$ splits in $A_n$ if and only if no odd permutations commute with $\sigma$. Find conjugacy classes of $ A_4,\ A_5 $ and hence show that $A_5$ is a simple group. 

    Are $ A_n,\ n\ge 5 $ simple?
  \end{questionparts}
\end{question}

\begin{question}
  \begin{questionparts}
    \item  Define the \mobius group and show that it is well-defined. Show that every \mobius map can be written as a composition of $ az, z+b,1/z $. Show that the action $ \mcM \curvearrowright \hat{\mathbb{C}} $ is faithful, hence that $ \mathcal{M}\le \sym(\hat{\bbC}) $.
    \item Explain what a fixed point of a \mobius map is. Show that a \mobius map with $\ge 3$ fixed points is the identity. Hence show there is a unique \mobius map sending any 3 distinct points of $ \hat{\mathbb{C}} $ to any 3 distinct points of $ \hat{\bbC} $. i.e., given $ z_1,z_2,z_3, w_1,w_2,w_3\in \hat{\bbC} $ distinct, $ \exists ! f, f(z_i)=w_i $ for $i=1,2,3$.
    \item Let $ f,h\in \mcM $. Show that $ \ord(hfh^{-1})=\ord f $ and $ f $ fixes $z$ if and only if $ hfh^{-1} $ fixes $ h(z) $. Show that every non-identity $f\in \mcM$ has either 1 or 2 fixed points and find the corresponding map of conjugation. Explain how we can use this to compute $ f^n $.
    \item Give the equations of a circle and a line in $ \hat{\bbC} $. Explain how we can regard both of them as ``circles''. Show that \mobius maps send circles to circles in $ \hat{\bbC} $.
    \item Define the cross-ratio of $ z_1,z_2,z_3,z_4\in \hat{\bbC} $. You should show that it is well-defined. Give an explicit formula of the cross-ratio. Show that double transpositions of $z_i$ fix the cross-ratio. Show that \mobius maps preserve cross-ratios.
    \item Show that four distinct points $z_1,z_2,z_3,z_4\in \hat{\bbC}$ lie on a circle if and only if $[z_1,z_2,z_3,z_4]\in \bbR$.
  \end{questionparts}
\end{question}

\begin{question}
  List all lectured matrix groups. Show that 
  \begin{enumerate}
    \item $ \GL_n(\bbF) \to \bbF^*:=\bbF \setminus \{0\} $ is a surjective homomorphism.
    \item $ \det : \Or_n\to \left\{ \pm 1 \right\} $ is a surjective homomorphism.
    \item The function $ \varphi:\SL_2(\bbC)\to \mcM $ defined by 
    \[
        \begin{pmatrix}
            a&b\\
            c&d
        \end{pmatrix} \mapsto \frac{az+b}{cz+d}
    \]
    is a surjective homomorphism with kernel $ \{\pm I\} $.
    \item $ \mcM \cong \SL_2(\bbC)/\{\pm I\} $.
  \end{enumerate}

  Derive change of basis using group theory.

  Show that the elements of $ \SO_2 $ are exactly rotations and the elements of $ \Or_2\setminus\SO_2 $ are exactly reflections. Show that every element of $\Or_2$ is the composition of at most 2 reflections.

  Show that every element of $\Or_3$ is the composition of at most 3 reflections.
\end{question}

\begin{question}
  Define the group of quaternions. Hence classify all groups of order 8.
\end{question}

\end{document}

