\documentclass[10pt]{article}
\pdfoutput=1 
\usepackage{NotesTeX,lipsum}

%\usepackage{showframe}

\title{\begin{center}{\Huge \textit{Numbers and Sets Notes}}\\{{\itshape Based on Lectures and "The Higher Arithmetic"}}\end{center}}
\author{$\theta\omega\theta$}
\affiliation{
Not in University of Cambridge\\
skipped some takes irrelevant to contents\\
}

\emailAdd{not telling you}

\begin{document}
	\maketitle
	\flushbottom
	\newpage
	\pagestyle{fancynotes}
	\part{Number Theory}
	%contents
    \begin{margintable}\vspace{.8in}\footnotesize
    	\begin{tabularx}{\marginparwidth}{|X}
    		1.1 Perfect cover
		\end{tabularx}
	\end{margintable}

	%main doc
	\section*{Pre: Introduction}
	\begin{itemize}
		\item Number theory $\Rightarrow$ The reals $\Rightarrow $ Sets and functions $\Rightarrow $ Countability
		\item Recommended books: Allenby: “Numbers and Proofs”; Hamilton: “Numbers, sets and functions”; Davenport: “The Higher Arithmetic” 
	\end{itemize}
	%section 1
	\section{Preliminaries}
	\subsection{Q and A}
		\begin{itemize}
			\item Q: What is a proof?
			A: A proof is a \emph{logical argument} that establishes a \emph{conclusion}.
			\item Q: Why do we prove things?
			A: \begin{itemize}
				\item To be sure they are true.
				\item To understand why they are true.
			\end{itemize}
		\end{itemize}
	\subsection{Examples of Proofs and non-Proofs}
	\begin{enumerate}
		\item[\bf I] $ \forall n\in \mathbb{N}^*, 3|n^3-n $.
		\item[\bf II] $ \forall n $, if $n^2$ is even then $n$ is even.
		\item[\bf III] For any $n\in \mathbb{N}*$, if $9|n^2$ then $9|n$.	\marginnote{This is a \textit{wrong} claim. A counterexample is $n=3$.} 
	\end{enumerate}

	%end of section 1
\end{document}