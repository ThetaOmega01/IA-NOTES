\documentclass[10pt]{article}
\pdfoutput=1 
\usepackage{NotesTeX,lipsum}
\renewcommand{\ge}{\geqslant}
\renewcommand{\le}{\leqslant}
%\usepackage{showframe}

\title{\begin{center}{\Huge \textit{Numbers and Sets Notes}}\\{{\itshape Based on Lectures and "The Higher Arithmetic"}}\end{center}}
\author{$\theta\omega\theta$}
\affiliation{
Not in University of Cambridge\\
skipped some talks irrelevant to contents\\
}

\emailAdd{not telling you}

\begin{document}
	\maketitle
	\flushbottom
	\newpage
	\pagestyle{fancynotes}
	\part{Elementary Number Theory}

	%main doc
	\section*{Pre: Introduction}
	\begin{itemize}
		\item Number theory $\Rightarrow$ The reals $\Rightarrow $ Sets and functions $\Rightarrow $ Countability
		\item Recommended books: Allenby: “Numbers and Proofs”; Hamilton: “Numbers, sets and functions”; Davenport: “The Higher Arithmetic” 
	\end{itemize}
	%section 1
	\section{Preliminaries}
	\subsection{Q and A}
		\begin{itemize}
			\item Q: What is a proof?
			A: A proof is a \emph{logical argument} that establishes a \emph{conclusion}.
			\item Q: Why do we prove things?
			A: \begin{itemize}
				\item To be sure they are true.
				\item To understand why they are true.
			\end{itemize}
		\end{itemize}
	\subsection{Examples of Proofs and non-Proofs}
	\begin{enumerate}
		\item[\bf I] $ \forall n\in \mathbb{N}^*, 3|n^3-n $.
		\item[\bf II] $ \forall n $, if $n^2$ is even then $n$ is even.
		\item[\bf III] For any $n\in \mathbb{N}*$, if $9|n^2$ then $9|n$.	\marginnote{This is a \textit{wrong} claim. A counterexample is $n=3$.} 
	\end{enumerate}

	Definition of if and only if, ..., skipped

	%end of section 1
	\section{The Natural Numbers}
	\begin{definition}
		The set of natural numbers $ \mathbb{N} $ is a set containing an element “1” and with an operation “$+1$” satisfying
		\begin{enumerate}[(1)]
			\item $ \forall n\in \mathbb{N}, n+1\neq 1. $
			\item If $ m\neq n $, then $ m+1\neq n+1. $ \marginnote{These two rules ensure that all natural numbers are different.}
			\item (Induction Axiom) For any property $P(n)$, $ (P(1)\land P(n)\Rightarrow P(n+1)) \Rightarrow \forall n\in \mathbb{N} (P(n))$.
		\end{enumerate}
		They are called the \textit{Peano axioms}.
	\end{definition}
	Thus we can define $ +k $ recursively by $ 2=1+1 $, and $ n+(k+1)=(n+k)+1 $. Other usual operations can be defined similarly. Usual laws of arithmetic can be derived by induction.
	\begin{proposition}[Strong induction]\label{prop:strong_induction}
		$(P(1) \land \forall n (\forall m\le n, P(m) \Rightarrow P(n+1))) \Rightarrow \forall n(P(n))$.
	\end{proposition}
	\begin{remark}
		Some equivalent forms of strong induction:
		\begin{enumerate}
			\item If $P(n)$ fails for some $n$, then we have a minimum element $n'$ such that $P(n')$ is false but $P(m)$ is true for all $m\le n'$.
			\item If $P(n)$ for some $n$ then there is a least $n$ with $P(n)$.

			Often referred as the \textit{well-ordering principle}.
		\end{enumerate}
	\end{remark}
	\section{The Integers}
	Written in $ \mathbb{Z} $, consist of all symbols $n,-n,n\in \mathbb{N}$ and 0. Usual laws hold.

	Expand definition of order by $ a<b $ if and only if $ \exists c\in \mathbb{N} $, $a+c=b$. We have
	\[
		\forall a,b,c, a<b \land c>0 \Longrightarrow ac<bc
	.\]
	\section{The Rationals}
	Written in $ \mathbb{Q} $, consists of all expressions $ a/b, a,b\in \mathbb{Z}, b\neq 0 $ with $ a/b $ regarded as $ c/d $ if $ad=bc$.

	Addition is defined as 
	\[
		\frac{a}{b}+\frac{c}{d} = \frac{ad+bc}{bd}
	.\]
	This holds however we write those fractions.
	\begin{remark}
		Unlike clear unambiguity in $\mathbb{Z}$, we cannot define operations like $ a/b\mapsto a^2/b^3 $.
	\end{remark}

	All usual laws work, with order defined as $ a/b<c/d(b,d>0) $ if and only if $ ad<bc $.

	\section{Primes}
	\begin{definition}
		$m$ is said to be a $divisor$ of $n$ if and only if $\exists k\in \mathbb{N} , n=km$.

		$ p\in \mathbb{N} $ is \textit{prime} if and only if only $1$ and $p$ divide $p$.
	\end{definition}
	\begin{proposition}\label{prop:factor_into_primes}
		Every natural numebr $n\ge 2$ is expressible as a product of primes.
	\end{proposition}
	Proved by induction.
	\marginnote{Lecture 4.}
	\begin{theorem}\label{thm:infinite_primes}
		There are infinitely many primes.
	\end{theorem}
	\begin{definition}
		For $ a,b\in \mathbb{N}  $, a natural number $c$ is the hcf of $a,b$ if $ c|a \land c|b $ and $ d|a \land d|b \Rightarrow d|c $.
	\end{definition}
	\begin{proposition}\label{prop:division_alg}
		Let $ n,k $ be natural numbers. Then $ \exists q,r in \mathbb{Z} , 0\le r<k $ that $ n=qk+r $.
	\end{proposition}
	\begin{theorem}[Euclid's Algorithm]\label{thm:Euclid algorithm}
		skipped
	\end{theorem}
\end{document}