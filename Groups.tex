\documentclass[10pt]{article}
\pdfoutput=1 
\usepackage{NotesTeX,lipsum}

%\usepackage{showframe}

\title{\begin{center}{\Huge \textit{Groups}}\\{{\itshape Based on Lectures and "Algebra and Geometry"}}\end{center}}
\author{$\theta\omega\theta$}
\affiliation{
Not in University of Cambridge\\
skipped some takes irrelevant to contents\\
}

\emailAdd{not telling you}

\begin{document}
	\maketitle
	\flushbottom
	\newpage
	\pagestyle{fancynotes}
	\part{Groups and Permutations}

    \section{Definition of Groups}
    \begin{definition}[Group]
        A group is a set $G$ together with a binary operation $ \ast: G\times G \to G $ that 
        \begin{enumerate}
            \item (Closure) $ \forall g,h\in G, g\ast h $,
            \item (Identity)$ \exists e\in G, \forall g\in G, e*g = g*e = g $,
            \item (Inverse)$ \forall g\in G, \exists g^{-1}\in G, g * g^{-1} = g^{-1}*g = e $,
            \item (Associativity)$ \forall g,h,k\in G, (g*h)*k = g*(h*k) $. 
        \end{enumerate}
    \end{definition}
    \begin{remark}
        The inverse of $g$ is unique, for if there are two $g',g''$, both are inverses of $g$, we have 
        \[
            g' = g''*g*g' = g''
        .\]
    \end{remark}
    \begin{example}
        \begin{enumerate}[(1)]
            \item $G = \left\{ e\right\}$, the trivial group,
            \item $ G = \left\{ \text{symmetries of } \triangle \right\} $,
            \item $ (\mathbb{Z} , +) $,
            \item $ (\mathbb{R} ,+), (\mathbb{Q} , +), (\mathbb{C} , +) $,
            \item $ \mathbb{R}^* = \mathbb{R} \setminus \left\{ 0\right\}; (\mathbb{R}^*, \times) $,
            \item $ (\mathbb{Z}_n, + \bmod n), \mathbb{Z}_n = \left\{ 0,1,\dots, n-1\right\} $,
            \item Vector spaces with addition of vectors,
            \item $ (\rm{GL}_2(\mathbb{R}), \text{matrix multiplication}) $, set of invertible $2\times 2$ matrices,
        \end{enumerate}
    \end{example}
    \begin{example}[non-examples]
        \begin{enumerate}[(1)]
            \item $ (\mathbb{Z}_n, +) $, since it is not closed,
            \item $ (\mathbb{Z} , \times) $, since some inverses do not exist,
            \item $ (\mathbb{R} , *) $, where $ r*s = r^2 s $, since there is no identity,
            \item $ (\mathbb{N}, *), n*m = |n-m| $.\marginnote{Here $ \mathbb{N} $ is the set of all positive numbers, and it remains this definition unless specified otherwise.} Associativity fails.
        \end{enumerate}
    \end{example}
\end{document}