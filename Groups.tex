\documentclass[10pt]{article}
\pdfoutput=1 
\usepackage{NotesTeX,lipsum}
\DeclareMathOperator{\im}{Im}
\usepackage{IEEEtrantools}
\def\d{{\mathrm d}}
\def\e{{\mathrm e}}
\def\g{{\mathrm g}}
\def\h{{\mathrm h}}
\def\f{{\mathrm f}}
\def\p{{\mathrm p}}
\def\s{{\mathrm s}}
\def\t{{\mathrm t}}
\def\i{{\mathrm i}}

\def\A{{\mathrm A}}
\def\B{{\mathrm B}}
\def\E{{\mathrm E}}
\def\F{{\mathrm F}}
\def\G{{\mathrm G}}
\def\H{{\mathrm H}}
\def\P{{\mathrm P}}


\def\bb{\mathbf b}
\def \bc{\mathbf c}
\def\bx {\mathbf x}
\def\bn {\mathbf n}
\def\le{\leqslant}
\def\ge{\geqslant}
\def\arcosh{{\rm arcosh}\,}

\newcommand{\bluecomment}[1]{{\color{blue}#1}}
%\renewcommand{\comment}[1]{}
\newcommand{\redcomment}[1]{{\color{red}#1}}
\newcommand{\tm}{\times}
%\usepackage{showframe}
\DeclareMathOperator{\ord}{ord}
\DeclareMathOperator{\sym}{Sym}
\DeclareMathOperator{\lcm}{lcm}
%\usepackage{showframe}

\title{\begin{center}{\Huge \textit{Groups}}\\{{\itshape Based on Lectures and "Algebra and Geometry"}}\end{center}}
\author{$\theta\omega\theta$}
\affiliation{
Not in University of Cambridge\\
skipped some takes irrelevant to contents\\
}

\emailAdd{not telling you}

\begin{document}
	\maketitle
	\flushbottom
	\newpage
	\pagestyle{fancynotes}
	\part{Groups and Permutations}

    \section{Definition of Groups}
    \begin{definition}[Group]
        A group is a set $G$ together with a binary operation $ \ast: G\times G \to G $ that 
        \begin{enumerate}
            \item (Closure) $ \forall g,h\in G, g\ast h $,
            \item (Identity)$ \exists e\in G, \forall g\in G, e*g = g*e = g $,
            \item (Inverse)$ \forall g\in G, \exists g^{-1}\in G, g * g^{-1} = g^{-1}*g = e $,
            \item (Associativity)$ \forall g,h,k\in G, (g*h)*k = g*(h*k) $. 
        \end{enumerate}
    \end{definition}
    \begin{remark}
        The inverse of $g$ is unique, for if there are two $g',g''$, both are inverses of $g$, we have 
        \[
            g' = g''*g*g' = g''
        .\]
    \end{remark}
    \begin{example}
        \begin{enumerate}[(1)]
            \item $G = \left\{ e\right\}$, the trivial group,
            \item $ G = \left\{ \text{symmetries of } \triangle \right\} $,
            \item $ (\mathbb{Z} , +) $,
            \item $ (\mathbb{R} ,+), (\mathbb{Q} , +), (\mathbb{C} , +) $,
            \item $ \mathbb{R}^* = \mathbb{R} \setminus \left\{ 0\right\}; (\mathbb{R}^*, \times) $,
            \item $ (\mathbb{Z}_n, + \bmod n), \mathbb{Z}_n = \left\{ 0,1,\dots, n-1\right\} $,
            \item Vector spaces with addition of vectors,
            \item $ (\rm{GL}_2(\mathbb{R}), \text{matrix multiplication}) $, set of invertible $2\times 2$ matrices,
        \end{enumerate}
    \end{example}
    \begin{example}[non-examples]
        \begin{enumerate}[(1)]
            \item $ (\mathbb{Z}_n, +) $, since it is not closed,
            \item $ (\mathbb{Z} , \times) $, since some inverses do not exist,
            \item $ (\mathbb{R} , *) $, where $ r*s = r^2 s $, since there is no identity,
            \item $ (\mathbb{N}, *), n*m = |n-m| $.\marginnote{Here $ \mathbb{N} $ is the set of all positive numbers, and it remains this definition unless specified otherwise.} Associativity fails.
        \end{enumerate}
    \end{example}
    \section{Properties of Groups}
    \begin{proposition}\label{prop:groups_properties}
        Let $G$ be a group, then we have 
        \begin{enumerate}
            \item The identity is unique.
            \item THe inverse is unique.
            \item $ gh=g \land hg=g \Rightarrow h=e $.
            \item $ gh=e \Rightarrow hg=e, h=g^{-1} $.
            \item $ (g^{-1})^{-1}=g $.
        \end{enumerate}
    \end{proposition}
    \begin{definition}
        A group $G$ is called \textit{abelian} if $ \forall g,h\in G, gh=hg $.
    \end{definition}
    \begin{definition}
        $G$ is said to be \textit{finite} if it has finitely may elements. Denote $|G|$ as its number of elements.
    \end{definition}
    \begin{definition}
        Let $ (G,*) $ be a group. A subset $H\subseteq G$ is called a \textit{subgroup} of $G$ if $ (H,*) $ is a group, written as $ H\le G $.
    \end{definition}
    \begin{remark}
        To check $ H\le G $, simply check closure, identity, and inverses. Associativity is inherited.
    \end{remark}
    \begin{proposition}\label{prop:uniqueness of identity}
        Let $ e_H, e_G $ be the identities in $H$ and $G$ respectively, then $e_H=e_G$. 
    \end{proposition}
    \begin{example}
        \begin{enumerate}[(1)]
            \item $ \left\{ e\right\} \le G $.
            \item $ G\le G $.
            \item $ (\mathbb{Z} ,+)\le (\mathbb{Q} ,+)\le (\mathbb{R} ,+)\le (\mathbb{C} ,+) $.
        \end{enumerate}
    \end{example}
    \begin{lemma}[subgroup test]\label{lma:subgroup test}
        Let $ G $ be a group, then $ H\le G \Leftrightarrow H\neq \varnothing \land \forall a,b\in H, ab^{-1}\in H $.
    \end{lemma}
    \begin{proof}
        Since $gg^{-1}=e\in H$, identity is satisfied. Since $ \forall a,b\in H, a(b^{-1})^{-1}=ab\in H $, closure is satisfied. $ \forall g\in H, eg^{-1}=g^{-1}\in H $, inverse is satisfied.
    \end{proof}
    \begin{proposition}\label{prop:subgroups of integers}
        The subgroups of $ (\mathbb{Z} ,+) $ are precisely $ (n \mathbb{Z} ,+) $.
    \end{proposition}
    Proved by considering the minimal element.

    Usual laws:
    \begin{proposition}\label{prop:comparing_groups}
        \begin{enumerate}[(1)]
            \item Let $ H,K $ be subgroups of $G$ then $ H\cap K\le G $.
            \item $ K\le H \land H\le G \Rightarrow K\le G $.
            \item $ K \subseteq H, H\le G, K\le G \Rightarrow K\le H $.
        \end{enumerate}
    \end{proposition}
    \begin{definition}
        If $ X\neq \varnothing $ is a subset of group $G$, the subgroup \textit{generated} by $X$, written as $ \langle X \rangle $, is the intersection of all subgroups containing $X$.
    \end{definition}
    \begin{remark}
        \begin{itemize}
            \item $ e\in \langle X \rangle $.
            \item $ X \subseteq \langle X \rangle $.
            \item $ \langle X \rangle $ contains all possible products of elements of $X$ and their inverses.
        \end{itemize}
    \end{remark}
    \begin{proposition}\label{prop:generation_of_groups}
        Let $ \varnothing \neq X \subseteq G $. Then $ \langle X \rangle $ is the set of elements of $ G $ of the form
        \[
            x_1^{\alpha_1}x_2^{\alpha_2}\cdots x_k^{r_k},\quad x_i\in X, \alpha_i\in \left\{ -1,1\right\}, k\ge 0
        .\]
    \end{proposition}
    \begin{proof}
        Let $T$ be such a set. Then by definition $ T \subseteq \langle X \rangle $. On the other hand, $ X \subseteq T \Rightarrow \langle X \rangle  \subseteq T$ since $T$ clearly forms a subgroup. Hence $ T=\langle X \rangle $.
    \end{proof}
    \section{Homomorphisms}
    \subsection{Definition and basic properties}
    \begin{definition}
        Let $ (G,*_G), (H,*_H) $ be groups. A function $ \varphi: H \to G $ is a \textit{homomorphism} if
        \[
            \forall a,b\in H, \varphi (a*_HB)=\varphi (a)*_G \varphi (b)
        .\]
        It is called an \textit{isomorphism} if it is bijective.
    \end{definition}
    \begin{proposition}\label{prop:homom}
        Let $ \varphi :H\to G $ be a homomorphism.
        \begin{enumerate}[(1)]
            \item $ \varphi (e_H)=e_G $.
            \item $ \varphi (h^{-1})=\varphi (h)^{-1} $.
            \item If $ \psi:G\to K $ is also a homomorphism, then $ \psi \varphi :H\to K $ is a homomorphism.
        \end{enumerate}
    \end{proposition}
    \begin{proposition}\label{prop:isom_inverse_is_also_an_isom}
        Let $ \varphi :H\to G $ be an isomorphism. Then $ \varphi^{-1} $ is also an isomorphism and this implies that 
        \[
            G \cong H \Longleftrightarrow H \cong G
        .\]
    \end{proposition}
    %Lecture 4
    \subsection{Images and Kernels}\marginnote{Lecture 4.}
    \begin{definition}
        The \textit{image} of a homomorphism $ \varphi: H \to G $ is 
        \[
            \operatorname{Im}(\varphi)=\left\{ g\in G: \exists h\in H, \varphi(h)=g \right\}
        .\]
        The \textit{kernel} of $ \varphi $ is 
        \[
            \ker(\varphi)=\left\{ h\in H: \varphi(h)=e_G \right\}
        .\]
    \end{definition}
    We have two immediate consequences:
    \begin{proposition}\label{prop:ker, im are groups}
        $ \operatorname{Im}(\varphi), \ker (\varphi)  $ are subgroups of $G,H$ respectively.
    \end{proposition}
    \begin{proof}
        Take $ \operatorname{Im}(\varphi) $ as an example. Use lemma \ref{lma:subgroup test}: $ \operatorname{Im}(\varphi)  $ is non-empty since $ \varphi(e_H)=e_G $. For any $ a,b\in \operatorname{Im}(\varphi)  $, we have $ a=\varphi(h), b=\varphi(h') $ for $h,h'\in H$. Hence
        \[
            ab^{-1}=\varphi(h)\varphi(h')^{-1}=\varphi(hh'^{-1})\in \operatorname{Im}(\varphi) 
        .\]
        Hence $ \operatorname{Im}(\varphi)  $ is a subgroup. It is similar for $ \ker (\varphi) $.
    \end{proof}
    \begin{example}
        \begin{enumerate}[(1)]
            \item[(0)] Let $ \varphi:H\to G $ be the trivial homomorphism, i.e. $ \varphi(h)\equiv e_G $. Then $ \im(\varphi)=\left\{ e_G\right\} $ and $ \ker (\varphi)=H $.
            \item Let $ \iota: H\to G $, where $H\le G$, be the inclusion map. Then $ \im (\iota) = H, \ker (\iota) = \left\{ e_H\right\} $.
            \item $ \varphi: \mathbb{Z} \to \mathbb{Z}_n, \varphi(k)=k \bmod n $. $ \im (\varphi)=\mathbb{Z}_n, \ker (\varphi)= n \mathbb{Z}$.
        \end{enumerate}
    \end{example}
    \begin{proposition}\label{prop:homom surj}
        Let $ \varphi:H\to G $ be a homomorphism.
        \begin{enumerate}[(1)]
            \item $ \varphi $ is surjective if and only if $ \im \varphi = G $,
            \item $ \varphi $ is injective if and only if $ \ker \varphi = \left\{ e\right\} $.
        \end{enumerate}
    \end{proposition}
    \begin{proof}
        By definition, (1) holds.

        Suppose $ \varphi $ is injective. Take $ h\in \ker \varphi $. Then $ \varphi(h)=\varphi(e)=e_G \Leftrightarrow h=e $. Conversely suppose $ \ker \varphi=\left\{ e\right\} $. Take $ a,b $ such that $ \varphi(a)=\varphi(b) $. We have
        \[
            \varphi(ab_{-1})=\varphi(a)\varphi(b)^{-1}=e_G
        .\]
        Thus $ ab^{-1}=e_G \Leftrightarrow a=b $ and $ \varphi $ is injective.
    \end{proof}
    \section{Direct product of groups}
    \begin{definition}
        The \textit{direct product} of two groups $ G,H $ is the set $ G\times H $ with the operation of component-wise composition:
        \[
            (g_1,h_1) * (g_2,h_2):=(g_1*_G g_2, h_1 *_H h_2)
        .\]
    \end{definition}
    Closure and identity are easily verified. The inverse is component-wise and associativity is inherited from $G,H$.
    \begin{remark}
        $ G\times H $ contains subgroups isomorphic to $G$ and $H$, i.e., $ G \times \{e_H\} $ and $ \{e_G\} \times H $.
    \end{remark}
    \begin{example}
        $ \mathbb{Z} \times \{-1,1\} $ has elements $ (n,\pm 1), n\in \mathbb{Z} $ with $ (n,-1)*(m,-1)=(n+m,(-1)(-1))=(n+m,1) $, etc. Addition in the first component and multiplication in the second.

        The identity of $\mathbb{Z} \times \{-1,1\}$ is $(0,1)$.
    \end{example}
    \begin{remark}
        In $ G \times H $, everything in (the isomorphic copy of) $G$ \textit{commutes} with everything in (the isomorphic copy of) $H$. That is to say,
        \[
            \forall (g,e_H), (e_G,h), (g,e_H)*(e_G,h) = (e_G, h)*(g,e_H)=(g,h)
        .\]
    \end{remark}
    \begin{theorem}[Direct Product Theorem]\label{thm:Direct Product Theorem}
        \marginnote{This gives us two ways to think about direct products:
        \begin{itemize}
            \item Given two groups $H,K$, one can form their direct products $ H \times K $ and view $H,K$ as subgroups via $ H \times \{e_K\} $ and $ \{e_H\}\times K $.
            \item Given a group $G$ with subgroups $H,K$ that satisfiy these conditions, then we are equivalently dealing with $ H \times K $.
        \end{itemize}
        By convention, we can simply regard $ H \times \{e_K\},\{e_H\}\times K $ as $ H,K. $}
        Let $ H,K\le G $ such that
        \begin{enumerate}[(1)]
            \item $ H \cap K=\left\{ e\right\} $: they are \textit{disjoint},
            \item $ \forall h,k, hk=kh $: they are \textit{commutative},
            \item $ \forall g\in G, \exists h\in H, k\in K, g=hk $: $ G=HK $.
        \end{enumerate}
        Then $ G \cong H \times K $.
    \end{theorem}
    \begin{proof}
        Consider the function $ \varphi: H\times K \to G $ defined by $ \varphi(h,k)=hk $. Note that 
        \[
            \varphi(h,k) * \varphi(h',k') = hkh'k'=hh'kk'=\varphi(hh',kk')=\varphi((h,k)*(h',k'))
        ,\]
        so $ \varphi $ is a homomorphism. From (3) we know that $ \varphi $ is surjective. Let $ \varphi(h,k)=e $, then $ hk=e \Leftrightarrow h=k^{-1} $. Hence $ h,k^{-1}\in H \cap K $ so $ h=k=e $. Hence it is injective. Thus $\varphi$ is an isomorphism and $ G \cong H \times K $.
    \end{proof}
    \section{Important Examples}
    \subsection{Cyclic groups}
    \begin{definition}
        Let $G$ be a group and let $ X \subseteq G, X\neq \varnothing $. If $ \langle X \rangle =G $, then $X$ is called \textit{a generating set}\mn{It is not necessary unique.} of $G$.

        $G$ is \textit{cyclic} if $\exists a\in G$ such that $ \langle a \rangle =G $. In this case, $ \forall b\in G, \exists k\in \mathbb{Z} , b=a^k $. $a$ is called a \textit{generator} of $G$.
    \end{definition}
    \begin{example}
        \begin{enumerate}[(1)]
            \item[(0)] Trivial group $ \left\{ e\right\}=\langle e \rangle  $.
            \item $ (\mathbb{Z} ,+)=\langle 1 \rangle =\langle -1 \rangle  $.
            \item $ (\mathbb{Z}_n, +_n)=\langle 1 \rangle =\langle k \rangle $, where $(k,n)=1$.
            \item $ E=\left(\left\{ e^{\frac{2\pi ik}{n}}:0\le k\le n-1 \right\}, \cdot\right) =\langle e^{\frac{2\pi i m}{n}} \rangle $, where $(m,n)=1$.\marginnote{Hence, $ E \cong \mathbb{Z}_n $.}
            \item $ \left\{ e,a,a^2,\dots, a^{n-1}\right\} $ with
            \[
                a^k * a^j = \begin{cases}
                a^{k+j} &\text{if } k+j<n,\\
                a^{k+j-n} &\text{if } k+j\ge n.\\
                \end{cases} 
            \]
            Again, it is isomorphic to $ \mathbb{Z}_n $.
        \end{enumerate}
    \end{example}
    Write $ C_n = \left\{ e,a,a^2,\dots, a^{n-1}\right\} $. Then every cyclic group is isomorphic to $C_n$ and we can write all cyclic groups in this form, or $ \cong \mathbb{Z} $, which is the infinite case.
    %Lecture 5
    \marginnote{Lecture 5}
    \begin{theorem}\label{thm:cyclic group isom}
        A cyclic group $G$ is isomorphic to $ \mathbb{Z}  $ or $ C_n $ for some $ n\in \mathbb{N} $.
        \marginnote{Therefore we often write $ \mathbb{Z}  $ or $ C_n $ for a cyclic group, regardless of its description.}
    \end{theorem}
    \begin{proof}
        Let $ G= \langle b \rangle  $. Suppose that $ \exists n, b^n=e $. Take the smallest $n$. Define $ \varphi: C_n = \left\{ e,a,a^2,\dots,a^{n-1}\right\}\to G $ by $ \varphi(a^k)=b^k(0\le k\le n-1) $. Then $ \forall a^j,a^k\in C_n, j,k<n $, we have 
        \[
            \varphi(a^j\cdot a^k)=\varphi(a^{j+k})=b^{j+k}= b^j*b^k=\varphi(a^j)*\varphi(a^k)
        .\]
        If $ j+k \ge n $, 
        \[
            \varphi(a^j\cdot a^k)=\varphi(a^{j+k-n})=b^{j+k-n}=b^{j+k}*(b^{n})^{-1}=b^{j+k}=\varphi(a^j)*\varphi(a^k)
        .\]
        Hence $ \varphi $ is a homomorphism. Since $ b^n=e $, $ \varphi $ is surjective. Suppose $ \varphi(a^k)=e \Leftrightarrow b^k=e \Leftrightarrow k=0 $, since $ 0\le k\le n-1 $. Otherwise \# to minimality of $n$.

        If no such $ n $ exists, then define $ \varphi: \mathbb{Z} \to G $ ny $ \varphi(k)=b^k $. Note that
        \[
            \varphi(k+m)=b^{k+m}=b^k*b^m=\varphi(k)*\varphi(m)
        .\]
        Also $ \forall b^k\in G=\langle b \rangle, \varphi(k)=b^k $, and if $ m\in \ker \varphi $, then $ \varphi(m)=e=b^m \land \varphi(-m)=e $. If $ m\neq 0 $, then \# to the assumption that $ \nexists n, b^n=e $.

        Therefore, $ G \cong \mathbb{Z} \lor G \cong C_n $.
    \end{proof}
    \begin{definition}
        The \textit{order of an element} $ g\in G $ is the smallest $ n\in \mathbb{N}  $ that $ g^n=e $. If no such $n$ exists, we say $g$ has and \textit{infinite order}. The order of $g$ is written as $ \ord g. $
    \end{definition}
    \begin{proposition}\label{prop:div_order}
        If $ g^m=e ,m>0$, then $ \ord g|m $.
    \end{proposition}
    \begin{proof}
        If not, then $ m=q\ord g +r$ for some $ q,r\in \mathbb{N}  $ such that $ 0\le r\le \ord g-1 $, \#.
    \end{proof}
    \begin{remark}
        Given $ g\in G $, the subgroup $ \langle g \rangle \cong C_n $ if $ \ord g=n $, and $ \cong \mathbb{Z}  $ if $ \ord g=\infty $. Hence $ \ord g=|\langle g \rangle | $. 
    \end{remark}
    \begin{proposition}\label{prop:cyclic_abelian}
        Cyclic groups are abelian.
    \end{proposition}
    \subsection{Dihedral Groups}
    \begin{definition}
        The \textit{dihedral group} $ D_{2n} $ is the group of symmetries of a regular $n$-gon, the operation is composition of symmetries.
    \end{definition}
    \begin{example}
        $ D_{6}= $ symmetries of $ \triangle $.
    \end{example}
    What are the elements of $ D_{2n} $?

    Clearly we have $n$ rotations of angles
    \[
        \frac{2\pi k}{n}, \quad 0\le k<n
    .\]
    \begin{itemize}
        \item When $n$ is odd, we have $n$ reflections in axes through the centre and each of the vertices.
        \item When $n$ is even, we have $ n/2 $ reflections in axes through centre and pairs of opposite vertices. Another $ n/2 $ reflections in axes through pairs of opposite mid-points of edges.
    \end{itemize}
    Assert that these are all the elements of $ D_{2n} $. Indeed, let $ g\in D_{2n} $. Since $g$ is a symmetry, then $g$ must send vertices to vertices, e.g., $ g(v_1)=v_i $. $g$ must also send edges to edges, so $ v_2,v_n $ must be sent to $ \left\{ v_{i-1},v_{i+1}\right\} $.
    Note that once we know where $g(v_1),g(v_2)$, then $g(v_n)$ is determined. \textit{Inductively}, all other $g(v_j)$ are determined, and hence $g$ is known. Since there are $ n $ choices for $v_1$ and 2 choices for $v_2$, so we have $2n$ elements in total. Hence there are no other elements.

    It can be checked easily that $D_{2n}$ is a group.

    \begin{remark}
        Can generate $ D_{2n} $ by a rotation and a reflection. Let $ r $ be the rotation $ \frac{2\pi}{n} $ and $s$ be the reflection in axis through $v_1$ and centre, then $r^k$ give all rotations. Consider $ r^{i}sr^{-i} $:
        \[
            \begin{aligned}
                r^{i}sr^{-i}\marginnote{The form $ r^{i}sr^{-i} $ is called \textit{conjugation} and allows us to change the axis of operation.}: &v_{i+1}\mapsto v_1 \mapsto v_1 \mapsto v_{i+1},\\
                &v_{i+2} \mapsto v_2 \mapsto v_{n} \mapsto v_i,\\
                &v_i \mapsto v_n \mapsto v_2 \mapsto v_{i+2} \mapsto v_{i+2}.
            \end{aligned}
        \]
        We get reflection in axis through $ v_{i+1} $ and centre. If $ n $ is even, consider 
        \[
            \begin{aligned}
                r^{i+1}sr^{-i}:& v_{i+1} \mapsto v_1 \mapsto v_1 \mapsto v_{i+2},\\
                &v_{i+2} \mapsto v_2 \mapsto v_n \mapsto v_{i+1}.
            \end{aligned}
        .\]
        Hence they give all symmetries and $ D_{2n}=\langle r,s \rangle $ and $ rs=sr^{-1} $, so it is not abelian.
    \end{remark}
    \subsection{Presentation}
    One way to write groups is via a \textit{presentation}:
    \[
        \langle \text{generators}|\text{relation between generators} \rangle 
    .\]
    For example, $ C_n=\langle a|a^n=e \rangle  $, and $ D_{2n}=\langle r,s|r^n=e, s^2=e, rs=sr^{-1} \rangle  $.

    Should be able to deduce all the properties in the group from the relatios in the presentation. In general it is not easy to write down a presentation for a given group, or to determine the group from a given presentation. E.g., 
    \[
        \begin{aligned}
            &\langle a,b,c| aba^{-1}b^{-1}=b, bcb^{-1}c^{-1}=c, cac^{-1}a^{-1}=a \rangle\\
            &\langle a,b,c,d| aba^{-1}b^{-1}=b, bcb^{-1}c^{-1}=c,cdc^{-1}d^{-1}=d,dad^{-1}a^{-1}=a \rangle 
        \end{aligned}
    \]
    The first group is simply $\{e\}$ but the second group, known as Higman group, is very non-trivial.
    \subsection{Permutation groups}
    \begin{definition}
        Given a set $X$, a \textit{permutation} of $X$ is a bijective function $ \sigma:X\to X $. The set of all permutations of $X$ is denoted by $ \sym X $.
    \end{definition}
    Of course we have 
    \begin{theorem}\label{thm:perm_group}
        $ \sym X $ forms a group wrt compositions.
    \end{theorem}
    \begin{definition}
        If $ |X|=n $, we write $ S_n $ for (the isomorphism class of) $ \sym X $. $S_n$ is called \textit{symmetric group} on $n$ elements.
    \end{definition}
    \begin{remark}
        $ |S_n|=n! $. Usually use $ X=\left\{ 1,2,\dots,n\right\} $ to study $S_n$.
    \end{remark}
    One way to write permutations is using a two-row notation. For example, consider $ \sigma\in S_3 $ such that $ \sigma(1)=2, \sigma(2)=3, \sigma(3)=1 $ can be represented as
    \[
        \begin{pmatrix}
            1&2&3\\
            2&3&1
        \end{pmatrix}
    .\]
    In general, write $ \sigma\in S_n $ as 
    \[
        \begin{pmatrix}
            1&2&3&\cdots&n\\
            \sigma(1)&\sigma(2)&\sigma(3)&\cdots&\sigma(n)
        \end{pmatrix}
    .\]
    Given a permutation that "cycles" some elements $ a_1,\dots,a_k \in \left\{ 1,2,\dots,n\right\}$ and leaves the other unchanged, then we can write as
    \[
        (a_1 a_2 \dots a_k) = \begin{pmatrix}
            a_1&a_2&a_3&\cdots&a_k\\
            \sigma(a_1)&\sigma(a_2)&\sigma(a_3)&\cdots&\sigma(a_k)
        \end{pmatrix}
    .\]
    So in general,
    \[
        (a_1 \dots a_k)(x) =\begin{cases}
        a_{i+1} &\text{ if } x=a_i(i<k)\\
        a_1 &\text{ if } x=a_k\\
        x &\text{ otherwise.}\\
        \end{cases} 
    \]
    Note that $ (a_1 \dots a_k)=(a_2\dots a_{k} a_1)=\cdots $.
    \begin{definition}
        A permutation of the form $ \sigma = (a_1 \dots a_k) $ is called a $k$-\textit{cycle}. If $k=2$ then it is called a \textit{transposition}.
    \end{definition}
    \begin{example}
        \begin{enumerate}[(1).]
            \item Consider $ (1234)(324) $. $ 1 \mapsto 2 $, $ 2 \mapsto 1 $, $ 3 \mapsto 3 $, $4 \mapsto 4 $. Hence
            \[
                (1234)(324)=(12)
            .\]
            \item In $S_5$, $ (254)(534)=(1)(253)(4) = (253) $.
        \end{enumerate}
    \end{example}
    \begin{remark}
        \begin{enumerate}[(1).]
            \item The inverse of $ (a_1 \dots a_k) $ is $ (a_k a_{k-1}\dots a_1) $.
            \item $ S_3=D_6 $, but in general $ D_{2n}\le S_n $.
        \end{enumerate}
    \end{remark}
    \begin{definition}
        \begin{enumerate}[(1).]
            \item Two cycles are \textit{disjoint} if no element appears in both of them.
            \item $ g,h\in G $ are \textit{commute} if $gh=hg$ in $G$.
        \end{enumerate}
    \end{definition}
    \begin{lemma}\label{lma:disjoint_cyc_commute}
        Disjoint cycles commute.

        \bluecomment{Note that $S_n$ is non-abelian for $n\ge 3$.}
    \end{lemma}
    \begin{proof}
        Let $ \sigma, \tau\in S_n $ such that $ \sigma, \tau $ are disjoint. Let $ x\in \{1,2,\dots,n\} $.

        If $x$ is in neither of $\sigma, \tau$, then $ \sigma \tau(x)=\tau \sigma(x) $. If $ x\in \tau $ but not in $ \sigma $, then $ \tau(x)\in \tau\notin \sigma $, so $ \sigma \tau(x)=\tau \sigma(x)=\tau(x) $. Similar for $ x\in \sigma, x\notin \tau $.
    \end{proof}
    \begin{theorem}\label{thm:disjoint cycle decomp}
        Any $ \sigma\in S_n $ can be written as a composition of disjoint cycles, and this representation is unique up to reordering cycles, and "cycling" of cycles.
    \end{theorem}
    \begin{proof}
        Take $ \sigma\in S_n $ and consider $ 1,\sigma(1),\sigma^2(1),\dots $. Since $ \{1,2,\dots,n\} $ is finite, $ \exists a>b $, $ \sigma^a(1)=\sigma^b(1) $, so that $ \sigma^{a,b}(1)=1 $. Let $k$ be the smallest integer that $ \sigma^{k}(1)=1 $. Then $ \forall l>m\in [0,k] $, if $ \sigma^{l}(1)=\sigma^{m}(1) $ then $ \sigma^{l-m}=1 $, contradicting with the minimality of $k$, so $ 1,\sigma(1),\dots,\sigma^{k-1}(1) $ are distinct. This cycle 
        \[
            \begin{pmatrix}
                1&\sigma(1)&\sigma^2(x)&\cdots&\sigma^{k-1}(1)
            \end{pmatrix}
        \]
        is the first cycle in decomposition. We can repeat this with the next number in $\{1,2,\dots,n\}$ that has not already appeared.

        Since $ \sigma $ is a bijection, no number can reappear. Continue with this we exhaust $\{1,2,\dots,n\}$ and we get 
        \[
            \begin{pmatrix}
                1&\sigma(1)&\cdots&\sigma^{k-1}(1)
            \end{pmatrix}
            \begin{pmatrix}
                a&\sigma(a)&\cdots&\sigma^{k-1}(a)
            \end{pmatrix}
            \cdots
        .\]
        Hence it exists. To show it is unique, suppose we have to decompositions:
        \[
            \begin{aligned}
                 \sigma&= \begin{pmatrix} a_1&\cdots&a_{k_1} \end{pmatrix}\begin{pmatrix} a_{k_2}&\cdots&a_{k_3} \end{pmatrix}\cdots \begin{pmatrix} a_{k_{n-1}}&\cdots&a_{k_n} \end{pmatrix}\\
                 &= \begin{pmatrix} b_1&\cdots&b_{l_1} \end{pmatrix}\begin{pmatrix} b_{l_2}&\cdots&b_{l_3} \end{pmatrix}\cdots \begin{pmatrix} b_{l_{s-1}}&\cdots&b_{l_s} \end{pmatrix},
            \end{aligned}
        \]
        so each $ j\in \{1,2,\dots,n\} $ appears exactly once in both. Then we have $ a_1=b_t $ for some $t$, and the other numbers appearing in the cycle of $b_t$ are uniquely determined by $ \sigma(a_1),\sigma^2(a_1),\dots $. So
        \[
            \begin{pmatrix} a_1&\cdots&a_{k_1} \end{pmatrix}\cdots = \begin{pmatrix} b_t&\cdots \end{pmatrix}\cdots
        \]
        since disjoint cycles commute and we can cycle cycles. Continue in this way, we see that all other cycles match. 
    \end{proof}
    \begin{definition}\marginnote{Lecture 7}
        The set of cycle lengths of the disjoint cycle decomposition of $ \sigma $ is its \textit{cycle type} of $\sigma$.
    \end{definition}
    \begin{example}
        $ (123)(56) $ has cycle type 3,2(or 2,3).
    \end{example}
    \begin{theorem}\label{thm:lcm of cycle type is order}
        The order of $ \sigma\in S_n $ is the lcm of the cycle length in its cycle type.
    \end{theorem}
    \begin{proof}
        Firstly note that the order of a $k$-cycle is $k$. Suppose $ \sigma= \tau_1 \tau_2\cdots \tau_r $, where $ \tau_i $ are disjoint cycles, we have 
        \[
            \sigma^m = \tau^m_1 \tau^m_2\cdots \tau^m_r
        ,\]
        since disjoint cycles commute. Let each $ \tau_i $ be a $k_i$-cycle, then if $ \sigma^m=e $, we have $ \tau^m_1,\tau^m_2,\dots,\tau^m_r=e $, and so $ \tau_1^m=\tau_2^{-m}\tau_{3}^{-m}\cdots \tau_{r}^{-m} $. The numbers permuted by LHS and RHS are disjoint since $ \tau_i $ are disjoint, so LHS, RHS must be $e$. So $ \tau_1^m=e $ and $ k_1|m$.

        This holds for any $k_i$ and $k_i|m$, so $ l=\lcm (k_1,\dots,k_r)| \ord(\sigma) $. But if we take
        \[
            \sigma^{l}=\tau^l_1 \tau^l_2\cdots \tau^l_r = \prod_{i=1}^r (\tau^{k_i})^{l/k_i}=e
        .\]
        So $ \ord(\sigma)=\lcm (k_1,\dots,k_r) $.
    \end{proof}
    \begin{remark}
        Disjoint cycle notation allows us to quickly compare elements of $S_n$, and to read off their orders.
    \end{remark}
    Disjoint cycle notation is just one useful way to express elements of $S_n$. Another is as a product of transpositions:
    \begin{proposition}\label{prop:2.16}
        Let $ \sigma\in S_n $, then $\sigma$ is a product of transpositions.
    \end{proposition}
    \begin{proof}
        By theorem \ref{thm:disjoint cycle decomp}, it's enough to do this for a cycle. We observe that 
        \[
            (a_1 a_2 a_3 \cdots a_k)=(a_1a_2)(a_2a_3)\cdots(a_{k-1}a_k)
        .\]
    \end{proof}
    \begin{remark}
        This is not unique. e.g., (1234)=(12)(23)(34)=(12)(23)(12)(34)(12). But the \textit{parity} of the numbers of transpositions is well-defined..
    \end{remark}
    \begin{theorem}\label{thm:parity_transposition}
        Writing $ \sigma\in S_n $ as a product of transpositions in different ways, $\sigma$ is either always a product of an even number of transpositions, or always a product of an odd number of transpositions.
    \end{theorem}
    \begin{proof}
        Write $ \#(\sigma) $ for the number of cycles in $ \sigma $ in disjoint cycle decompositions, including any one-cycles. For example, $ \#((12)(34))=\#((123))=2, \#(e)=4 $. Let's see   what happens to $ \#(\sigma) $ if we multiply $ \sigma $ by a transposition $ \tau=(cd) $.
        \begin{itemize}[-]
            \item This will not affect any cycles not including $c$ or $d$.
            \item If $c,d$ are in the same cycle in (disjoint cycle decomposition) of $ \sigma$, say $ (c a_2 a_3 \cdots a_{k-1} d  a_{k+1} \cdots a_l) $, then 
            \[
                (c a_2 a_3 \cdots a_{k-1} d  a_{k+1} \cdots a_l)(cd)=(c a_{k+1} a_{k+2} \cdots a_l)(d a_2 \cdots a_{k-1})
            ,\]
            so $ \#(\sigma \tau) =\#(\sigma)+1$.
            \item If $c,d$ are in different cycles (possibly 1-cycle),
            \[
                (ca_2\cdots a_k)(db_2\cdots b_l)(cd)=(cdb_2\cdots b_ldca_2\cdots a_k)
            .\]
            So $ \# (\sigma\tau)=\#(\sigma)-1 $.
        \end{itemize}
        So far any $ \sigma $ and any transposition $ \tau $, $ \#(\sigma)\equiv\#(\sigma \tau)+1(\bmod 2)$. Now suppose $ \sigma $ is written as 2 different products of transpositions
        \[
            \sigma=\tau_{1} \cdots \tau_{k}=\tau_{1}^{\prime} \ldots \tau_{l}^{\prime}
        .\]
        We know by the previous theorem that $ \#(\sigma) $ is uniquely determined by $\sigma$. Also we have 
        \[
            \sigma=e\cdot \tau_{1} \cdots \tau_{k}= e\cdot \tau_{1}^{\prime} \ldots \tau_{l}^{\prime} 
        ,\]
        and so applying the above several times, we get 
        \[
            \#(\sigma)\equiv \#(e)+k\equiv n+k(\bmod 2); \#(\sigma)\equiv \#(e)+l\equiv n+l(\bmod 2)
        .\]
        So $ n+k\equiv n+2(\bmod 2) \Leftrightarrow k\equiv l(\bmod 2) $. Hence $k,l$ has the same parity.
    \end{proof}
    \begin{definition}
        Writing $ \sigma\in S_n $ as a product of transositions, $ \sigma =\tau_{1} \cdots \tau_{k}$, the \textit{sign} of $ \sigma $ is defined as $ \epsilon(\sigma)=(-1)^k $. If $ \epsilon(\sigma)=1 $, we say $ \sigma $ is an \textit{even} permutation, and odd permutation if $ \epsilon(\sigma)=-1 $.
    \end{definition}
    \begin{theorem}\label{thm:2.19}
        For $ n\ge 2 $, the sign function $ \epsilon: S_n\to \langle -1 \rangle $ is a surjective homomorphism.
    \end{theorem}
    \begin{proof}
        If $ \sigma,\sigma' $ can be written as $k,l$ transpositions respectively, then $ \sigma \sigma' $ can be written as a product of $k+l$ transpositions and $ \epsilon(\sigma \sigma')=(-1)^{k+l}=(-1)^k\cdot (-1)^l=\epsilon(\sigma)\cdot \epsilon(\sigma') $. To see it is surjective, since $n\ge 2$, $ \epsilon(e)=1 $ and $ \epsilon(12)=-1 $, so it is.
    \end{proof}
    \begin{definition}
        The \textit{kernel} of the homomorphism $ \epsilon $ is called the \textit{alternating group}, $ A_n \le S_n$.
    \end{definition}
    \begin{proposition}\label{prop:2.21}
        $ \sigma\in S_n $ is even if and only if its disjoint cycle decomposition contains an \textit{even number} of \textit{even} cycles.\marginnote{Here "even cycle" means a cycle of even number of elements.}
    \end{proposition}
    \begin{proof}
        Write 
        \[
            \sigma = \delta_1 \delta_2 \cdots \delta_k \chi_1 \chi_2\cdots \chi_{l}
        ,\]
        where $\delta$ are even cycles, and $\chi$ are odd cycles. Then $ \epsilon(\sigma)=(-1)^{k} $ and the result follows.
    \end{proof}
\end{document}