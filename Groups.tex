\documentclass[10pt]{article}
\pdfoutput=1 
\usepackage{NotesTeX,lipsum}

%\usepackage{showframe}

\title{\begin{center}{\Huge \textit{Groups}}\\{{\itshape Based on Lectures and "Algebra and Geometry"}}\end{center}}
\author{$\theta\omega\theta$}
\affiliation{
Not in University of Cambridge\\
skipped some takes irrelevant to contents\\
}

\emailAdd{not telling you}

\begin{document}
	\maketitle
	\flushbottom
	\newpage
	\pagestyle{fancynotes}
	\part{Groups and Permutations}

    \section{Definition of Groups}
    \begin{definition}[Group]
        A group is a set $G$ together with a binary operation $ \ast: G\times G \to G $ that 
        \begin{enumerate}
            \item (Closure) $ \forall g,h\in G, g\ast h $,
            \item (Identity)$ \exists e\in G, \forall g\in G, e*g = g*e = g $,
            \item (Inverse)$ \forall g\in G, \exists g^{-1}\in G, g * g^{-1} = g^{-1}*g = e $,
            \item (Associativity)$ \forall g,h,k\in G, (g*h)*k = g*(h*k) $. 
        \end{enumerate}
    \end{definition}
    \begin{remark}
        The inverse of $g$ is unique, for if there are two $g',g''$, both are inverses of $g$, we have 
        \[
            g' = g''*g*g' = g''
        .\]
    \end{remark}
    \begin{example}
        \begin{enumerate}[(1)]
            \item $G = \left\{ e\right\}$, the trivial group,
            \item $ G = \left\{ \text{symmetries of } \triangle \right\} $,
            \item $ (\mathbb{Z} , +) $,
            \item $ (\mathbb{R} ,+), (\mathbb{Q} , +), (\mathbb{C} , +) $,
            \item $ \mathbb{R}^* = \mathbb{R} \setminus \left\{ 0\right\}; (\mathbb{R}^*, \times) $,
            \item $ (\mathbb{Z}_n, + \bmod n), \mathbb{Z}_n = \left\{ 0,1,\dots, n-1\right\} $,
            \item Vector spaces with addition of vectors,
            \item $ (\rm{GL}_2(\mathbb{R}), \text{matrix multiplication}) $, set of invertible $2\times 2$ matrices,
        \end{enumerate}
    \end{example}
    \begin{example}[non-examples]
        \begin{enumerate}[(1)]
            \item $ (\mathbb{Z}_n, +) $, since it is not closed,
            \item $ (\mathbb{Z} , \times) $, since some inverses do not exist,
            \item $ (\mathbb{R} , *) $, where $ r*s = r^2 s $, since there is no identity,
            \item $ (\mathbb{N}, *), n*m = |n-m| $.\marginnote{Here $ \mathbb{N} $ is the set of all positive numbers, and it remains this definition unless specified otherwise.} Associativity fails.
        \end{enumerate}
    \end{example}
    \section{Propertie of Groups}
    \begin{proposition}\label{prop:groups_properties}
        Let $G$ be a group, then we have 
        \begin{enumerate}
            \item The identity is unique.
            \item THe inverse is unique.
            \item $ gh=g \land hg=g \Rightarrow h=e $.
            \item $ gh=e \Rightarrow hg=e, h=g^{-1} $.
            \item $ (g^{-1})^{-1}=g $.
        \end{enumerate}
    \end{proposition}
    \begin{definition}
        A group $G$ is called \textit{abelian} if $ \forall g,h\in G, gh=hg $.
    \end{definition}
    \begin{definition}
        $G$ is said to be \textit{finite} if it has finitely may elements. Denote $|G|$ as its number of elements.
    \end{definition}
    \begin{definition}
        Let $ (G,*) $ be a group. A subset $H\subseteq G$ is called a \textit{subgroup} of $G$ if $ (H,*) $ is a group, written as $ H\le G $.
    \end{definition}
    \begin{remark}
        To check $ H\le G $, simply check closure, identity, and inverses. Associativity is inherited.
    \end{remark}
    \begin{proposition}\label{prop:uniqueness of identity}
        Let $ e_H, e_G $ be the identities in $H$ and $G$ respectively, then $e_H=e_G$. 
    \end{proposition}
    \begin{example}
        \begin{enumerate}[(1)]
            \item $ \left\{ e\right\} \le G $.
            \item $ G\le G $.
            \item $ (\mathbb{Z} ,+)\le (\mathbb{Q} ,+)\le (\mathbb{R} ,+)\le (\mathbb{C} ,+) $.
        \end{enumerate}
    \end{example}
    \begin{lemma}[subgroup test]\label{lma:subgroup test}
        Let $ G $ be a group, then $ H\le G \Leftrightarrow H\neq \varnothing \land \forall a,b\in H, ab^{-1}\in H $.
    \end{lemma}
    \begin{proof}
        Since $gg^{-1}=e\in H$, identity is satisfied. Since $ \forall a,b\in H, a(b^{-1})^{-1}=ab\in H $, closure is satisfied. $ \forall g\in H, eg^{-1}=g^{-1}\in H $, inverse is satisfied.
    \end{proof}
    \begin{proposition}\label{prop:subgroups of integers}
        The subgroups of $ (\mathbb{Z} ,+) $ are precisely $ (n \mathbb{Z} ,+) $.
    \end{proposition}
    Proved by considering the minimal element.

    Usual laws:
    \begin{proposition}\label{prop:comparing_groups}
        \begin{enumerate}[(1)]
            \item Let $ H,K $ be subgroups of $G$ then $ H\cap K\le G $.
            \item $ K\le H \land H\le G \Rightarrow K\le G $.
            \item $ K \subseteq H, H\le G, K\le G \Rightarrow K\le H $.
        \end{enumerate}
    \end{proposition}
    \begin{definition}
        If $ X\neq \varnothing $ is a subset of group $G$, the subgroup \textit{generated} by $X$, written as $ \langle X \rangle $, is the intersection of all subgroups containing $X$.
    \end{definition}
    \begin{remark}
        \begin{itemize}
            \item $ e\in \langle X \rangle $.
            \item $ X \subseteq \langle X \rangle $.
            \item $ \langle X \rangle $ contains all possible products of elements of $X$ and their inverses.
        \end{itemize}
    \end{remark}
    \begin{proposition}\label{prop:generation_of_groups}
        Let $ \varnothing \neq X \subseteq G $. Then $ \langle X \rangle $ is the set of elements of $ G $ of the form
        \[
            x_1^{\alpha_1}x_2^{\alpha_2}\cdots x_k^{r_k},\quad x_i\in X, \alpha_i\in \left\{ -1,1\right\}, k\ge 0
        .\]
    \end{proposition}
    \begin{proof}
        Let $T$ be such a set. Then by definition $ T \subseteq \langle X \rangle $. On the other hand, $ X \subseteq T \Rightarrow \langle X \rangle  \subseteq T$ since $T$ clearly forms a subgroup. Hence $ T=\langle X \rangle $.
    \end{proof}
    \section{Homomorphisms}
    \begin{definition}
        Let $ (G,*_G), (H,*_H) $ be groups. A function $ \varphi: H \to G $ is a \textit{homomorphism} if
        \[
            \forall a,b\in H, \varphi (a*_HB)=\varphi (a)*_G \varphi (b)
        .\]
        It is called an \textit{isomorphism} if it is bijective.
    \end{definition}
    \begin{proposition}\label{prop:homom}
        Let $ \varphi :H\to G $ be a homomorphism.
        \begin{enumerate}[(1)]
            \item $ \varphi (e_H)=e_G $.
            \item $ \varphi (h^{-1})=\varphi (h)^{-1} $.
            \item If $ \psi:G\to K $ is also a homomorphism, then $ \psi \varphi :H\to K $ is a homomorphism.
        \end{enumerate}
    \end{proposition}
    \begin{proposition}\label{prop:isom_inverse_is_also_an_isom}
        Let $ \varphi :H\to G $ be an isomorphism. Then $ \varphi^{-1} $ is also an isomorphism and this implies that 
        \[
            G \cong H \Longleftrightarrow H \cong G
        .\]
    \end{proposition}
\end{document}