\documentclass[10pt]{article}
\pdfoutput=1 
\usepackage{NotesTeX,lipsum}
\DeclareMathOperator{\im}{Im}
\renewcommand{\le}{\leqslant}
\renewcommand{\ge}{\geqslant}

%\usepackage{showframe}

\title{\begin{center}{\Huge \textit{Groups}}\\{{\itshape Based on Lectures and "Algebra and Geometry"}}\end{center}}
\author{$\theta\omega\theta$}
\affiliation{
Not in University of Cambridge\\
skipped some takes irrelevant to contents\\
}

\emailAdd{not telling you}

\begin{document}
	\maketitle
	\flushbottom
	\newpage
	\pagestyle{fancynotes}
	\part{Groups and Permutations}

    \section{Definition of Groups}
    \begin{definition}[Group]
        A group is a set $G$ together with a binary operation $ \ast: G\times G \to G $ that 
        \begin{enumerate}
            \item (Closure) $ \forall g,h\in G, g\ast h $,
            \item (Identity)$ \exists e\in G, \forall g\in G, e*g = g*e = g $,
            \item (Inverse)$ \forall g\in G, \exists g^{-1}\in G, g * g^{-1} = g^{-1}*g = e $,
            \item (Associativity)$ \forall g,h,k\in G, (g*h)*k = g*(h*k) $. 
        \end{enumerate}
    \end{definition}
    \begin{remark}
        The inverse of $g$ is unique, for if there are two $g',g''$, both are inverses of $g$, we have 
        \[
            g' = g''*g*g' = g''
        .\]
    \end{remark}
    \begin{example}
        \begin{enumerate}[(1)]
            \item $G = \left\{ e\right\}$, the trivial group,
            \item $ G = \left\{ \text{symmetries of } \triangle \right\} $,
            \item $ (\mathbb{Z} , +) $,
            \item $ (\mathbb{R} ,+), (\mathbb{Q} , +), (\mathbb{C} , +) $,
            \item $ \mathbb{R}^* = \mathbb{R} \setminus \left\{ 0\right\}; (\mathbb{R}^*, \times) $,
            \item $ (\mathbb{Z}_n, + \bmod n), \mathbb{Z}_n = \left\{ 0,1,\dots, n-1\right\} $,
            \item Vector spaces with addition of vectors,
            \item $ (\rm{GL}_2(\mathbb{R}), \text{matrix multiplication}) $, set of invertible $2\times 2$ matrices,
        \end{enumerate}
    \end{example}
    \begin{example}[non-examples]
        \begin{enumerate}[(1)]
            \item $ (\mathbb{Z}_n, +) $, since it is not closed,
            \item $ (\mathbb{Z} , \times) $, since some inverses do not exist,
            \item $ (\mathbb{R} , *) $, where $ r*s = r^2 s $, since there is no identity,
            \item $ (\mathbb{N}, *), n*m = |n-m| $.\marginnote{Here $ \mathbb{N} $ is the set of all positive numbers, and it remains this definition unless specified otherwise.} Associativity fails.
        \end{enumerate}
    \end{example}
    \section{Properties of Groups}
    \begin{proposition}\label{prop:groups_properties}
        Let $G$ be a group, then we have 
        \begin{enumerate}
            \item The identity is unique.
            \item THe inverse is unique.
            \item $ gh=g \land hg=g \Rightarrow h=e $.
            \item $ gh=e \Rightarrow hg=e, h=g^{-1} $.
            \item $ (g^{-1})^{-1}=g $.
        \end{enumerate}
    \end{proposition}
    \begin{definition}
        A group $G$ is called \textit{abelian} if $ \forall g,h\in G, gh=hg $.
    \end{definition}
    \begin{definition}
        $G$ is said to be \textit{finite} if it has finitely may elements. Denote $|G|$ as its number of elements.
    \end{definition}
    \begin{definition}
        Let $ (G,*) $ be a group. A subset $H\subseteq G$ is called a \textit{subgroup} of $G$ if $ (H,*) $ is a group, written as $ H\le G $.
    \end{definition}
    \begin{remark}
        To check $ H\le G $, simply check closure, identity, and inverses. Associativity is inherited.
    \end{remark}
    \begin{proposition}\label{prop:uniqueness of identity}
        Let $ e_H, e_G $ be the identities in $H$ and $G$ respectively, then $e_H=e_G$. 
    \end{proposition}
    \begin{example}
        \begin{enumerate}[(1)]
            \item $ \left\{ e\right\} \le G $.
            \item $ G\le G $.
            \item $ (\mathbb{Z} ,+)\le (\mathbb{Q} ,+)\le (\mathbb{R} ,+)\le (\mathbb{C} ,+) $.
        \end{enumerate}
    \end{example}
    \begin{lemma}[subgroup test]\label{lma:subgroup test}
        Let $ G $ be a group, then $ H\le G \Leftrightarrow H\neq \varnothing \land \forall a,b\in H, ab^{-1}\in H $.
    \end{lemma}
    \begin{proof}
        Since $gg^{-1}=e\in H$, identity is satisfied. Since $ \forall a,b\in H, a(b^{-1})^{-1}=ab\in H $, closure is satisfied. $ \forall g\in H, eg^{-1}=g^{-1}\in H $, inverse is satisfied.
    \end{proof}
    \begin{proposition}\label{prop:subgroups of integers}
        The subgroups of $ (\mathbb{Z} ,+) $ are precisely $ (n \mathbb{Z} ,+) $.
    \end{proposition}
    Proved by considering the minimal element.

    Usual laws:
    \begin{proposition}\label{prop:comparing_groups}
        \begin{enumerate}[(1)]
            \item Let $ H,K $ be subgroups of $G$ then $ H\cap K\le G $.
            \item $ K\le H \land H\le G \Rightarrow K\le G $.
            \item $ K \subseteq H, H\le G, K\le G \Rightarrow K\le H $.
        \end{enumerate}
    \end{proposition}
    \begin{definition}
        If $ X\neq \varnothing $ is a subset of group $G$, the subgroup \textit{generated} by $X$, written as $ \langle X \rangle $, is the intersection of all subgroups containing $X$.
    \end{definition}
    \begin{remark}
        \begin{itemize}
            \item $ e\in \langle X \rangle $.
            \item $ X \subseteq \langle X \rangle $.
            \item $ \langle X \rangle $ contains all possible products of elements of $X$ and their inverses.
        \end{itemize}
    \end{remark}
    \begin{proposition}\label{prop:generation_of_groups}
        Let $ \varnothing \neq X \subseteq G $. Then $ \langle X \rangle $ is the set of elements of $ G $ of the form
        \[
            x_1^{\alpha_1}x_2^{\alpha_2}\cdots x_k^{r_k},\quad x_i\in X, \alpha_i\in \left\{ -1,1\right\}, k\ge 0
        .\]
    \end{proposition}
    \begin{proof}
        Let $T$ be such a set. Then by definition $ T \subseteq \langle X \rangle $. On the other hand, $ X \subseteq T \Rightarrow \langle X \rangle  \subseteq T$ since $T$ clearly forms a subgroup. Hence $ T=\langle X \rangle $.
    \end{proof}
    \section{Homomorphisms}
    \subsection{Definition and basic properties}
    \begin{definition}
        Let $ (G,*_G), (H,*_H) $ be groups. A function $ \varphi: H \to G $ is a \textit{homomorphism} if
        \[
            \forall a,b\in H, \varphi (a*_HB)=\varphi (a)*_G \varphi (b)
        .\]
        It is called an \textit{isomorphism} if it is bijective.
    \end{definition}
    \begin{proposition}\label{prop:homom}
        Let $ \varphi :H\to G $ be a homomorphism.
        \begin{enumerate}[(1)]
            \item $ \varphi (e_H)=e_G $.
            \item $ \varphi (h^{-1})=\varphi (h)^{-1} $.
            \item If $ \psi:G\to K $ is also a homomorphism, then $ \psi \varphi :H\to K $ is a homomorphism.
        \end{enumerate}
    \end{proposition}
    \begin{proposition}\label{prop:isom_inverse_is_also_an_isom}
        Let $ \varphi :H\to G $ be an isomorphism. Then $ \varphi^{-1} $ is also an isomorphism and this implies that 
        \[
            G \cong H \Longleftrightarrow H \cong G
        .\]
    \end{proposition}
    %Lecture 4
    \subsection{Images and Kernels}\marginnote{Lecture 4.}
    \begin{definition}
        The \textit{image} of a homomorphism $ \varphi: H \to G $ is 
        \[
            \operatorname{Im}(\varphi)=\left\{ g\in G: \exists h\in H, \varphi(h)=g \right\}
        .\]
        The \textit{kernel} of $ \varphi $ is 
        \[
            \ker(\varphi)=\left\{ h\in H: \varphi(h)=e_G \right\}
        .\]
    \end{definition}
    We have two immediate consequences:
    \begin{proposition}\label{prop:ker, im are groups}
        $ \operatorname{Im}(\varphi), \ker (\varphi)  $ are subgroups of $G,H$ respectively.
    \end{proposition}
    \begin{proof}
        Take $ \operatorname{Im}(\varphi) $ as an example. Use lemma \ref{lma:subgroup test}: $ \operatorname{Im}(\varphi)  $ is non-empty since $ \varphi(e_H)=e_G $. For any $ a,b\in \operatorname{Im}(\varphi)  $, we have $ a=\varphi(h), b=\varphi(h') $ for $h,h'\in H$. Hence
        \[
            ab^{-1}=\varphi(h)\varphi(h')^{-1}=\varphi(hh'^{-1})\in \operatorname{Im}(\varphi) 
        .\]
        Hence $ \operatorname{Im}(\varphi)  $ is a subgroup. It is similar for $ \ker (\varphi) $.
    \end{proof}
    \begin{example}
        \begin{enumerate}[(1)]
            \item[(0)] Let $ \varphi:H\to G $ be the trivial homomorphism, i.e. $ \varphi(h)\equiv e_G $. Then $ \im(\varphi)=\left\{ e_G\right\} $ and $ \ker (\varphi)=H $.
            \item Let $ \iota: H\to G $, where $H\le G$, be the inclusion map. Then $ \im (\iota) = H, \ker (\iota) = \left\{ e_H\right\} $.
            \item $ \varphi: \mathbb{Z} \to \mathbb{Z}_n, \varphi(k)=k \bmod n $. $ \im (\varphi)=\mathbb{Z}_n, \ker (\varphi)= n \mathbb{Z}$.
        \end{enumerate}
    \end{example}
    \begin{proposition}\label{prop:homom surj}
        Let $ \varphi:H\to G $ be a homomorphism.
        \begin{enumerate}[(1)]
            \item $ \varphi $ is surjective if and only if $ \im \varphi = G $,
            \item $ \varphi $ is injective if and only if $ \ker \varphi = \left\{ e\right\} $.
        \end{enumerate}
    \end{proposition}
    \begin{proof}
        By definition, (1) holds.

        Suppose $ \varphi $ is injective. Take $ h\in \ker \varphi $. Then $ \varphi(h)=\varphi(e)=e_G \Leftrightarrow h=e $. Conversely suppose $ \ker \varphi=\left\{ e\right\} $. Take $ a,b $ such that $ \varphi(a)=\varphi(b) $. We have
        \[
            \varphi(ab_{-1})=\varphi(a)\varphi(b)^{-1}=e_G
        .\]
        Thus $ ab^{-1}=e_G \Leftrightarrow a=b $ and $ \varphi $ is injective.
    \end{proof}
    \section{Direct product of groups}
    \begin{definition}
        The \textit{direct product} of two groups $ G,H $ is the set $ G\times H $ with the operation of component-wise composition:
        \[
            (g_1,h_1) * (g_2,h_2):=(g_1*_G g_2, h_1 *_H h_2)
        .\]
    \end{definition}
    Closure and identity are easily verified. The inverse is component-wise and associativity is inherited from $G,H$.
    \begin{remark}
        $ G\times H $ contains subgroups isomorphic to $G$ and $H$, i.e., $ G \times \{e_H\} $ and $ \{e_G\} \times H $.
    \end{remark}
    \begin{example}
        $ \mathbb{Z} \times \{-1,1\} $ has elements $ (n,\pm 1), n\in \mathbb{Z} $ with $ (n,-1)*(m,-1)=(n+m,(-1)(-1))=(n+m,1) $, etc. Addition in the first component and multiplication in the second.

        The identity of $\mathbb{Z} \times \{-1,1\}$ is $(0,1)$.
    \end{example}
    \begin{remark}
        In $ G \times H $, everything in (the isomorphic copy of) $G$ \textit{commutes} with everything in (the isomorphic copy of) $H$. That is to say,
        \[
            \forall (g,e_H), (e_G,h), (g,e_H)*(e_G,h) = (e_G, h)*(g,e_H)=(g,h)
        .\]
    \end{remark}
    \begin{theorem}[Direct Product Theorem]\label{thm:Direct Product Theorem}
        \marginnote{This gives us two ways to think about direct products:
        \begin{itemize}
            \item Given two groups $H,K$, one can form their direct products $ H \times K $ and view $H,K$ as subgroups via $ H \times \{e_K\} $ and $ \{e_H\}\times K $.
            \item Given a group $G$ with subgroups $H,K$ that satisfiy these conditions, then we are equivalently dealing with $ H \times K $.
        \end{itemize}
        By convention, we can simply regard $ H \times \{e_K\},\{e_H\}\times K $ as $ H,K. $}
        Let $ H,K\le G $ such that
        \begin{enumerate}[(1)]
            \item $ H \cap K=\left\{ e\right\} $: they are \textit{disjoint},
            \item $ \forall h,k, hk=kh $: they are \textit{commutative},
            \item $ \forall g\in G, \exists h\in H, k\in K, g=hk $: $ G=HK $.
        \end{enumerate}
        Then $ G \cong H \times K $.
    \end{theorem}
    \begin{proof}
        Consider the function $ \varphi: H\times K \to G $ defined by $ \varphi(h,k)=hk $. Note that 
        \[
            \varphi(h,k) * \varphi(h',k') = hkh'k'=hh'kk'=\varphi(hh',kk')=\varphi((h,k)*(h',k'))
        ,\]
        so $ \varphi $ is a homomorphism. From (3) we know that $ \varphi $ is surjective. Let $ \varphi(h,k)=e $, then $ hk=e \Leftrightarrow h=k^{-1} $. Hence $ h,k^{-1}\in H \cap K $ so $ h=k=e $. Hence it is injective. Thus $\varphi$ is an isomorphism and $ G \cong H \times K $.
    \end{proof}
    \section{Important Examples}
    \subsection{Cyclic groups}
    \begin{definition}
        Let $G$ be a group and let $ X \subseteq G, X\neq \varnothing $. If $ \langle X \rangle =G $, then $X$ is called \textit{a generating set}\mn{It is not necessary unique.} of $G$.

        $G$ is \textit{cyclic} if $\exists a\in G$ such that $ \langle a \rangle =G $. In this case, $ \forall b\in G, \exists k\in \mathbb{Z} , b=a^k $. $a$ is called a \textit{generator} of $G$.
    \end{definition}
    \begin{example}
        \begin{enumerate}[(1)]
            \item[(0)] Trivial group $ \left\{ e\right\}=\langle e \rangle  $.
            \item $ (\mathbb{Z} ,+)=\langle 1 \rangle =\langle -1 \rangle  $.
            \item $ (\mathbb{Z}_n, +_n)=\langle 1 \rangle =\langle k \rangle $, where $(k,n)=1$.
            \item $ E=\left(\left\{ e^{\frac{2\pi ik}{n}}:0\le k\le n-1 \right\}, \cdot\right) =\langle e^{\frac{2\pi i m}{n}} \rangle $, where $(m,n)=1$.\marginnote{Hence, $ E \cong \mathbb{Z}_n $.}
            \item $ \left\{ e,a,a^2,\dots, a^{n-1}\right\} $ with
            \[
                a^k * a^j = \begin{cases}
                a^{k+j} &\text{if } k+j<n,\\
                a^{k+j-n} &\text{if } k+j\ge n.\\
                \end{cases} 
            \]
            Again, it is isomorphic to $ \mathbb{Z}_n $.
        \end{enumerate}
    \end{example}
    Write $ C_n = \left\{ e,a,a^2,\dots, a^{n-1}\right\} $. Then every cyclic group is isomorphic to $C_n$ and we can write all cyclic groups in this form, or $ \cong \mathbb{Z} $, which is the infinite case.
\end{document}