\section{Classical Probability}
\subsection{Probability Space}
\begin{definition}[Probability Space]
    Suppose $ \Omega $ is a set(called \textbf{sample space}) and $ \mathcal{F} $ is a collection of subsets of $ \Omega $. We call $\mcF$ a \textbf{$\sigma$-algebra} if 
    \begin{enumerate}
        \item $ \Omega\in \mathcal{F} $,
        \item $ A\in \mathcal{F} \Rightarrow A^\complement \in \mathcal{F} $,
        \item For any \textit{countable} collection $ (A_n)_{n\ge 1} $ with $A_n\in \mcF$, $ \bigcup_{n=1}^{\infty}A_n\in \mathcal{F} $.
    \end{enumerate}
    Suppose $\mcF$ is a $ \sigma $-algebra on $ \Omega $. A function $ \mathbb{P}:\mathcal{F}\to [0,1] $ is called a \textbf{probability measure} if
    \begin{enumerate}
        \item $ \bbP(\Omega)=1 $,
        \item For any \textit{countable disjoint} collection $ (A_n)_{n\ge 1} $ in $ \mathcal{F} $, 
        \[
            \mathbb{P}\left( \bigcup_{n=1}^{\infty}A_n \right) = \sum_{n=1}^{\infty} \mathbb{P}(A_n).
        \]
        We say $ \mathbb{P}(A) $ is the \textbf{probability} of $A$.
    \end{enumerate}
    We call $ (\Omega,\mathcal{F},\mathbb{P}) $ a \textbf{probability space}.
\end{definition}
\begin{remark}
    When $ \Omega $ is countable, we take $\mcF$ be all subsets of $ \Omega $.
\end{remark}
\begin{definition}[Outcomes, Events]
    The elements of $ \Omega $ are called \textbf{outcomes} and the elements of $\mcF$ are called \textbf{events}.
\end{definition}
\begin{remark}
    We talk about probabilities of \textit{events} rather than \textit{outcomes}.
\end{remark}
\begin{proposition}[Properties of $\bbP$]\label{prop:Properties of P}
    Immediate from the definition,
    \begin{itemize}
        \item $ \mathbb{P}(A^\complement) = 1 - \mathbb{P}(A) $.
        \item $ \mathbb{P}(\varnothing)=0 $.
        \item If $ A \subseteq B $, then $ \mathbb{P}(A) \le \mathbb{P}(B) $.
        \item $ \mathbb{P}(A \cup B) = \mathbb{P}(A)+\mathbb{P}(B)-\mathbb{P}(A \cap B) $.
    \end{itemize}
\end{proposition}
\begin{example}
    \underline{Rolling a fair die.} Here $ \Omega = \{1,2,3,4,5,6\}, \mathcal{F} = 2^\Omega $. For $ \omega\in \Omega$, $\mathbb{P}(\{\omega\})=1/6 $, and if $ A \subseteq \Omega, \mathbb{P}(A)=|A|/6 $, since all outcomes are \textit{equally likely}.
\end{example}
\begin{example}
    \underline{Equally likely outcomes.} Let $ \Omega = \{\omega_1,\dots, \omega_n\} $ be a finite set and $ \mathcal{F}=2^\Omega $. Define $\bbP$ as $ \mathbb{P}(A)=|A|/|\Omega| $. In classical probability, this models picking a random element of $ \Omega $. Note that $ \mathbb{P}(\{\omega\})=1/|\Omega|, \forall \omega\in \Omega $.
\end{example}
\begin{example}
    \underline{Picking balls from a bag.} Suppose we have $n$ balls with $n$ labels $ \{1,\dots,n\} $ so they are distinct. Pick $k\le n$ balls at ramdom \textit{without replacement}. Take $ \Omega = \{A \subseteq \{1,\dots,n\}: |A|=k\} $, so $ |\Omega|=\binom{n}{k} $. $ \mathbb{P}(\{\omega\}) = 1/|\Omega| = 1/\binom{n}{k} $.
\end{example}
\begin{example}
    \underline{Deck of cards.} Take a \textit{well-shuffled} deck of 52 cards. Well-shuffled means all possible permutations are equally likely. Let $ \Omega = \{\text{all permutations of 52 cards}\}, |\Omega|=52! $. Now $ \mathbb{P}(\text{top 2 cards are aces}) = 4 \cdot 3 \cdot 50!/52! = 1/221$.
\end{example}
\begin{example}
    \underline{Largest digit.} Consider a string of $n$ random digits from 0 to 9. $ \Omega = \{0,\dots,9\}^n $, $ |\Omega| = 10^n, \mathcal{F}=2^\Omega  $. Let 
    \[
         A_k = \{\text{no digit exceeds }k\}, B_k =\{\text{largest digit is }k\} .
    \]
    $ \mathbb{P}(B_k) = |B_k|/10^n $. Notice $ B_k = A_k \setminus A_{k-1} $ and $ |A_k| = (k+1)^n $, so $ |B_k|=(k+1)^n-k^n $, so $ \mathbb{P}(B_k) = \frac{(k+1)^n-k^n}{10^n} $.
\end{example}
\begin{example}
    \underline{Birthday problem.} There are $n$ people. What is the probability that at least 2 of them share the same birthday?\footnote{Assume that every year has 365 days. Also assume each birthday is equally likely.} So $ \Omega = \{1,\dots,365\}^n $, $ \mathcal{F} = 2^\Omega, \mathbb{P}(\{\omega\}) = 1/365^n $. Let 
    \[
        A=\{\text{at least 2 people share the same birthday}\} ,
    \]
     then $ A^\complement = \{\text{all birthdays are different}\} $ and since $ \mathbb{P}(A)=1-\mathbb{P}(A^\complement) $, it suffices to calculate $ \mathbb{P}(A^\complement) = |A^\complement|/|\Omega| = 365 \cdot 364 \cdot \cdots \cdot (365-n+1)/365^n $ and hence 
    \[
        \mathbb{P}(A) = 1-\frac{365 \cdot 364 \cdot \cdots \cdot (365-n+1)}{365^n}.
    \]
    Note that $n=22, \mathbb{P}(A) \approx 0.476$ and $ n=23, \mathbb{P}(A)\approx 0.507 $. 
\end{example}

\subsection{Combinatorial Analysis}

\begin{question}{1}
    Suppose $ |\Omega|=n $. Want to partition $ \Omega $ into $k$ \textit{disjoint} subsets $ \Omega_1,\dots,\Omega_k$, $ |\Omega_i|=n_i, \sum_{i=1}^{k}n_i=n $. How many ways are there to do so?
\end{question}

\begin{definition}[Multinomial Coefficient]
    Let $ M=\# \text{ of ways} $, then 
    \[
        M= \binom{n}{n_1}\cdots \binom{n-(n_1+\cdots+n_{k-1})}{n_k} = \frac{n!}{n_1!n_2!\cdots n_k!}
        =: \binom{n}{n_1,n_2,\dots,n_k}
    \]
    is called \textbf{multinomial coefficient} and it is the answer to the question above.
\end{definition}

\begin{question}{2}
    How many strictly increasing functions $ f: \{1,\dots,k\} \to \{1,\dots,n\}$ exist?
\end{question}

Any function is uniquely determined by its range which is a subset of $ \{1,\dots,n\} $ of size $k$. There are $ \binom{n}{k} $ such subsets, and hence $ \binom{n}{k} $ such functions.

\begin{question}{3}
    How many non-decreasing functions $ f: \{1,\dots,k\} \to \{1,\dots,n\}$ exist?
\end{question}

Define a bijection from $ \{f: \{1,\dots,k\}\to \{1,\dots,n\}: f \text{ is non-decreasing}\} $ to $ \{g: \{1,\dots,k\}\to \{1,\dots,n+k-1\}: g \text{ is strictly increasing}\} $. $ \forall f \nearrow $, define $ g(i)=f(i+i-1) $. It is clear that $g$ is strictly increasing and takes values $ \{1,\dots,n+k-1\} $, and vice versa.
\begin{proposition}[Combination with Repetition]\label{prop:Combination with Repetition}
    The number of ways to choose $k$ items from $n$ items with repetition is 
    \[
        \binom{n+k-1}{n},
    \]
    and it is the answer to question 3.
\end{proposition}
\subsection{Stirling's Formula, Countable Subadditivity property}