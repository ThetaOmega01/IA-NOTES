\section{Multidimensional Gaussian r.v.s}
\begin{definition}
    A random variable $X$ in $\mathbb R$ is called \textbf{Gaussian} if $X=\mu+\sigma Z$ for $Z\sim\mathcal N(0,1)$ and $ \mu\in \mathbb{R}, \sigma\in [0,\infty) $.
    It has density, as we have seen,
    $$f_X(x)=\frac{1}{\sqrt{2\pi\sigma^2}}e^{-(x-\mu)^2/(2\sigma^2)}$$
    and we denote this by $X\sim\mathcal N(\mu,\sigma^2)$.
    We want to generalize this to higher dimensions.
\end{definition}
\begin{definition}
    A random variable $\bfX=(X_1,\ldots,X_n)$ is \textbf{Gaussian} if for any $\bfu\in\mathbb R^n$, $\bfu^\top \bfX$ is Gaussian in $\mathbb R$.
    We call $\bfX$ a \textbf{Gaussian vector}.
\end{definition}

\begin{example}
    Suppose that $ \bfX $ is Gaussian and we have an $m\times n$ matrix $A$ and $\bfb\in\mathbb R^m$, then $A\bfX+\bfb$ is also Gaussian. Indeed, let $ \bfu\in \mathbb{R}^{m} $, then 
    \[
        \bfu^\top (A\bfX+\bfb) = (\bfu^\top A)\bfX+\bfb.
    \]
    Set $ \bfv = A^\top \bfu $, then 
    \[
        \bfu^\top (A\bfX+\bfb) = \bfv^\top \bfX + \bfu^\top \bfb.
    \]
    Since $\bfX$ is Gaussian, $ \bfv^\top \bfX $ is Gaussian and $ \bfv^\top \bfX + \bfu^\top \bfb $ is Gaussian.
\end{example}