\section{Limits and convergence}
\subsection{Review for Numbers and Sets}
\begin{enumerate}[align=left]
    \item[\textit{Sequences}.] $ a_n, (a_n)_{n=1}^{\infty}, a_n\in \mathbb{R}  $. $ a_n\to a $ as $n\to \infty$ if $ \forall \epsilon>0, \exists N $ such that $ \forall n\ge N, |a_n-a|<\epsilon $. $N$ dependents on $\epsilon$. 
    \item[\textit{Monotone sequences}.] $ a_n\le a_{n+1} $ or $ a_n\ge a_{n+1} $. No ``equal'' is in the relation if we add ``strictly''.
    \item[\textit{Fundamental axiom of the real numbers}.] If $ a_n\in \mathbb{R}  $, $ \forall n\ge 1 $, $A\in \mathbb{R}$, $ a_n \nearrow $ and $ \forall i, a_i\le A $ then $ \exists a\in \mathbb{R} $ such that $a_n\to a$. ``An increasing sequence of real numbers bounded above converges.'' Equivalent for decreasing sequences. Equivalent also to ``every non-empty set of real numbers bounded above has a supremum'', the least upper bound property.
    \item[\textit{Supremum and infimum}.] For $ S \subseteq \mathbb{R}, S\neq \varnothing  $, $ \sup S=k $ if $ \forall x\in S, x\le k $ and $ \forall \epsilon>0, \exists x\in S $ such that $ x>k-\epsilon $. Similar for infimum.
    \item[\textit{Extension to $\mathbb{C}$}.] The above things make perfect sense for $ a_n\in \mathbb{C} $, with $ |\cdot| $ defined as modulus. However, complex numbers \textit{do not} have an order.
\end{enumerate}

Several useful deductions:
\begin{lemma}\label{lma:1.1}
    \begin{enumerate}
        \item The limit is unique.
        \item $ a_n\to a $ and $n_1<n_2<n_3<\cdots$ then $ a_{n_j}\to a $.
        \item If $ a_n\equiv n $ then $a_n\to c$.
        \item Convergent sequences are bounded.
        \item If $ a_n\to a,b_n\to b $ then $ a_n\pm b_n\to a\pm b $ and $ a_nb_n\to ab, a_n/b_n\to a/b $ for $b_n\neq 0$.
        \item If $ a_n\le A $ and $a_n\to a$, then $a\le A$.
    \end{enumerate}
\end{lemma}
\begin{lemma}[Archimedes]\label{lma:1.2}
    $ 1/n\to 0 $ as $ n\to \infty $.
\end{lemma}
\begin{proof}
    See Numbers and Sets.
\end{proof}

\subsection{Properties of the reals}
\begin{theorem}[Bolzano-Weierstrass]\label{thm:bolzano-weierstrass}
    If $ x_n\in \mathbb{R} $ and there exists $K$ such that
    \[
        \left| x_n \right| \le K,\quad \forall n,
    \]
    then we can find $ n_1<n_2<\cdots $ and $x\in \bbR$ such that 
    \[
        x_{n_j} \to x, \quad \text{as } j\to \infty.
    \]
    In other words, \textit{every bounded sequence has a convergent subsequence}.
\end{theorem}
\begin{remark}
    This says \textit{nothing} about uniqueness of the limit. e.g. $x_n=(-1)^n$.
\end{remark}
\begin{proof}
    Set $ [a_1,b_1] = [-K,K] $ and $ c=(a_1+b_1)/2 $. Consider the following alternatives:
    \begin{enumerate}
        \item $ x_n\in [a_1,c] $ for \textit{infinitely} many values of $n$.
        \item $ x_n\in [c,b_1] $ for \textit{infinitely} many values of $n$.
    \end{enumerate}
    Note that 1. and 2. could hold at the same time. If 1. holds, we set $a_2=a_1, b_2=c$. If 1. fails, then 2. must hold and we set $a_2=c,b_2=b_1$. Inductively, we construct sequences $ a_n,b_n $ such that $ x_m\in [a_n,b_n] $ for infinitely many values of $m$. By construction, 
    \begin{equation}\label{eq:bolzano-weierstrass-proof}\tag{$*$}
        a_{n-1}\le a_n\le b_n\le b_{n-1}\quad\text{and}\quad b_n-a_n = \frac{b_{n-1}-a_{n-1}}{2}.
    \end{equation}
    Now $a_n \nearrow$, $b_n \searrow$ and both are bounded, we know that 
    \[
        a_n\to a\in [a_1,b_1] \quad \text{and}\quad b_n \to b\in [a_1,b_1].
    \]
    By (\ref{eq:bolzano-weierstrass-proof}), $ b-a=\frac{b-a}{2} \Rightarrow b=a $.

    Since every $[a_n,b_n]$ has infinitely many $x_m$, having chosen $ x_{n_j}\in [a_j,b_j] $, $ \exists n_{j+1}>n_j $ such that $ x_{n_{j+1}}\in [a_{j+1},b_{j+1}] $. Hence 
    \[
        a_j\le x_{n_j}\le b_j \Rightarrow x_{n_j}\to a.
    \]
    This is a convergent subsequence, as required.
\end{proof}

\begin{definition}[Cauchy sequences]
    $a_n\in \bbR$ is called a \textbf{Cauchy sequence} if 
    \[
        \forall \epsilon>0, \exists N>0: \forall n,m\ge N, \left| a_n-a_m \right| <\epsilon.
    \]
\end{definition}

\begin{lemma}\label{lma:convergent -> cauchy}
    A convergent sequence is a Cauchy sequence.
\end{lemma}
\begin{proof}
    Given $ a_n\to a $, $ \forall \epsilon>0, \exists N: \forall n\ge N, |a_n-a|<\frac{\epsilon}{2} $. Take $m,n\ge N$ and by triangular inequality,
    \[
        |a_n-a_m|\le |a_n-a|+|a_m-a|<\epsilon.
    \]
\end{proof}

\begin{theorem}\label{thm:cauchy -> convergent}
    Every Cauchy sequence converges.
\end{theorem}
\begin{proof}
    Claim that Cauchy sequences are bounded. Let $a_n$ be a Cauchy sequence. Take $\epsilon=1$ and $N=N(1)$, then $ \forall n,m\ge N, |a_n-a_m|<1 $. Then
    \[
        |a_m|\le |a_m-a_N|+|a_N|<1+|a_N|, \quad \forall m\ge N.
    \]
    Take $ K= \max\{|a_1|,\dots,|a_{N-1}|, 1+|a_N|\} $, then $|a_n|\le K$.

    By the Bolzano-Weierstrass theorem, $ a_{n_j}\to a $. Claim that $a_n\to a$. Indeed, given $ \epsilon>0 $, $ \exists j_0 $ such that $ \forall j\ge j_0, |a_{n_j}-a|<\epsilon/2 $. Also $ \exists N(\epsilon) $ such that $ \forall m,n\ge N, |a_m-a_n|<\epsilon/2 $. Therefore, take $j$ such that $ n_j\ge \max \{N, n_{j_0}\} $ we have
    \[
        \left| a_n-a \right| \le \left| a_n-a_{n_j} \right| + \left| a_{n_j} -a\right| < \epsilon,
    \]
    as required.
\end{proof}
\begin{note}
    Thus on $ \mathbb{R} $, $ \text{convergent} \Leftrightarrow \text{Cauchy} $. This is known as the \textbf{general principle of convergence}.
\end{note}

\subsection{Series}
\subsubsection*{Series in general}

\begin{definition}[Series]
    Let $ a_n\in \mathbb{R} \text{ or } \mathbb{C} $. We say that 
    \[
        \sum_{j=1}^{\infty}a_j \text{ converges to } s \Longleftrightarrow s_N = \sum_{j=1}^{N} a_j \to s \text{ as }N\to \infty.
    \]
    We write 
    \[
        \sum_{j=1}^{\infty} a_j=s.
    \]
    If $s_N$ diverges, we say that $\sum_{j=1}^{\infty} a_j$ diverges.
\end{definition}
\begin{remark}
    Any problem in series can be turned into a problem in sequences by considering partial sums.
\end{remark}

\begin{lemma}\label{lma:1.6} 
    Let $ a_n,b_n\in \mathbb{C} $.
    \begin{enumerate}
        \item If $ \sum_{j=1}^{\infty}a_j,\sum_{j=1}^{\infty}b_j$ converge, then so does $ \sum_{j=1}^{\infty}\lambda a_j+\mu b_j$, where $ \lambda,\mu\in \mathbb{C} $.
        \item Suppose $ \exists N $ such that $ \forall j\ge N, a_j=b_j $, then either $ \sum_{j=1}^{\infty}a_j,\sum_{j=1}^{\infty}b_j$ both converge or they both diverge\footnote{Initial terms do not matter.}.
    \end{enumerate}
\end{lemma}
\begin{proof}
    For 1. note that 
    \begin{align*}
        s_N &= \sum_{j=1}^{N}\lambda a_j+\mu b_j = \lambda \sum_{j=1}^{N}a_j + \mu \sum_{j=1}^{N}b_j\\ 
        &= \lambda c_N + \mu d_N \to \lambda c+ \mu d.
    \end{align*}

    For 2. let $n\ge N$,
    \begin{align*}
        s_n &= \sum_{j=1}^{n} a_j = \sum_{j=1}^{N-1}a_j+\sum_{j=N}^{n}a_j,\\ 
        t_n &= \sum_{j=1}^{n} b_j = \sum_{j=1}^{N-1}b_j+\sum_{j=N}^{n}b_j.
    \end{align*}
    Hence 
    \[
        s_n-t_n = \sum_{j=1}^{N-1}a_j-\sum_{j=1}^{N-1}b_j=\text{constant}.
    \]
    Therefore they both converge or diverge.
\end{proof}

\begin{sprop}
    $ x^n\to 0 $ if $|x|<1$.
\end{sprop}
\begin{proof}
    Wlog, let $0<x<1$. Write $ 1/x=1+\delta $, where $ \delta>0 $. Then 
    \[
        x^n = \frac{1}{(1+\delta)^n}\le \frac{1}{1+n\delta}\to 0,
    \]
    where the last inequality comes from the fact that 
    \[
        (1+\delta)^{n} = \sum_{i=0}^{n} \binom{n}{i}\delta^i\ge 1+\binom{n}{1}\delta=1+n\delta,
    \]
    known as the \textbf{Bernoulli's inequality}.
\end{proof}

\begin{example}[Geometric series]
    Set $a_n=x^{n-1}$, then 
    \[
        s_n=\sum_{j=1}^{n}a_j=\sum_{j=0}^{n-1}x^j= \begin{cases}
        \frac{1-x^n}{1-x} &\text{if } x\neq 1\\
        n &\text{if }x=1\\
        \end{cases}
        \to \begin{cases}
            \frac{1}{1-x} &\text{if } |x|<1,\\
            \infty &\text{if } x\ge 1,\\
            \text{oscillates} &\text{if } x\le -1.\\
            \end{cases} 
    \]
    Thus a geometric series converges if and only if $|x|<1$.
\end{example}
\begin{note}
    $s_n\to \infty(-\infty): \forall A>0, \exists N: \forall n\ge N, s_n>(<)A(-A) $. If $s_n$ does not converge or $\to \pm \infty$, we say $s_n$ \textit{oscillates}.
\end{note}

\begin{lemma}\label{lma:1.7}
    If $ \sum_{n=1}^{\infty}a_n$ converges, then $ \lim_{n \to \infty} a_n=0$. 
\end{lemma}
\begin{proof}
    Note that $ a_n=s_{n}-s_{n-1}\to 0 $.
\end{proof}
\begin{remark}
    If $ \lim_{n \to \infty} a_n=0$, it is \textit{not} generally true that $ \sum_{n=1}^{\infty}a_n$ converges.
\end{remark}
\begin{sprop}[Harmonic series]
    $ \displaystyle \sum_{n=1}^{\infty}\frac{1}{n} $ diverges.
\end{sprop}
\begin{proof}
    Note that 
    \[
        s_{2n}=s_{n}+\frac{1}{n+1}+\cdots +\frac{1}{2n}>s_{n}+\sum_{k=n+1}^{2n}\frac{1}{2n}=s_n+\frac{1}{2},
    \]
    so it diverges.
\end{proof}

\subsubsection*{Series of non-negative terms}
Throughout this part, assume $a_n\ge 0$.

\begin{theorem}[Comparison test]\label{thm:Comparison test}
    Suppose $ 0\le b_n\le a_n, \forall n $. Then if $ \sum_{n=1}^{\infty} a_n $ converges, so does $\sum_{n=1}^{\infty} b_n$.
\end{theorem}
\begin{proof}
    Let $ s_N=\sum_{n=1}^{N}a_n, b_N=\sum_{n=1}^{N}b_n $, then $ d_N\le s_N $. But $s_N\to s$, then 
    \[
        d_N\le s_N\le s, \quad\forall N.
    \]
    Also $ d_N $ is increasing, so it converges.
\end{proof}

\begin{example}
    Consider $ \sum_{n=1}^{\infty}\frac{1}{n^2} $. Note that 
    \[
        \frac{1}{n^2}<\frac{1}{n(n-1)} = \frac{1}{n-1}-\frac{1}{n} = a_n
    \]
    for $n\ge 2$, and that
    \[
        \sum_{n=2}^{N} a_n = 1-\frac{1}{N}\to 1,
    \]
    so $ \sum_{n=1}^{\infty}a_n $ converges and thus $ \sum_{n=1}^{\infty}\frac{1}{n^2} $ converges\footnote{See \href{https://www.cnblogs.com/misaka01034/p/BaselProof.html}{this article} and \href{https://www.math.cmu.edu/~bwsulliv/basel-problem.pdf}{this article}.}.
\end{example}

\begin{theorem}[Root test/Cauchy's test]\label{thm:Root test}
    Assume $a_n\ge 0$ and $ (a_n)^{1/n}\to a $, then if $a<1$, then $ \sum_{n=1}^{\infty}a_n$ converges. If $ a>1 $, then $\sum_{n=1}^{\infty}a_n$ diverges.
\end{theorem}
\begin{remark}
    If $a=1$, then the root test does not work. e.g. $ n^{1/n}\to 1 $ but $\sum n$ certainly diverges. To see this limit, write
    \[
        n^{1/n} = 1+\delta_n, \quad \delta_n>0.
    \]
    Then $ n = (1+\delta_n)^n\ge \frac{n(n-1)}{2}\delta_n^2 $, and thus $ \delta_n^2<\frac{2}{n-1}\to 0 $.
\end{remark}
\begin{proof}
    If $ a<1 $, choose $r$ such that $a<r<1$. Then there exists $N$ that 
    \[
        \forall n\ge N, (a_n)^{1/n}<r \Longrightarrow a_n<r^n.
    \]
    Since $ \sum r^n $ converges, $\sum a_n$ converges by comparison test. If $a>1$ then $a_n>1\nrightarrow 0$, so $\sum a_n$ diverges.
\end{proof}

\begin{theorem}[Ratio test/D'Alembert's test]\label{thm:Ratio test}
    Suppose $a_n>0$ and $ \frac{a_{n+1}}{a_n}\to \ell $. If $ \ell <1 $, then $ \sum a_n $ converges and if $ \ell >1 $, $ \sum a_n $ diverges.
\end{theorem}
\begin{note}
    If $\ell=1$, the ratio test does not work.
\end{note}
\begin{proof}
    Suppose $\ell<1$ and choose $\ell<r<1$. Then
    \[
        \exists N, \forall n\ge N: \frac{a_{n+1}}{a_n}<r \Longrightarrow a_n = \frac{a_n}{a_{n-1}}\frac{a_{n-1}}{a_{n-2}}\cdots \frac{a_{N+1}}{a_{N}}a_N<a_N r^{n-N} .
    \]
    By comparison test, $ \sum a_n $ converges. If $ \ell >1 $, choose $\ell>r>1$. Then $ \exists N, \forall n\ge N, \frac{a_{n+1}}{a_n}>r $, so 
    \[
        a_n = \frac{a_n}{a_{n-1}}\frac{a_{n-1}}{a_{n-2}}\cdots \frac{a_{N+1}}{a_{N}}a_N>a_N r^{n-N}\to \infty.
    \]
    Hence $ \sum a_n $ diverges.
\end{proof}

\begin{theorem}[Cauchy's condensation test]\label{thm:Cauchy's condensation test}
    Let $ a_n \searrow, a_n>0 $, then $ \sum_{n=1}^{\infty}a_n $ converges if and only if 
    \[
        \sum_{n=1}^{\infty}2^n a_{2^n}
    \]
    converges.
\end{theorem}
\begin{proof}
    Observe that 
    \begin{equation}\label{eq:condensation_test}\tag{$ * $}
        a_{2^k}\le a_{2^{k-1}+i}\le a_{2^{k-1}},\quad \forall k\ge 1, \forall 1\le i\le 2^{k-1}.
    \end{equation}

    Assume that $ \sum a_n $ converges to $A$, then by (\ref{eq:condensation_test}),
    \[
        2^{n-1}a_{2^n}\le \sum_{m=2^{n-1}+1}^{2^n}a_m \Longrightarrow \sum_{n=1}^{N}2^{n-1}a_{2^n}\le \sum_{n=1}^{N}\sum_{m=2^{n-1}+1}^{2^n}a_m = \sum_{m=2}^{2^N}a_m.
    \]
    Therefore we have 
    \[
        \sum_{n=1}^{N}2^n a_{2^n} \le 2 \sum_{m=2}^{2^N}a_m\le 2(A-a_1).
    \]
    Thus it converges.

    Conversely assume that $ \sum_{n=1}^{\infty}2^na_{2^n} $ converges. Then 
    \[
        \sum_{m=2^{n-1}+1}^{2^n}a_m\le \sum_{m=2^{n-1}+1}^{2^n}a_{2^{n-1}} = 2^{n-1}a_{2^{n-1}}.
    \]
    Therefore 
    \[
        \sum_{m=2}^{N}a_m = \sum_{n=1}^{N}\sum_{m=2^{n-1}+1}^{2^n} a_m \le \sum_{n=1}^{N}2^{n-1}a_{2^{n-1}}\le B.
    \]
    Hence $ \sum_{m=2}^{N}a_m $ is bounded and increasing, thus converges.
\end{proof}

\begin{theorem}[Alternating series test]\label{thm:Alternating series test}
    If $a_n\searrow$ and $a_n \to 0 $, then 
    \[
        \sum_{n=1}^{\infty}(-1)^{n+1}a_n
    \]
    converges.
\end{theorem}
\begin{proof}
    Define $ s_n = a_1-a_2+\cdots+(-1)^{n+1}a_n $. Note that 
    \[
        s_{2n} = (a_1-a_2)+\cdots+(a_{2n-1}-a_{2n})\ge s_{2n-2},
    \]
    and 
    \[
        s_{2n} = a_1 - (a_2-a_3)-\cdots-(a_{2n-2}-a_{2n-1}) - a_{2n},
    \]
    so $ s_{2n} \nearrow $ and bounded above, so $ s_{2n}\to s $. Also
    \[
        s_{2n+1}=s_{2n}+a_{2n+1}\to s.
    \]
    Therefore $ s_n\to 0 $.
\end{proof}

\begin{example}
    Consider the following series: 
    \[ 
        \text{(1). } \sum_{n=1}^{\infty} = \frac{n}{2^n};\quad \text{(2). } \sum_{n=1}^{\infty}\left( \frac{n+1}{3n+5} \right)^{n}; \quad \text{(3). } \sum_{n=1}^{\infty}\frac{1}{n^k}\ (k\in \mathbb{R}).
    \]

    \begin{enumerate}[(1).]
        \item Note that $ \frac{n+1}{2^{n+1}}\frac{2^n}{n}=\frac{n+1}{2n}\to \frac{1}{2}<1 $, so it converges by ratio test.
        \item $ \frac{n+1}{3n+5}\to \frac{1}{3}<1 $, so it converges by root test.
        \item Claim that it converges if and only if $k>1$. The case when $ k\le 0 $ is obvious, so we only need to consider $k>0$. Note that 
        \[
            2^na_{2^n} = 2^{n-nk}=(2^{1-k})^n
        \]
        Therefore by condensation test, it converges if and only if $ 1-k<0 \Leftrightarrow k>1 $. We can also deduce it from 
        \[
            \frac{1}{(n+1)^k}<\frac{1}{n^k} \Longrightarrow \left( \frac{n}{n+1} \right)^k<1.
        \]
    \end{enumerate}
\end{example}

\subsubsection*{Absolute convergence}
\begin{definition}[Absolute convergence]
    Take $ a_n\in \mathbb{C}  $. If $ \sum_{n=1}^{\infty}|a_n| $ is convergent, then the series is \textbf{absolutely convergent}
\end{definition}
\begin{note}
    Since $ |a_n|\ge 0 $, we can use previous tests to check absolute convergence. This is particularly useful for \textit{complex} numbers.
\end{note}

\begin{theorem}\label{thm:absolute convergence -> convergence}
    If $ \sum a_n $ is \textit{absolutely convergent}, it is \textit{convergent}.
\end{theorem}
\begin{proof}
    Suppose first that $ a_n\in \mathbb{R} $. Let 
    \[
        v_n = \begin{cases}
        a_n &\text{if }a_n\ge 0\\
        0 &\text{if }a_n<0\\
        \end{cases} \quad \text{and}\quad w_n = \begin{cases}
            0 &\text{if }a_n\ge 0\\
            -a_n &\text{if }a_n<0\\
        \end{cases},
    \]
    so that 
    \[
        v_n = \frac{|a_n|+a_n}{2},\quad w_n = \frac{|a_n|-a_n}{2}.
    \]
    Clearly, $ v_n,w_n\ge 0 $, $ a_n = v_n-w_n $, and $ |a_n|=v_n+w_n\ge v_n,w_n $.

    If $ \sum |a_n| $ converges, by comparison test, $ \sum v_n,\sum w_n $ converge, and thus $ \sum a_n $ converges.

    If $ a_n\in \mathbb{C}  $, write $ a_n = x_n+ i y_n $. Note that 
    \[
        |x_n|,|y_n|\le |a_n|,
    \]
    so $ \sum |x_n|,\sum |y_n| $ converge and thus $ \sum x_n,\sum y_n $ converge. Therefore $ \sum a_n $ converges.
\end{proof}

\begin{example}
    \begin{enumerate}
        \item $ \sum (-1)^n/n $ is convergent but not absolutely convergent.
        \item $ \sum z^n/2^n, z\in \mathbb{C} $. Note that $ \sum |z^n/2^n| = \sum (|z|/2)^n $, so if $ |z|<2 $, it is absolutely convergent, thus convergent. If $ |z|\ge 2 $, then $ |z^n/2^n|\ge 1 $, so $ a_n \nrightarrow 0 $ and thus $\sum z^n/2^n$ diverges.
    \end{enumerate}
\end{example}

\begin{definition}[Conditional convergence]
    We say that $ \sum a_n $ is \textbf{conditionally convergent} if $ \sum a_n $ converges but $ \sum |a_n| $ does not.
\end{definition}
\begin{note}
``Conditional'' because the sum to which the series converges is \textit{conditional} on the \textit{order} in which the elements of the sequence are taking. If rearranged, the sum could be altered.
\end{note}

\begin{definition}[Rearrangement]
    Let $ \sigma $ be a bijection of $ \mathbb{N} $, then 
    \[
        a_n'=a_{\sigma(n)}
    \]
    is called a \textbf{rearrangement}.
\end{definition}
\begin{theorem}\label{thm:absolute convergence -> rearr convergence}
    If $ \sum a_n, a_n\in \mathbb{C} $ is absolutely convergent, then for any rearrangement $ \sigma $,
    \[
        \sum_{n=1}^{\infty}a_{\sigma(n)} = \sum_{n=1}^{\infty} a_n.
    \]
\end{theorem}
\begin{proof}
    First assume $ a_n\in \mathbb{R} $. Let $ \sum a_n' $ be a rearrangement of $ \sum a_n $, and let $ s_n,t_n $ be the partial sums of $ \sum a_n,\sum a_n' $ respectively. 
    
    Suppose first that $a_n\ge 0$. Given $n$, there exists $s_q$ such that every term of $t_n$ is in $s_q$. Thus $ t_n\le s_q\le s $, where $ s = \sum a_n $. Note that $ t_n \nearrow $, so $ t_n\to t $ and $ t\le s $. Now we can reverse the process to find $t_p$ such that every term in $s_n$ is in $t_p$, so by the same arguments, $ s\le t $ and thus $ s=t $.
    
    If $ a_n $ has any sign, consider $ v_n,w_n $ from the proof of theorem \ref{thm:absolute convergence -> convergence}. Consider $ \sum a_n'$, $\sum v_n'$, $\sum w_n' $. Since $ \sum |a_n| $ converges, $ \sum v_n,\sum w_n $ converge. Since $v_n,w_m\ge 0$,
    \[
        \sum_{n=1}^{\infty} v_n' = \sum_{n=1}^{\infty} v_n,\quad\sum_{n=1}^{\infty} w_n'=\sum_{n=1}^{\infty} w_n
    \]
    by previous arguments. Therefore $ \sum a_n' = \sum a_n $. For the case $ a_n\in \mathbb{C}  $, write $ a_n = x_n+iy_n $ and note that $ |x_n|,|y_n|\le |a_n| $, so that $ \sum x_n,\sum y_n $ are absolutely convergent. Hence $ \sum a_n' = \sum x_n'+iy_n'=\sum a_n $.
\end{proof}