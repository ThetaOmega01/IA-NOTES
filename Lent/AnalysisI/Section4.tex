\section{Power Series}
We look at series in the form 
\[
    \sum_{n=0}^{\infty} a_nz^n,\quad z\in \mathbb{C}, a_n\in \mathbb{C}.
\]
Or, for $ z_0 $ fixed, we look at 
\[
    \sum_{n=0}^{\infty} a_n(z-z_0)^n.
\]
\subsection{Radius of convergence}
\begin{lemma}
    If $ \sum_{n=0}^{\infty} a_n z_1^n$ converges, and $ |z|<|z_1| $, then $ \sum_{n=0}^{\infty} a_n z^n $ converges absolutely.
\end{lemma}
\begin{proof}
    Since $ \sum_{n=0}^{\infty} a_n z_1^n$ converges, $a_nz_1^n\to 0$. Thus $ \{a_nz_1^n\} $ is bounded by $K>0$. Now 
    \[
        \left| a_nz^n \right| \le K \left| \frac{z}{z_1} \right| ^n.
    \]
    Since the sum of RHS converges, the lemma follows by comparison.
\end{proof}
Using this lemma we will show that every power series has a radius of convergence.
\begin{theorem}\label{thm:4.2}
    A power series either
    \begin{enumerate}
        \item converges absolutely for all $z$, or
        \item converges absolutely for all $z$ inside a circle $ |z|=R, R> 0 $, and diverges for all $ |z|>R $, or 
        \item converges for $R=0$ only.
    \end{enumerate}
\end{theorem}
\begin{definition}[Radius of convergence]
    The circle $ |z|=R $ is called the \textbf{circle of convergence} and $R$ is the \textbf{radius of convergence}.
\end{definition}
\begin{proof}
    Let $ S = \{x\in \mathbb{R}: x\ge 0 \land \sum a_n x^n \text{ converges}\} $. Clearly $0\in S$. If $ x_1\in S, [0,1]\in S $. If $ S=[0,\infty) $ then we have case 1. If not, $ \exists R = \sup S \ge 0 $ finite.

    If $ R>0 $, We will prove that if $ |z_1|<R $ then $ \sum a_nz_1^n $ converges absolutely. Choose $ R_0 $ such that $ |z_1|<R_0<R $. Then $ R_0\in S $ by definition of sup, and the series converges for $ z=R_0 $. By lemma, $ \sum |a_n z^n| $ converges.

    Finally we show that if $ |z_2|>R\ge 0 $, then the series does not converge. Take $ R_0 $ such that $ R<R_0<|z_2| $. If $ \sum a_nz_2^n $ converges, then $ \sum a_n R_0^n $ converges, contradicting with the choice of $R_0$.
\end{proof}

The following lemma is useful for finding $R$: 
\begin{lemma}\label{lma:4.3}
    If $ |a_{n+1}/a_n|\to \ell $, then $ R = 1/\ell $.
\end{lemma}
\begin{proof}
    By ratio test, we have absolute convergence if 
    \[
        \lim_{n \to \infty} \left| \frac{a_{n+1}}{a_n}\frac{z^{n+1}}{z^n} \right| <1, 
    \]
    so if $ |z|<\frac{1}{\ell} $, we have absolute convergence. If $ |z|>\frac{1}{\ell} $, then the series diverges.
\end{proof}
\begin{remark}
    One can also use root test to get that $ |a_n|^{1/n}\to \ell $, then $ R = 1/\ell $.
\end{remark}
\begin{example}
    \begin{enumerate}
        \item $\displaystyle \sum_{n=0}^{\infty}\frac{z^n}{n!}$. Note that 
        \[
            \left| \frac{a_{n+1}}{a_n} \right|  = \frac{1}{n+1}\to 0=\ell,
        \]
        so $ R=\infty $.
        \item $\displaystyle \sum_{n=0}^{\infty}z^n$. Immediately $ R=1 $. Note that on the circle we have divergence.
        \item $\displaystyle \sum_{n=0}^{\infty}n!z^n$. Clearly $ R=0 $.
        \item $\displaystyle \sum_{n=1}^{\infty}\frac{z^n}{n}$. $ R=1 $. But it diverges for $z=1$. If $ |z|=1,z\neq 1 $, it converges by Abel's test (see example sheet 1). Alternatively, consider 
        \[
            \sum_{n=1}^{\infty}\frac{z^n}{n}(1-z) \Longrightarrow \begin{aligned}
                s_N &= \sum_{n=1}^{N}\left( \frac{z^n-z^{n+1}}{n} \right) = \sum_{n=1}^{N}\frac{z^n}{n}-\sum_{n=2}^{N+1}\frac{z^{n}}{n-1}\\ 
                &= z-\frac{z^{N+1}}{N}+ \sum_{n=2}^{N+1}\frac{-z^n}{n(n-1)}.
            \end{aligned}
        \]
        If $ |z|=1 $, then $ \frac{z^{N+1}}{N}\to 0 $ and $ \sum_{n=2}^{\infty}\frac{1}{n(n-1)} $, so $s_N$ converges.
        \item $\displaystyle \sum_{n=1}^{\infty} \frac{z^n}{n^2}$. $ R=1 $ and it converges for all $z$ with $|z|=1$.
        \item $\displaystyle \sum_{n=0}^{\infty} nz^n$. $ R=1 $ but it diverges for all $ |z|=1 $.  
    \end{enumerate}
\end{example}
\begin{remark}
    \textit{No} claim is made \textit{on} the circle of convergence. Within the radius of convergence, power series will behave as if they were polynomials.
\end{remark}

\subsection{Differentiation}
\begin{theorem}\label{thm:4.4}
    Suppose the series
    \[
        f(z) = \sum_{n=0}^{\infty}a_n z^n
    \]
    has radius of convergence $R$. Then $f$ is differentiable at all points $ |z|<R $ with derivative 
    \[
        f'(z) = \sum_{n=1}^\infty na_n z^{n-1}.
    \]
\end{theorem}
\begin{proof}
    We need 2 lemmas:
    \begin{lemma}\label{lma:4.5}
        If $ \sum_0^\infty a_nz^n $ has radius of convergence $R$, then so do $ \sum_{1}^{\infty}na_n z^{n-1} $ and $ \sum_{2}^\infty n(n-1)a_n z^{n-2} $.
    \end{lemma}
    \begin{lemma}\label{lma:4.6}
        \begin{enumerate}
            \item For all $2\le r\le n$, $\displaystyle \binom{n}{r}\le n(n-1) \binom{n-2}{r-2}$.
            \item $ \left| (z+h)^n-z^n-nhz^{n-1} \right| \le n(n-1)(|z|+|h|)^{n-2}|h|^2,\ \forall z,h\in \bbC $. 
        \end{enumerate}
    \end{lemma}

    By lemma 4.5 we may define 
    \[
        f'(z) := \sum_{n=1}^{\infty} na_n z^{n-1},\quad |z|<R.
    \]
    We are required to prove that 
    \[
        \lim_{h \to 0} I = \lim_{h \to 0} \frac{f(z+h)-f(z)-hf'(z)}{h}=0.
    \]
    Note that (by lemma 4.6),
    \begin{IEEEeqnarray*}{rCl}
        I &=& \frac{1}{h}\sum_{n=0}^{\infty} a_n((z+h)^n-z^n-hnz^{n-1})\\ 
        \Longrightarrow  |I| &=& \frac{1}{|h|}\left| \lim_{N \to \infty} \sum_{n=0}^{N} a_n((z+h)^n-z^n-hnz^{n-1})\right|\\ 
        &=& \frac{1}{|h|}\lim_{N \to \infty}\left|  \sum_{n=0}^{N} a_n((z+h)^n-z^n-hnz^{n-1})\right|\\ 
        &\le& \frac{1}{|h|}\sum_{n=0}^{\infty} |a_n||(z+h)^n-z^n-hnz^{n-1}|\\ 
        &\le& \frac{1}{|h|}\sum_{n=2}^{\infty}|a_n|\cdot n(n-1)(|z|+|h|)^{n-2}|h|^2\\ 
        &=& |h|\sum_{n=2}^{\infty} |a_n| \cdot n(n-1)(|z|+|h|)^{n-2}
    \end{IEEEeqnarray*}
    By lemma 4.5, for $|h|$ small enough, 
    \[
        \sum_{n=2}^{\infty} |a_n| \cdot n(n-1)(|z|+|h|)^{n-2}
    \]
    converges to $A(h)$, but $A(h)\le A(r)$ for $r$ such that $|h|<r$ and $ |z|+r<R $. Hence 
    \[
        0\le |I|\le |h|A(h) \le |h|A(r)\to 0 \quad \text{as}\quad h\to 0,
    \]
    as required.
\end{proof}
\begin{proof}[Proof of lemma 4.5]
    The lemma holds trivially for $z=0$.
    Take $z,R_0$ such that $ 0<|z|<R_0<R $. Since $ a_nR_0^n\to 0 $, $ \exists K $ such that $ |a_n R_0^n|\le K, \forall n\ge 0 $. Thus 
    \[
        \left| a_n n z^{n-1} \right| = \frac{n}{|z|} \left| a_n R_0^n \right| \left| \frac{z}{R_0} \right|^n \le \frac{Kn}{|z|}\left| \frac{z}{R_0} \right|^n.
    \]
    But the series $\sum \frac{n}{|z|}| \frac{z}{R_0} |^n$ converges by the ratio test (check it), so $ \sum a_n n z^{n-1} $ converges absolutely by comparison, thus converges.

    If $ |z|>R $, then $ |a_nz^n| $ is \textit{unbounded} (check \href{https://math.stackexchange.com/questions/799481/absolute-sequence-unbounded-within-radius-of-convergence}{this}), and so is $ n|a_nz^n| $, so $ \sum na_n z^{n-1} $ diverges. The same proof applies to $\sum_{2}^\infty n(n-1)a_n z^{n-2}$.
\end{proof}
\begin{proof}[Proof of lemma 4.6]
    Part 1 follows by direct check. For part 2, note that
    \[
        (z+h)^n-z^n-nhz^{n-1} = \sum_{r=2}^{n}\binom{n}{r} z^{n-r}h^r.
    \] 
    Thus 
    \begin{align*}
        \left| (z+h)^n-z^n-nhz^{n-1} \right| &\le \sum_{r=2}^{n}\binom{n}{r} |z|^{n-r}|h|^r \\ 
        &\le |h|^2 n(n-1) \sum_{r=2}^{n}\binom{n-2}{r-2}|z|^{n-r}|h|^{r-2}\\
        &= n(n-1)(|z|+|h|)^{n-2}|h|^2. \qedhere
    \end{align*}
\end{proof}

\subsection{The standard functions}
We have already seen that 
\[
    \sum_{n=0}^{\infty} \frac{z^n}{n!}
\]
has $ R=\infty $.
\begin{definition}
    The \textbf{exponential function} is defined by $ e:\bbC\to\bbC $, where
    \[
        e(z) = \sum_{n=0}^{\infty} \frac{z^n}{n!}.
    \]
\end{definition}
Straight from theorem \ref{thm:4.4}, $ e $ is differentiable and $ e'(z)=e(z) $.

\begin{sprop}
    If $ F:\bbC\to\bbC $ has $ F'(z)=0, \forall z\in \bbC $, then $F$ is constant.
\end{sprop}
\begin{proof}
    Consider $g(t)=F(tz)=u(t)+iv(t)$. By chain rule,
    \[
        g'(t)=F'(tz)z=0=\footnote{Check}u'(t)+iv'(t) \Longrightarrow u'=v'=0.
    \]
    Hence $u,v$ are constant by corollary \ref{col:3.5}, thus $F$ is constant.
\end{proof}

Now let $a,b\in \bbC$ and consider 
\[
    F(z) = e(a+b-z)e(z) \Longrightarrow F'(z) = 0.
\]
So $F$ is constant and $e(a+b-z)e(z)=e(a+b).$ Set $z=b$ gives 
\[
    \boxed{e(a)e(b)=e(a+b)} 
\]