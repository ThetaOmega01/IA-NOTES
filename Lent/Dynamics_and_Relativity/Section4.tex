\section{Orbits}
\subsection{Angular momentum}
\begin{definition}
    The \textbf{angular momentum} for a particle of mass $m$ moving under influence of a force $ \bfF $, with position vector $\bfr(t)$ and velocity $ \dot{\bfr}(t) $, is defined to be 
    \[
        \bfL = \bfr \times \bfp = \bfr \times m \dot{\bfr}.
    \]
    Note that $ \dot{\bfL} = m \dot{\bfr}\times \dot{\bfr}+m\bfr \times \ddot{\bfr} = \bfr \times \bfF $. Define the \textbf{torque}(or \textbf{moment} of $\bfF$) to be $ \bfG = \bfr \times \bfF = \dot{\bfL} $.
\end{definition}
\begin{note}
    Values of $ \bfL,\bfG $ depend on choice of origin. We can say `about the origin' or `about any specified point'.
\end{note}
\begin{note}
    If $ \bfr \times \bfF =\bfG= \mathbf{0} $, then the angular momentum is constant. We say that the angular momentum is \textbf{conserved}. For the same system, we can choose the origin wisely so that the angular momentum is conserved.
\end{note}

The basic problem: consider a particle moving in a force that depends \textit{only} on the radius.
\[
    m \ddot{\bfr} = - \nabla V(r).
\]
We see that force is directed towards (or away) from the origin. Assume that the central mass (relative to orbiting mass) is remaining fixed. This is well approximated if the central mass is much larger than the orbiting mass $m$.

\subsection{Central forces}
\begin{definition}
    \textbf{Central forces} are a special class of conservative forces with $ V(\bfr) = V(r) $. 
\end{definition}
Note that $ \bfF(\bfr) = - \nabla V(r) = -\frac{\mathrm{d}V}{\mathrm{d}r}\hat{\bfr}  $, where $ |\hat{\bfr}|=1 $ is a unit vector directed away from the origin.
\begin{center}
    \begin{tikzpicture}
        \draw [->-=0.5] (0,0) -- (2,0) node [pos=0, circ] {} node [pos=0.5,below] {$ \bfr $} node [pos=1, below] {$m$} node [pos=1, dot=4pt] {};
        \draw [->>, blue] (2,0) -- (3.2,0) node [right] {$ \bfF $}; 
    \end{tikzpicture}
\end{center}

\begin{proposition}
    Angular momentum about $O$ is conserved for a central force.
\end{proposition}
\begin{proof}
    Consider the angular momentum $\bfL$ about $O$: 
    \[
        \dot{\bfL} = \bfr \times \bfF = \bfr \times \left( -\frac{\mathrm{d}V}{\mathrm{d}r} \hat{\bfr} \right) = \mathbf{0}.\qedhere
    \]
\end{proof}
\begin{proposition}
    The motion of a particle under a central force is in a plane through the origin, with orientation determined by $\bfL$.
\end{proposition}
\begin{proof}
    We have $ \bfL = \text{constant} $ and $ \bfL \cdot \bfr=0 $, which is the equation of a plane through $O$.
\end{proof}
Hence we reduce a 3D problem to a 2D problem.

\subsection{Polar coordinates}
Choose $z$ axis so that the orbit lies in $z=0$ plane. To describe the orbit, we introduce polar coordinates $(r, \theta)$:
\[
  x = r\cos\theta, \quad y = r\sin \theta.
\]
Our object is to separate the motion of the particle into radial and angular components. We do so by definitionning unit vectors in the directions of increasing $r$ and increasing $\theta$:
\begin{center}
  \begin{tikzpicture}
    \node at (-4.5,1.5) {$\bfe_r = \begin{pmatrix}\cos \theta\\ \sin \theta\end{pmatrix}$,};
    \node at (-2,1.5) {$\bfe_\theta = \begin{pmatrix}-\sin \theta\\\cos\theta \end{pmatrix}$.};
    \draw [->] (0, 0) -- (4, 0) node [right] {$x$};
    \draw [->] (0, 0) -- (0, 3) node [above] {$y$};
    \draw (0, 0) -- (2, 1.5) node [dot=3pt]{} node [pos = 0.5, anchor = south east] {$r$};
    \draw [->] (2, 1.5) -- (2.5, 1.875) node [anchor = south west] {$\bfe_r$};
    \draw [->] (2, 1.5) -- (1.625, 2) node [anchor = south east] {$\bfe_\theta$};
    \draw (0.7, 0) arc (0:36.87:0.7);
    \node at (0.9, 0.3) {$\theta$};
  \end{tikzpicture}
\end{center}
$ \bfe_r,\bfr_\theta $ form an orthonormal basis at any point, but orientation depends on position. Note that 
\[
    \frac{\mathrm{d}}{\mathrm{d}\theta}\bfe_r = \bfe_\theta ,\quad \frac{\mathrm{d}}{\mathrm{d}\theta}\bfe_\theta=-\bfe_r. 
\]
For a moving particle $ \theta $ is a function of position and hence of time. If coordinates are $ (r(t),\theta(t)) $ then
\[
    \dot{\bfe}_r = \dot{\theta}\ \frac{\mathrm{d}\bfe_r}{\mathrm{d}\theta} = \dot{\theta}\bfe_\theta .
\]
Similarly
\[
    \dot{\bfe}_\theta = - \dot{\theta}\bfe_r.
\]

\subsubsection*{Motion in terms of polars}
We have
\[
    \bfr = r\bfe_r,\quad \dot{\bfr} = \dot{r} \bfe_r + r \dot{\bfe}_r = \dot{r}\bfe_r+r\dot{\theta}\bfe_\theta.
\]
Hence we can decompose $ \dot{\bfr} $ into \textbf{radial velocity} $ \dot{r} $ and \textbf{angular velocity} $ r\dot{\theta} $. Note that $ [\dot{\theta}] = T^{-1} $. The acceleration is given by
\begin{align*}
    \ddot{\bfr} &= \frac{\mathrm{d}}{\mathrm{d}t}\left( \dot{r}\bfe_r+r\dot{\theta}\bfe_\theta  \right)\\
    &=  \ddot{r}\bfe_r+\dot{r}\dot{\bfe}_r+\dot{r}\dot{\theta}\bfe_\theta+r \ddot{\theta}\bfe_\theta+r \dot{\theta}\dot{\bfe}_\theta\\ 
    &= \left( \ddot{r}-r\dot{\theta}^2 \right)\bfe_r+\left( 2 \dot{r}\dot{\theta}+r \ddot{\theta} \right)\bfe_\theta.
\end{align*}
\begin{example}[Circular motion with constant angular velocity]
    Here $ r=a $ is constant, and $ \dot{\theta} = \omega $ is constant. It follows that $ \dot{r}=\ddot{r}=\ddot{\theta}=0 $, so 
    \[
        \dot{\bfr} = a \omega \bfe_\theta,\quad \ddot{\bfr} = -a\omega^2\bfe_r.
    \]
    $ \ddot{\bfr} $ is also called the \textbf{centripetal acceleration}. Newton's 2nd law implies that a \textbf{centripetal force} is required to maintain this circular motion.

    \begin{center}
        \begin{tikzpicture}
            \node [left] at (0,0) {$O$};
            \draw [dashed] (0,0) -- (1.732,1);
            \node [circ] at (1.732,1) {};
            \node at (2.132,1.2) {$m$};
            \draw [dashed] (0,0) -- (2,0);
            \draw (0.5,0) arc (0:30:0.5) node [pos=0.7, right] {$\theta$}; 
            \draw [dashed] (0,0) circle (2);
            \draw [->] (0,-0.7) -- (0,0);
            \draw [->] (0,-1.3) -- (0,-2);
            \node at (0,-1) {$a$};
            \centerarc[blue,->](0,0)(20:40:2.8);
            \node [above right,blue] at (2.42,1.4) {$ a\omega $};
            \draw [->>,blue] (2.26,0.82) -- (1.5,0.55) node [pos=0.4, below] {$ -a\omega^2\bfe_r$};
        \end{tikzpicture}
    \end{center}
\end{example}

\subsection{Motion in a central force field}
\subsubsection*{Equation of motion}
By Newton's 2nd law,
\[
    m \ddot{\bfr} = \bfF = - \nabla V = -\frac{\mathrm{d}V}{\mathrm{d}r}\bfe_r. 
\]
By previous results,
\[
    m\left( \ddot{r}-r\dot{\theta}^2 \right)\bfe_r+m\left( 2 \dot{r}\dot{\theta}+r \ddot{\theta} \right)\bfe_\theta=-\frac{\mathrm{d}V}{\mathrm{d}r}\bfe_r. 
\]
Consider angular component: 
\[
    m\left( 2 \dot{r}\dot{\theta}+r \ddot{\theta} \right) =0
    \Longrightarrow\frac{m}{r}\frac{\mathrm{d}}{\mathrm{d}t}\left( r^2\dot{\theta} \right)=0,
\]
which is equivalent to 
\[
    m r^2 \dot{\theta} = \text{const}.
\]
Consider the angular momentum 
\[
    \bfL = m\bfr \times \dot{\bfr} = mr\bfe_r \times (\dot{r}\bfe_r+r \dot{\theta}\bfe_\theta) =mr^2 \dot{\theta}\bfe_z.
\]
i.e. magnitude of angular momentum is constant: $ r^2 \dot{\theta}=h $ constant.

Consider radial component: 
\begin{equation}\label{eq:radial component}\tag{$*$}
    m\left( \ddot{r}-r\dot{\theta}^2 \right) = -\frac{\mathrm{d}V}{\mathrm{d}r} \Longrightarrow m \ddot{r}=-\frac{\mathrm{d}V}{\mathrm{d}r}+\frac{mh^2}{r^3}=: -\frac{\mathrm{d}V_{\text{eff}}}{\mathrm{d}r},  
\end{equation}
where $ V_{\text{eff}}(r)=V(r)+\frac{mh^2}{2r^2} $ is called the \textbf{effective potential}. This means that motion of particle is equivalent to 1D motion under $ V_{\text{eff}} $.
\subsubsection*{Equation of energy}
Consider energy of the particle:
\[
    T+V(r) = \frac{1}{2}m \left( \dot{r}^2+r^2 \dot{\theta}^2 \right) +V(r) = \frac{1}{2}m \dot{r}^2+\underbrace{\frac{1}{2}m\frac{h^2}{r^2}+V(r)}_{V_{\text{eff}}(r)}.
\]
This indeed matches with the definitionnition before.

\begin{example}[Inverse square law force]
    Consider 
    \[
        V(r) = -\frac{GMm}{r},\quad V_{\text{eff}}(r) = -\frac{GMm}{r}+\frac{mh^2}{2r^2},
    \]
    given $h$.
    \begin{center}
        \begin{tikzpicture}[xscale=0.5]
          \draw [->] (0, 0) -- (8, 0) node [right] {$r$};
          \draw [->] (0, -2) -- (0, 2) node [above] {$V_{\text{eff}}$};
          \draw [red,samples=70, domain=0.5:7.8] plot (\x, {-3/\x + 2/(\x*\x)});
    
          \draw [dashed] (1.33, -1.125) -- (0, -1.125) node [left] {$E_{\min}$};
          \draw [dashed] (1.33, -1.125) -- (1.33, 0) node [above] {$r_*$};
        \end{tikzpicture}
    \end{center}
    Let $ V_{\text{eff}}(r_0)=0, V_{\text{eff}}'(r_{*})=0 $. Then
    \[
        r_0=\frac{h^2}{2gm},r_*=\frac{h^2}{gm}, E_{\text{min}}=-\frac{m(GM)^2}{2h^2} .
    \]
    \begin{itemize}
        \item $ E=E_{\text{min}} $. Then $ r(t)\equiv r_* $. Note that $ \dot{\theta} = h/r_*^2 $ so the particle reaches an equilibrium and does circular motion with constant angular velocity.
        \item $E_{\min} < E < 0$. Then $r$ oscillates and $\dot{\theta}=h/r^2$ does also. This is a non-circular, bounded orbit.
        \item $E \geq 0$. Then $ r(t)\to \infty $ as $ t\to \infty  $. The particle escapes and the orbit is unbounded.
    \end{itemize}
\end{example}


\begin{definition}[Periapsis, apoapsis and apsides]
    The points of minimum and maximum $r$ in such an orbit are called the \textbf{periapsis} and \textbf{apoapsis}. They are collectively known as the \textbf{apsides}.
\end{definition}

\begin{definition}[Perihelion and aphelion]
For an orbit around the Sun, the periapsis and apoapsis are known as the \textbf{perihelion} and \textbf{aphelion}.
\end{definition}

In particular
\begin{definition}[Perigee and apogee]
The perihelion and aphelion of the Earth are known as the \textbf{perigee} and \textbf{apogee}.
\end{definition}

\subsection{Stability of circular orbits}
Consider a genreral potential $V(r)$. Does a circular orbit exists and is it stable? Assume that angular momentum is given and non-zero.

For a circular orbit, $r(t)=r_*=\text{const}$ requires $ \ddot{r}=0 $ and hence $ {\color{blue}V_{\text{eff}}'(r_*)=0} $. It is stable if $ V_{\text{eff}} $ has a minimum at $r_*$ and unstable if $V_{\text{eff}}$ has a maximum at $r_*$. Hence it is stable if $ {\color{blue}V_{\text{eff}}''(r_*)>0} $ and is unstable $ {\color{blue}V_{\text{eff}}''(r_*)<0} $. 

Write in terms of $V(r)$: 
\[
    V'(r_*)-\frac{mh^2}{r_*^3}=0.
\]
Therefore it is stable if
\[
    V''(r_*)+\frac{3mh^2}{r_*^4}>0 \Longrightarrow {\color{blue}V''(r_*)+\frac{3V'(r_*)}{r_*}>0}.
\]

\begin{example}
    Consider $ V(r)=-km/r^p,p>0,k>0 $. For circular motion,
    \[
        \frac{pkm}{r_*^{p+1}}-\frac{mh^2}{r_*^3}=0 \Longrightarrow r_*^{p-2}=\frac{pk}{h^2}.
    \]
    Hence it is circular for all $h$ except $p=2$. Stability check:
    \begin{align*}
        V''(r_*) + \frac{3}{r_*}V'(r_*) &= \big( -p(p + 1) + 3p\big)\frac{mk}{r_*^{p + 2}}\\
        &= p(2-p)\frac{mk}{r_*^{p + 2}} \begin{cases}
            >0 &\text{if }0<p<2,\text{ stable}.\\
            <0 &\text{if }p>2,\text{ unstable}.\\
            \end{cases} 
    \end{align*}
    \begin{center}
        \begin{tikzpicture}[xscale=0.5]
          \draw (0, 0) -- (8, 0);
          \draw (0, -2) -- (0, 2) node [above] {$V_{\text{eff}}$};
          \draw [blue, samples=10, domain=0.5:1.5] plot (\x, {-3/\x + 2/(\x*\x)});
          \draw [blue, domain=1.5:7.8] plot (\x, {-3/\x + 2/(\x*\x)}) node [right] {$p = 1$};
          \draw [red, dashed, samples=70, domain=0.4:7.8] plot (\x, {-0.6/(\x*\x*\x) + 1.2/(\x*\x)}) node [right] {$p = 3$};
        \end{tikzpicture}
    \end{center}
\end{example}

\subsection{The orbit equation}
The shape of the orbit is determined by the joint variants of $ r(t),\theta(t) $. In principle we could determine $r(t)$ via 
\[
    E=\frac{1}{2}m \dot{r}^2+V_{\text{eff}}(r)=\text{const} \Longrightarrow t = \pm \frac{m}{2}\int^{r} \frac{\rmd r'}{\sqrt{E-V_{\text{eff}}(r')}},
\]
giving $r(t)$. Then we use $ r(t)^2 \dot{\theta}=h $ to obtain $\theta$. In practice this is not very helpful, e.g . analytic solutions are possible only for a small family of $ V_{\text{eff}}(r) $. A better approach is to use $ \theta $ is the independent variable, by writing
\[
    \frac{\mathrm{d}}{\mathrm{d}t} = \frac{\mathrm{d}\theta}{\mathrm{d}t}\frac{\mathrm{d}}{\mathrm{d}\theta} = \frac{h}{r^2}\frac{\mathrm{d}}{\mathrm{d}\theta}.    
\]
Apply Newton's 2nd law: 
\[
    m \left( \frac{h}{r^2}\frac{\mathrm{d}}{\mathrm{d}\theta} \right)^2(r)-\frac{mh^2}{r^3}=F(r)
\]
(refer to (\ref{eq:radial component})) for some $F(r)$. This suggests using $u=1/r$: 
\begin{align*}
    &mhu^2 \frac{\mathrm{d}}{\mathrm{d}\theta}\left( -h \frac{\mathrm{d}u}{\mathrm{d}\theta}  \right) -mh^2u^3=F(u^{-1})\\ 
    \Longrightarrow & {\color{blue}\frac{\mathrm{d}^2 u}{\mathrm{d}\theta^2}+u = -\frac{1}{mh^2u^2} F(u^{-1})}.
\end{align*}
This is called the \textbf{orbit equation} or \textbf{Binet equation}. Solve this to find $ u(\theta) $, then use $ \dot{\theta} = hu^2 $ to get $ \theta(t) $ and thus $ r(t) $.

\subsection{The Kepler problem}
\subsubsection*{General shapes}
This is the orbit problem for the special case of a gravitational central force, such that 
\[
    F(r) = -\frac{mk}{r^2}.
\]
From the previous section we know that 
\[
    \frac{\mathrm{d}^2 u}{\mathrm{d}\theta^2}+u = \frac{1}{mh^2u^2} \cdot mku^2 = \frac{k}{h^2},
\]
which is linear in $u$. This is a simple harmonic problem and the solution is 
\[
    u = \frac{k}{h^2}A \cos (\theta-\theta_0)
\]
for some $A$ and $ \theta_0 $ to be determined by I.C.s. 

Assume wlog that $ A\ge 0 $.
\begin{itemize}
    \item If $ A=0,u=k/h^2 $ so we have a \textit{circular orbit}.
    \item If $A>0$, $u$ is maximised when $ \theta=\theta_0 $ and $r$ is minimised (periapsis).
\end{itemize}

Now choose I.C.s so that $ \theta_0=0 $. Let 
\[
    r = \frac{1}{u} = \frac{l}{1+e \cos \theta},\quad \text{where }l=\frac{h^2}{k},e=\frac{Ah^2}{k}.
\]
Thus we get the polar form of \textbf{conic sections}. $e$ is called the \textbf{eccentricity}, and it determines the shape of the curve.
\subsubsection*{Rewrite in Cartesians}
In Cartesian coordinates, $r(1+e\cos\theta)=l \Rightarrow r+ex=l \Rightarrow r=l-ex$. Squaring both sides gives $ x^2+y^2=(l-ex)^2 $, and thus 
\begin{equation}\label{eq:kepler}\tag{$\dagger$}
    (1-e^2)x^2+y^2+2elx = l^2.
\end{equation} 
\begin{itemize}
    \item $ 0\le e<1 $. Ellipse - orbit is bounded. We have 
    \[
        \frac{1}{1+e}\le r(\theta)\le \frac{l}{1-e}.
    \]
    Note that (\ref{eq:kepler}) can be standardised:
    \[
        \frac{(x+ea)^2}{a^2}+\frac{y^2}{b^2}=1,\quad \text{where }a = \frac{l}{1-e^2},\ b=\frac{l}{\sqrt{1-e^2}},\ b\le a.
    \]
    When $e=0$, $a=b=$ radius of the circle. When $e>0$, the origin lies at one of the foci. Note that $e,l$ will determine $a,b$ and vice versa.
    \begin{center}
        \begin{tikzpicture}[xscale=1.5]
          \draw [gray] (-2, 0) -- (2, 0);
          \draw [gray] (0, -1.6) -- (0, 1.6);
          \draw [gray, ->] (0.7, -0.2) -- (0, -0.2) node [gray!50!black, pos = 0.5, below] {$ae$};
          \draw [gray, ->] (0, -0.2) -- (0.7, -0.2);
          \draw [gray, ->] (-0.2, 0) -- (-0.2, 1.6) node [gray!50!black, pos = 0.5, left] {$b$};
          \draw [gray, ->] (-0.2, 1.6) -- (-0.2, 0);
  
          \draw [gray, ->] (0, -0.2) -- (-2, -0.2) node [gray!50!black, pos = 0.5, below] {$a$};
          \draw [gray, ->] (-2, -0.2) -- (0, -0.2);
          \draw [gray, dashed] (0.7, 0) -- (0.7, 1.5) node [gray!50!black, pos = 0.5, right] {$l$};
          \draw (.7, 0) node [anchor = south west] {$O$} node [dot=8pt] {};
          \draw [->-=0.1] (0, 0) circle [x radius = 2, y radius = 1.6];
        \end{tikzpicture}
    \end{center} 
    \item $e>1$. Then the orbit is hyperbolic and $r\to \infty$ as $ \theta\to \pm \alpha $, where $ \alpha=\arccos (-1/e)\in (\pi/2,\pi) $. We can standardise (\ref{eq:kepler}) as 
    \[
        \frac{(x-ea)^2}{a^2}-\frac{y^2}{b^2}=1,\quad a = \frac{l}{e^2-1},\ b=\frac{l}{\sqrt{e^2-1}}.
    \]
    This represents an incoming body with large velocity which is deflected by gravity. The asymptotes are 
    \[
        y = \pm \frac{b}{a}(x-ea) \Longleftrightarrow bx\mp ay=eba.
    \]
    The normal vectors are $ \bfn= \frac{1}{\sqrt{a^2+b^2}}(b,\mp a) $. The perpendicular distance between incoming trajectory and $O$ is $ \bfr \cdot \bfn = \frac{bx\mp ay}{\sqrt{a^2+b^2}}= \frac{eba}{\sqrt{a^2+b^2}}=b$, which is often called the \textbf{impact parameter}. 
    \begin{center}
    \begin{tikzpicture}
        \draw [gray] (-3, -2) -- (3, 2);
        \draw [gray] (3, -2) -- (-3, 2);
        \draw [->-=0.3] (-3, -1.9) .. controls (-0.2, 0) .. (-3, 1.9);
        \draw [gray, ->] (0, 0) -- (-0.9, 0);
        \draw [gray, ->] (-0.9, 0) -- (0, 0) node [pos = 0.5, above] {$a$};
        \node at (-2, 0) [dot=6pt] {};
        \node at (-2, 0) [left] {$O$};
        \draw [gray, dashed] (-2, 0) -- (-2, 1.2) node [pos = 0.5, left] {$l$};
        \draw [gray, dashed] (-2, 0) -- (-1.38462, -0.923077) node [pos = 0.5, right] {$b$};
    \end{tikzpicture}
    \end{center}
    \item $e=1$. Then the orbit is parabolic and the equation is given by 
    \[
        r=\frac{l}{1+\cos \theta}.
    \]
    Note that $ r\to \infty $ as $ \theta\to \pm \pi $. In Cartesians,
    \[
        y^2=2l(l-x).
    \]
\end{itemize}

\subsubsection*{Energy and eccentricity}
Recall that in polars,
\begin{align*}
    E&=\frac{1}{2}m \left( \dot{r}^2+r^2 \dot{\theta}^2 \right) - \frac{mk}{r}=\frac{1}{2}mh^2\left( \left( \frac{\mathrm{d}u}{\mathrm{d}\theta} \right)^2+u^2  \right)-mku\\
    &= \frac{mh^2}{2l^2}\left( e^2 \sin ^2 \theta+(1+e \cos\theta)^2 \right)-\frac{mk}{l}(1+ \cos \theta)\\ 
    &= \frac{mk}{2l}(e^2-1).
\end{align*}
Hence the orbit is bounded if $ E<0 \Leftrightarrow e<1 $, and unbounded if $ e>1 \Leftrightarrow E>0 $. We have a parabola if $ e=1 \Leftrightarrow E=0 $, the marginal case. (Note also that $ e=\sqrt{\frac{2lE}{mk}+1}$.)

\subsubsection*{Kepler's laws of planetary motion}
When Kepler first studied the laws of planetary motion, he took a telescope, observed actual planets, and came up with his famous three laws of motion. We are now going to derive the laws with pen and paper instead.

\begin{law}[Kepler's first law]
  The orbit of each planet is an ellipse with the Sun at one focus.
\end{law}

\begin{law}[Kepler's second law]
  The line between the planet and the sun sweeps out equal areas in equal times.
\end{law}

\begin{law}[Kepler's third law]
  The square of the orbital period is proportional to the cube of the semi-major axis, or
  \[
    P^2 \propto a^3.
  \]
\end{law}

Law 1 is consistent with our solution of orbit equation. Law 2 follows simply from the conservation of angular momentum: The area swept out by moving $\rmd \theta$ is $\rmd A = \frac{1}{2}r^2\;\rmd \theta$ (area of sector of circle). So
\[
  \frac{\rmd A}{\rmd t} = \frac{1}{2}r^2\dot{\theta} = \frac{h}{2} = \text{const}.
\]
and is true for \emph{any} central force.

Law 3 follows from this: the total area of the ellipse is $A = \pi ab = \frac{h}{2}P$ (by the second law). But $b^2 = a^2( 1 - e^2)$ and $h^2 = k\ell = ka(1 - e^2)$ (check!). So
\[
  P^2 = \frac{(2\pi)^2a^4(1 - e^2)}{ka(1 - e^2)} = \frac{(2\pi)^2 a^3}{k}.
\]
Note that the third law is very easy to prove directly for circular orbits. Since the radius is constant, $\ddot{r} = 0$. So the equations of motion give
\[
  -r \dot{\theta}^2 = -\frac{k}{r^2}
\]
So
\[
  r^3 \dot{\theta}^2 = k
\]
Since $\dot{\theta}\propto P^{-1}$, the result follows.
\subsection{Rutherford scattering}

Consider a positive charge fixed towards another fixed positive charge. What is the scattering angle $ \beta $?

Consider motion in a \textit{repulsive} square law force
\[
    V(r) = \frac{mk}{r},\quad F(r) = \frac{mk}{r^2}.
\]
The solution orbit equation is 
\[
    u = -\frac{k}{h^2}+ a \cos (\theta-\theta_0).
\]
Wlog suppose $ \theta_0=0,A\ge 0 $. Rewrite as 
\[
    r = \frac{l}{e\cos \theta-1 },\quad l=\frac{h^2}{k},e=\frac{Ah^2}{k}.
\]
We require $ e>1 $ for $r>0$ and for some $ \theta $. Then $ r\to \infty $ as $ \theta\to \pm \alpha $ with $ \alpha=\arccos (1/e)\in (0,\pi/2) $. This represents a hyperbola. In Cartesians,
\[
    \frac{(x-ea)^2}{a^2}-\frac{y^2}{b^2}=1,\quad a=\frac{l}{e^2-1},b=\frac{l}{\sqrt{e^2-1}}.
\]
\begin{center}
    \begin{tikzpicture}
      \draw [gray] (-3, -2) -- (3, 2);
      \draw [gray] (3, -2) -- (-3, 2);
      \draw [->-=0.3] (3, -1.9) .. controls (0.2, 0) .. (3, 1.9);
      \node at (-2, 0) [dot] {};
      \node at (-2, 0) [left] {$O$};
      \draw [dashed] (-2, 0) -- (-1.38462, 0.923077) node [pos = 0.5, right] {$b$};
      \draw (0.3, 0) arc (0:-33:0.3);
      \draw (0.3, 0) arc (0:33:0.3);
      \node at (0.3, 0) [right] {$2\alpha$};
      \draw (0, 0.4) arc (90:33:0.4);
      \draw (0, 0.4) arc (90:147:0.4);
      \node at (0, 0.4) [above] {$\beta$};
    \end{tikzpicture}
\end{center}
It seems as if the particle is deflected by $O$.

We can characterize the path of the particle by the impact parameter $b$ and the incident speed $v$ (i.e.\ the speed when far away from the origin). We know that the angular momentum per unit mass is $h = bv$ (velocity $\times$ perpendicular distance to $O$).

How does the scattering angle $\beta = \pi - 2\alpha$ depend on the impact parameter $b$ and the incident speed $v$?

Recall that the angle $\alpha$ is given by $\alpha = \cos^{-1} (1/e)$. So we obtain
\[
  \frac{1}{e} = \cos \alpha = \cos \left(\frac{\pi}{2} - \frac{\beta}{2}\right) = \sin\left(\frac{\beta}{2}\right),
\]
So
\[
  b = \frac{l}{\sqrt{e^2 - 1}} = \frac{(bv)^2}{k}\tan \frac{\beta}{2}\Longrightarrow \beta = 2\tan^{-1}\left(\frac{k}{bv^2}\right).
\]
We see that if we have a small impact parameter, i.e.\ $b \ll k/v^2$, then we can have a scattering angle approaching $\pi$.