\section{Newtonian Dynamics - Basic Concepts}
\subsection{Particles}
\begin{definition}[Particle]
    A \textbf{particle} is an object of negligible size and a finite mass $m>0$. (Perhaps) it has an electric charge $q$. 
\end{definition}
\begin{definition}[Position vector]
    Position of a particle is described by a \textbf{position vector} $\bfr(t)$ or $\bfx(t)$ wrt to origin $O$. A particle at the origin has position vector $ \mathbf{0} $. The Cartesian components of $\bfr(t)$ are given by $(x,y,z): 
    \mathbf{r}= x \bfi + y\bfj + z\bfk$. 
\end{definition}
\begin{definition}[Frame of reference]
    The choice of coordinate axes defines a \textbf{frame of reference} $S$.
\end{definition}
\begin{definition}[Velocity]
    The \textbf{velocity} of a particle is 
    \[
        \bfu = \frac{\mathrm{d}}{\mathrm{d}t} \bfr(t) = \dot{\bfr}. 
    \]
    Note that $ \bfu $ is \textit{tangent} to path. See Vector Calculus for more details.
\end{definition}
\begin{definition}[Momentum]
    The \textbf{momentum} of a particle is 
    \[
        m\bfu = m \dot{\bfr} = \bfp.
    \]
\end{definition}
\begin{definition}[Acceleration]
    The \textbf{acceleration} of the particle is 
    \[
        \dot{\bfu}= \ddot{\bfr} = \frac{\mathrm{d}^2\bfr}{\mathrm{d}t^2}.
    \]
\end{definition}
\begin{note}
    Time derivative of $\bfu(t)$ is
    \[
        \dot{\bfu}(t) = \lim_{h \to 0} \frac{\bfu(t+h)-\bfu(t)}{h},
    \]
    with $\bfu\to\bfu_0$ if and only if $|\bfu-\bfu_0|\to 0$. Then we can evaluate derivative by taking derivative of each component:
    \[
        \frac{\mathrm{d}\bfr}{\mathrm{d}t} = \left( \frac{\mathrm{d}x}{\mathrm{d}t}, \frac{\mathrm{d}y}{\mathrm{d}t}, \frac{\mathrm{d}z}{\mathrm{d}t} \right). 
    \]
\end{note}
\begin{proposition}[Product rules]\label{prop:product rules}
\begin{align*}
    \frac{\mathrm{d}}{\mathrm{d}t}(f\bfg) &= \frac{\mathrm{d}f}{\mathrm{d}t} \cdot \bfg + f \frac{\mathrm{d}\bfg}{\mathrm{d}t},\\
    \frac{\mathrm{d}}{\mathrm{d}t}(\bfg \cdot \bfh) &= \frac{\mathrm{d}\bfg}{\mathrm{d}t} \cdot \bfh + \bfg \cdot \frac{\mathrm{d}\bfh}{\mathrm{d}t},\\ 
    \frac{\mathrm{d}}{\mathrm{d}t}(\bfg \times \bfh) &= \frac{\mathrm{d}\bfg}{\mathrm{d}t} \times \bfh + \bfg \times \frac{\mathrm{d}\bfh}{\mathrm{d}t}\quad\text{Order matters}.\\
\end{align*}
\end{proposition}
\subsection{Newton's Laws of Motions}
\begin{law}[Newton's First Law of Motion]
    There exists \textbf{inertial frame of reference} in which a particle remains at rest or move in a straight line at constant \textit{speed}(i.e. it moves at constant \textit{velocity}) unless it is acted on by a \textit{force}.
\end{law}
\begin{law}[Newton's Second Law of Motion]
    In an inertial frame, the rate of change of momentum of a particle is equal to the force acting on it.
\end{law}
\begin{law}[Newton's Third Law of Motion]
    To every action there is an \textit{equal} and \textit{opposite} reaction
\end{law}
\begin{note}
    All of these statements about particles can be extended to \textit{finite bodies} composed of many particles.
\end{note}