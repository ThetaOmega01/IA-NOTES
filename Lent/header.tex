\usepackage{sidenotes,fancyhdr,titlesec,geometry}
\usepackage[dvipsnames]{xcolor}
\usepackage[many]{tcolorbox}
\usepackage{xifthen}
\usepackage{import}
\usepackage{parskip}
\usepackage{pdfpages}
\usepackage{transparent}
\usepackage{mathtools,amssymb,amsfonts,amsthm,bm}   % Math Presets
\usepackage{array,tabularx,booktabs}                % Table Presets
\usepackage{graphicx,wrapfig,float,caption}         % Figure Presets
\usepackage{setspace,multicol}                      % Text Presets
\usepackage{tikz,physics,cancel,tkz-euclide,pgfplots,tikz-3dplot}                    % Physics Presets
\usepackage{amsmath}
\usepackage{mathrsfs}
\usepackage{enumerate}
\usepackage[shortlabels]{enumitem}
\usepackage{hyperref}
\usepackage{lipsum}
\tdplotsetmaincoords{60}{120}

\usetikzlibrary{arrows.meta}
\usetikzlibrary{decorations.markings}
\usetikzlibrary{decorations.pathmorphing}
\usetikzlibrary{positioning}
\usetikzlibrary{fadings}
\usetikzlibrary{intersections}
\usetikzlibrary{cd}
\pgfarrowsdeclarecombine{twolatex'}{twolatex'}{latex'}{latex'}{latex'}{latex'}
\tikzset{->/.style = {decoration={markings,
                                  mark=at position 1 with {\arrow[scale=1.6]{latex'}}},
                      postaction={decorate}}}
\tikzset{<-/.style = {decoration={markings,
                                  mark=at position 0 with {\arrowreversed[scale=1.6]{latex'}}},
                      postaction={decorate}}}
\tikzset{<->/.style = {decoration={markings,
                                   mark=at position 0 with {\arrowreversed[scale=1.6]{latex'}},
                                   mark=at position 1 with {\arrow[scale=1.6]{latex'}}},
                       postaction={decorate}}}
\tikzset{->-/.style = {decoration={markings,
                                   mark=at position #1 with {\arrow[scale=1.6]{latex'}}},
                       postaction={decorate}}}
\tikzset{-<-/.style = {decoration={markings,
                                   mark=at position #1 with {\arrowreversed[scale=1.6]{latex'}}},
                       postaction={decorate}}}
\tikzset{->>/.style = {decoration={markings,
                                  mark=at position 1 with {\arrow[scale=1.6]{twolatex'}}},
                      postaction={decorate}}}
\tikzset{<<-/.style = {decoration={markings,
                                  mark=at position 0 with {\arrowreversed[scale=1.6]{twolatex'}}},
                      postaction={decorate}}}
\tikzset{<<->>/.style = {decoration={markings,
                                   mark=at position 0 with {\arrowreversed[scale=1.6]{twolatex'}},
                                   mark=at position 1 with {\arrow[scale=1.6]{twolatex'}}},
                       postaction={decorate}}}
\tikzset{->>-/.style = {decoration={markings,
                                   mark=at position #1 with {\arrow[scale=1.6]{twolatex'}}},
                       postaction={decorate}}}
\tikzset{-<<-/.style = {decoration={markings,
                                   mark=at position #1 with {\arrowreversed[scale=1.6]{twolatex'}}},
                       postaction={decorate}}}

\tikzset{
set arrow inside/.code={\pgfqkeys{/tikz/arrow inside}{#1}},
set arrow inside={end/.initial=>, opt/.initial=},
/pgf/decoration/Mark/.style={
    mark/.expanded=at position #1 with
    {
        \noexpand\arrow[\pgfkeysvalueof{/tikz/arrow inside/opt}]{\pgfkeysvalueof{/tikz/arrow inside/end}}
    }
},
arrow inside/.style 2 args={
    set arrow inside={#1},
    postaction={
        decorate,decoration={
            markings,Mark/.list={#2}
        }
    }
},
}

\tikzstyle{circ}=[fill=black, draw=black, shape=circle]
\tikzset{
dot/.style = {circle, fill, minimum size=#1,
              inner sep=0pt, outer sep=0pt},
dot/.default = 5pt% size of the circle diameter 
}

\def\centerarc[#1](#2)(#3:#4:#5)% Syntax: [draw options] (center) (initial angle:final angle:radius)
    { \draw[#1] ($(#2)+({#5*cos(#3)},{#5*sin(#3)})$) arc (#3:#4:#5); }

\hypersetup{
    colorlinks=true,
    linkcolor=blue,
    filecolor=blue,
    citecolor = black,      
    urlcolor=cyan,
    }

%%%%%%%%%%% Snippets %%%%%%%%%%%%%%%%
\newcommand*\widefbox[1]{\fbox{\hspace{2em}#1\hspace{2em}}}
\newcommand{\xint}{\int_{x_1}^{x_2}}
\newcommand{\mw}{\sqrt{m\omega}}
\newcommand{\de}{\delta}
\newcommand{\dde}{\dot{\delta}}
\newcommand{\di}{\delta_i}
\newcommand{\ddi}{\dot{\delta_i}}
\newcommand{\dddi}{\ddot{\delta_i}}
\newcommand{\dipl}{\delta_{i+1}}
\newcommand{\dimi}{\delta_{i-1}}
\newcommand{\ddt}[1]{\frac{{d} #1}{dt}}
\newcommand{\ddtt}[1]{\frac{d^2 #1}{dt^2}}
\newcommand{\ddx}[1]{\frac{d #1}{dx}}
\newcommand{\ddxx}[1]{\frac{d^2 #1}{dx^2}}
\newcommand{\eps}{\epsilon}
\newcommand{\del}[2]{\frac{\partial #1}{\partial #2}}
\newcommand{\deltwo}[2]{\frac{\partial^2 #1}{\partial #2^2}}
\newcommand{\lam}{\lambda}
\newcommand{\Lam}{\Lambda}
\newcommand{\sig}{\sigma}
\newcommand{\Sig}{\Sigma}
\newcommand{\half}{\frac{1}{2}}
\newcommand{\munu}{{\mu\nu}}
\newcommand{\thalf}{\tfrac{1}{2}}

\DeclareMathOperator{\orb}{Orb}
\DeclareMathOperator{\stab}{Stab}
\DeclareMathOperator{\adj}{adj}
\DeclareMathOperator{\ccl}{ccl}
\let\var\relax
\DeclareMathOperator{\var}{Var}
\DeclareMathOperator{\cov}{Cov}
\DeclareMathOperator{\corr}{Corr}

\newcommand{\bfA}{{\bf A}}
\newcommand{\bfB}{{\bf B}}
\newcommand{\bfC}{{\bf C}}
\newcommand{\bfD}{{\bf D}}
\newcommand{\bfE}{{\bf E}}
\newcommand{\bfF}{{\bf F}}
\newcommand{\bfG}{{\bf G}}
\newcommand{\bfH}{{\bf H}}
\newcommand{\bfI}{{\bf I}}
\newcommand{\bfJ}{{\bf J}}
\newcommand{\bfK}{{\bf K}}
\newcommand{\bfL}{{\bf L}}
\newcommand{\bfM}{{\bf M}}
\newcommand{\bfN}{{\bf N}}
\newcommand{\bfO}{{\bf O}}
\newcommand{\bfP}{{\bf P}}
\newcommand{\bfQ}{{\bf Q}}
\newcommand{\bfR}{{\bf R}}
\newcommand{\bfS}{{\bf S}}
\newcommand{\bfT}{{\bf T}}
\newcommand{\bfU}{{\bf U}}
\newcommand{\bfV}{{\bf V}}
\newcommand{\bfW}{{\bf W}}
\newcommand{\bfX}{{\bf X}}
\newcommand{\bfY}{{\bf Y}}
\newcommand{\bfZ}{{\bf Z}}

\newcommand{\bfa}{{\bf a}}
\newcommand{\bfb}{{\bf b}}
\newcommand{\bfc}{{\bf c}}
\newcommand{\bfd}{{\bf d}}
\newcommand{\bfe}{{\bf e}}
\newcommand{\bff}{{\bf f}}
\newcommand{\bfg}{{\bf g}}
\newcommand{\bfh}{{\bf h}}
\newcommand{\bfi}{{\bf i}}
\newcommand{\bfj}{{\bf j}}
\newcommand{\bfk}{{\bf k}}
\newcommand{\bfl}{{\bf l}}
\newcommand{\bfm}{{\bf m}}
\newcommand{\bfn}{{\bf n}}
\newcommand{\bfo}{{\bf o}}
\newcommand{\bfp}{{\bf p}}
\newcommand{\bfq}{{\bf q}}
\newcommand{\bfr}{{\bf r}}
\newcommand{\bfs}{{\bf s}}
\newcommand{\bft}{{\bf t}}
\newcommand{\bfu}{{\bf u}}
\newcommand{\bfv}{{\bf v}}
\newcommand{\bfw}{{\bf w}}
\newcommand{\bfx}{{\bf x}}
\newcommand{\bfy}{{\bf y}}
\newcommand{\bfz}{{\bf z}}

\newcommand{\mcA}{{\mathcal{A}}}
\newcommand{\mcB}{{\mathcal{B}}}
\newcommand{\mcC}{{\mathcal{C}}}
\newcommand{\mcD}{{\mathcal{D}}}
\newcommand{\mcE}{{\mathcal{E}}}
\newcommand{\mcF}{{\mathcal{F}}}
\newcommand{\mcG}{{\mathcal{G}}}
\newcommand{\mcH}{{\mathcal{H}}}
\newcommand{\mcI}{{\mathcal{I}}}
\newcommand{\mcJ}{{\mathcal{J}}}
\newcommand{\mcK}{{\mathcal{K}}}
\newcommand{\mcL}{{\mathcal{L}}}
\newcommand{\mcM}{{\mathcal{M}}}
\newcommand{\mcN}{{\mathcal{N}}}
\newcommand{\mcO}{{\mathcal{O}}}
\newcommand{\mcP}{{\mathcal{P}}}
\newcommand{\mcQ}{{\mathcal{Q}}}
\newcommand{\mcR}{{\mathcal{R}}}
\newcommand{\mcS}{{\mathcal{S}}}
\newcommand{\mcT}{{\mathcal{T}}}
\newcommand{\mcU}{{\mathcal{U}}}
\newcommand{\mcV}{{\mathcal{V}}}
\newcommand{\mcW}{{\mathcal{W}}}
\newcommand{\mcX}{{\mathcal{X}}}
\newcommand{\mcY}{{\mathcal{Y}}}
\newcommand{\mcZ}{{\mathcal{Z}}}

\newcommand{\bbA}{{\mathbb{A}}}
\newcommand{\bbB}{{\mathbb{B}}}
\newcommand{\bbC}{{\mathbb{C}}}
\newcommand{\bbD}{{\mathbb{D}}}
\newcommand{\bbE}{{\mathbb{E}}}
\newcommand{\bbF}{{\mathbb{F}}}
\newcommand{\bbG}{{\mathbb{G}}}
\newcommand{\bbH}{{\mathbb{H}}}
\newcommand{\bbI}{{\mathbb{I}}}
\newcommand{\bbJ}{{\mathbb{J}}}
\newcommand{\bbK}{{\mathbb{K}}}
\newcommand{\bbL}{{\mathbb{L}}}
\newcommand{\bbM}{{\mathbb{M}}}
\newcommand{\bbN}{{\mathbb{N}}}
\newcommand{\bbO}{{\mathbb{O}}}
\newcommand{\bbP}{{\mathbb{P}}}
\newcommand{\bbQ}{{\mathbb{Q}}}
\newcommand{\bbR}{{\mathbb{R}}}
\newcommand{\bbS}{{\mathbb{S}}}
\newcommand{\bbT}{{\mathbb{T}}}
\newcommand{\bbU}{{\mathbb{U}}}
\newcommand{\bbV}{{\mathbb{V}}}
\newcommand{\bbW}{{\mathbb{W}}}
\newcommand{\bbX}{{\mathbb{X}}}
\newcommand{\bbY}{{\mathbb{Y}}}
\newcommand{\bbZ}{{\mathbb{Z}}}

\newcommand{\mfa}{{\mathfrak{a}}}
\newcommand{\mfb}{{\mathfrak{b}}}
\newcommand{\mfc}{{\mathfrak{c}}}
\newcommand{\mfd}{{\mathfrak{d}}}
\newcommand{\mfe}{{\mathfrak{e}}}
\newcommand{\mff}{{\mathfrak{f}}}
\newcommand{\mfg}{{\mathfrak{g}}}
\newcommand{\mfh}{{\mathfrak{h}}}
\newcommand{\mfi}{{\mathfrak{i}}}
\newcommand{\mfj}{{\mathfrak{j}}}
\newcommand{\mfk}{{\mathfrak{k}}}
\newcommand{\mfl}{{\mathfrak{l}}}
\newcommand{\mfm}{{\mathfrak{m}}}
\newcommand{\mfn}{{\mathfrak{n}}}
\newcommand{\mfo}{{\mathfrak{o}}}
\newcommand{\mfp}{{\mathfrak{p}}}
\newcommand{\mfq}{{\mathfrak{q}}}
\newcommand{\mfr}{{\mathfrak{r}}}
\newcommand{\mfs}{{\mathfrak{s}}}
\newcommand{\mft}{{\mathfrak{t}}}
\newcommand{\mfu}{{\mathfrak{u}}}
\newcommand{\mfv}{{\mathfrak{v}}}
\newcommand{\mfw}{{\mathfrak{w}}}
\newcommand{\mfx}{{\mathfrak{x}}}
\newcommand{\mfy}{{\mathfrak{y}}}
\newcommand{\mfz}{{\mathfrak{z}}}

\newcommand{\mfA}{{\mathfrak{A}}}
\newcommand{\mfB}{{\mathfrak{B}}}
\newcommand{\mfC}{{\mathfrak{C}}}
\newcommand{\mfD}{{\mathfrak{D}}}
\newcommand{\mfE}{{\mathfrak{E}}}
\newcommand{\mfF}{{\mathfrak{F}}}
\newcommand{\mfG}{{\mathfrak{G}}}
\newcommand{\mfH}{{\mathfrak{H}}}
\newcommand{\mfI}{{\mathfrak{I}}}
\newcommand{\mfJ}{{\mathfrak{J}}}
\newcommand{\mfK}{{\mathfrak{K}}}
\newcommand{\mfL}{{\mathfrak{L}}}
\newcommand{\mfM}{{\mathfrak{M}}}
\newcommand{\mfN}{{\mathfrak{N}}}
\newcommand{\mfO}{{\mathfrak{O}}}
\newcommand{\mfP}{{\mathfrak{P}}}
\newcommand{\mfQ}{{\mathfrak{Q}}}
\newcommand{\mfR}{{\mathfrak{R}}}
\newcommand{\mfS}{{\mathfrak{S}}}
\newcommand{\mfT}{{\mathfrak{T}}}
\newcommand{\mfU}{{\mathfrak{U}}}
\newcommand{\mfV}{{\mathfrak{V}}}
\newcommand{\mfW}{{\mathfrak{W}}}
\newcommand{\mfX}{{\mathfrak{X}}}
\newcommand{\mfY}{{\mathfrak{Y}}}
\newcommand{\mfZ}{{\mathfrak{Z}}}

\newcommand{\rma}{\mathrm{a}}
\newcommand{\rmb}{\mathrm{b}}
\newcommand{\rmc}{\mathrm{c}}
\newcommand{\rmd}{\mathrm{d}}
\renewcommand{\dd}{\,\mathrm{d}}
\newcommand{\rme}{\mathrm{e}}
\newcommand{\rmf}{\mathrm{f}}
\newcommand{\rmg}{\mathrm{g}}
\newcommand{\rmh}{\mathrm{h}}
\newcommand{\rmi}{\mathrm{i}}
\newcommand{\rmj}{\mathrm{j}}
\newcommand{\rmk}{\mathrm{k}}
\newcommand{\rml}{\mathrm{l}}
\newcommand{\rmm}{\mathrm{m}}
\newcommand{\rmn}{\mathrm{n}}
\newcommand{\rmo}{\mathrm{o}}
\newcommand{\rmp}{\mathrm{p}}
\newcommand{\rmq}{\mathrm{q}}
\newcommand{\rmr}{\mathrm{r}}
\newcommand{\rms}{\mathrm{s}}
\newcommand{\rmt}{\mathrm{t}}
\newcommand{\rmu}{\mathrm{u}}
\newcommand{\rmv}{\mathrm{v}}
\newcommand{\rmw}{\mathrm{w}}
\newcommand{\rmx}{\mathrm{x}}
\newcommand{\rmy}{\mathrm{y}}
\newcommand{\rmz}{\mathrm{z}}
\newcommand{\rmA}{\mathrm{A}}
\newcommand{\rmB}{\mathrm{B}}
\newcommand{\rmC}{\mathrm{C}}
\newcommand{\rmD}{\mathrm{D}}
\newcommand{\rmE}{\mathrm{E}}
\newcommand{\rmF}{\mathrm{F}}
\newcommand{\rmG}{\mathrm{G}}
\newcommand{\rmH}{\mathrm{H}}
\newcommand{\rmI}{\mathrm{I}}
\newcommand{\rmJ}{\mathrm{J}}
\newcommand{\rmK}{\mathrm{K}}
\newcommand{\rmL}{\mathrm{L}}
\newcommand{\rmM}{\mathrm{M}}
\newcommand{\rmN}{\mathrm{N}}
\newcommand{\rmO}{\mathrm{O}}
\newcommand{\rmP}{\mathrm{P}}
\newcommand{\rmQ}{\mathrm{Q}}
\newcommand{\rmR}{\mathrm{R}}
\newcommand{\rmS}{\mathrm{S}}
\newcommand{\rmT}{\mathrm{T}}
\newcommand{\rmU}{\mathrm{U}}
\newcommand{\rmV}{\mathrm{V}}
\newcommand{\rmW}{\mathrm{W}}
\newcommand{\rmX}{\mathrm{X}}
\newcommand{\rmY}{\mathrm{Y}}
\newcommand{\rmZ}{\mathrm{Z}}

\newcommand{\GL}{\mathrm{GL}}
\newcommand{\Or}{\mathrm{O}}
\newcommand{\PGL}{\mathrm{PGL}}
\newcommand{\PSL}{\mathrm{PSL}}
\newcommand{\PSO}{\mathrm{PSO}}
\newcommand{\PSU}{\mathrm{PSU}}
\newcommand{\SL}{\mathrm{SL}}
\newcommand{\SO}{\mathrm{SO}}
\newcommand{\Spin}{\mathrm{Spin}}
\newcommand{\Sp}{\mathrm{Sp}}
\newcommand{\SU}{\mathrm{SU}}
\newcommand{\U}{\mathrm{U}}
\newcommand{\Mat}{\mathrm{Mat}}

% Matrix algebras
\newcommand{\gl}{\mathfrak{gl}}
\newcommand{\ort}{\mathfrak{o}}
\newcommand{\so}{\mathfrak{so}}
\newcommand{\su}{\mathfrak{su}}
\newcommand{\uu}{\mathfrak{u}}
\renewcommand{\sl}{\mathfrak{sl}}
\DeclareMathOperator{\spn}{span}

\newcommand{\mobius}{{M\"{o}bius }}

\renewcommand{\ge}{\geqslant}
\renewcommand{\le}{\leqslant}

\newcommand\independent{\protect\mathpalette{\protect\independenT}{\perp}}
\def\independenT#1#2{\mathrel{\rlap{$#1#2$}\mkern2mu{#1#2}}}

\setlength{\parindent}{0pt}

\newcommand{\incfig}[1]{%
    \def\svgwidth{0.4\columnwidth}
    \import{./figures/}{#1.pdf_tex}
}
%%%%%%%%%%%%%%%%%%%%%%%%%%%%%%%%%%%%%

%%%%%%%boxed enviroment for final layout%%%%%%%%%%%%%

% \tcolorboxenvironment{definition}{
%   boxrule=0pt,
%   boxsep=0pt,
%   colback={White!90!Cerulean},
%   enhanced jigsaw, 
%   borderline west={2pt}{0pt}{Cerulean},
%   sharp corners,
%   before skip=10pt,
%   after skip=10pt,
%   breakable,
% }

% \tcolorboxenvironment{theorem}{
%   boxrule=0pt,
%   boxsep=0pt,
%   colback={White!90!Dandelion},
%   enhanced jigsaw, 
%   borderline west={2pt}{0pt}{Dandelion},
%   sharp corners,
%   before skip=10pt,
%   after skip=10pt,
%   breakable,
% }

% \tcolorboxenvironment{lemma}{
%   boxrule=0pt,
%   boxsep=0pt,
%   blanker,
%   borderline west={2pt}{0pt}{Red},
%   before skip=10pt,
%   after skip=10pt,
%   sharp corners,
%   left=12pt,
%   right=12pt,
%   breakable,
% }

% \tcolorboxenvironment{corollary}{
%   boxrule=0pt,
%   boxsep=0pt,
%   blanker,
%   borderline west={2pt}{0pt}{Emerald},
%   before skip=10pt,
%   after skip=10pt,
%   sharp corners,
%   left=12pt,
%   right=12pt,
%   breakable,
% }

% \tcolorboxenvironment{proof}{
%   boxrule=0pt,
%   boxsep=0pt,
%   blanker,
%   borderline west={2pt}{0pt}{NavyBlue!80!white},
%   before skip=10pt,
%   after skip=10pt,
%   left=12pt,
%   right=12pt,
%   breakable,
% }

% \tcolorboxenvironment{remark}{
%   boxrule=0pt,
%   boxsep=0pt,
%   blanker,
%   borderline west={2pt}{0pt}{Green},
%   before skip=10pt,
%   after skip=10pt,
%   left=12pt,
%   right=12pt,
%   breakable,
% }

% \tcolorboxenvironment{note}{
%   boxrule=0pt,
%   boxsep=0pt,
%   blanker,
%   borderline west={2pt}{0pt}{Dark Green},
%   before skip=10pt,
%   after skip=10pt,
%   left=12pt,
%   right=12pt,
%   breakable,
% }


% \tcolorboxenvironment{example}{
%   boxrule=0pt,
%   boxsep=0pt,
%   blanker,
%   borderline west={2pt}{0pt}{Black},
%   sharp corners,
%   before skip=10pt,
%   after skip=10pt,
%   left=12pt,
%   right=12pt,
%   breakable,
% }

% \tcolorboxenvironment{proposition}{ boxrule=0pt, boxsep=0pt, colback={White!90!Yellow}, enhanced jigsaw, borderline west={2pt}{0pt}{Yellow}, sharp corners, before skip=10pt, after skip=10pt, breakable, }

%%%%%%%%%%%%%%%%%%%%%%%%%%%%%%%%%%%%%%%%%%%%%%%%%%%%5
%%%%%%% custom fonts%%%%%%%%%
\usepackage{crimson}
\usepackage[T1]{fontenc}
%%%%%%%%%%%%%%%%%%%%%%%%%%%%%


%layout full
\geometry{%
  a4paper,
  lmargin=2cm,
  rmargin=2.5cm,
  tmargin=3.5cm,
  bmargin=2.5cm,
  footskip=12pt,
  headheight=24pt}
% layout trim
% \geometry{
% papersize={379pt, 699pt},
% textwidth=345pt,
% textheight=596pt,
% left=17pt,
% top=54pt,
% right=17pt
% }
\pagestyle{fancy}
\theoremstyle{plain}

\theoremstyle{definition}
\newtheorem{theorem}{Theorem}[section]
\newtheorem{lemma}[theorem]{Lemma}
\newtheorem{proposition}[theorem]{Proposition}
\newtheorem{corollary}[theorem]{Corollary}
\newtheorem{problem}[theorem]{Problem}
\newtheorem*{claim}{Claim}
\newtheorem*{slemma}{Lemma}
\newtheorem*{sprop}{Proposition}

\newtheorem{inquestion}{Question}
\newenvironment{question}[1]
  {\renewcommand\theinquestion{#1}\inquestion}
  {\endinquestion}

\newtheorem{definition}{Definition}[section]
\newtheorem{conjecture}{Conjecture}[section]
\newtheorem{example}{Example}[section]
\newtheorem*{law}{Law}

\theoremstyle{remark}
\newtheorem*{remark}{Remark}
\newtheorem*{note}{Note}

\title{\textbf{\triposcourse{} Notes}}
\author{jt775}
