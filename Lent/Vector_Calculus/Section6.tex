\section{Maxwell's Equations}
\subsection{Brief introduction to electromagnetism}
\begin{definition}
    Denote by $ \mathbf{B} = \mathbf{B}(\mathbf{x},t) $ the \textbf{magnetic field} and by $ \mathbf{E}=\mathbf{E}(\mathbf{x},t) $ the \textbf{electric field}. These fields will depend on \textbf{charge density} $ \rho = \rho(\mathbf{x},t) $ (electric charge per unit volumn) and on \textbf{current density} $ \mathbf{J} = \mathbf{J}(\mathbf{x},t) $ (electric current per unit area).
\end{definition}
The \textbf{Maxwell's equations} are 
\begin{align}
    \div \mathbf{E} &= \frac{\rho}{\epsilon_0}\\ 
    \div\mathbf{B}&= 0\\ 
    \curl\mathbf{E}+\frac{\partial \mathbf{B}}{\partial t}&=\mathbf{0} \\ 
    \curl\mathbf{B}-\mu_0 \epsilon_0 \frac{\partial \mathbf{E}}{\partial t}&=\mu_0 \mathbf{J} 
\end{align}
The constants $ \epsilon_0,\mu_0 $ are the \textbf{permittivity} and \textbf{permeability} of free space, which obey 
\[
    \frac{1}{\mu_0 \epsilon_0}=c^2.
\]
Taking the divergence of (4) and use $ \div\curl\mathbf{B}=0 $ gives
\[
    0 = \mu_0 \epsilon_0 \frac{\partial }{\partial t}(\div\mathbf{E})+\mu_0 \div\mathbf{J} \Longrightarrow \frac{\partial \rho}{\partial t}+\div\mathbf{J}=0,  
\]
which is the \textit{conservation of charge}. By Noether's theorem there is a symmetry to it, which turned out to be ``gauge symmetry''.

\subsection{Integral formulations}
\subsubsection*{Integration of (1)}
By divergence theorem, we get the Gauss’ law for electric fields
\[
    \int_{V} \div\mathbf{E} \,\mathrm{d}V  =  \int_{V} \frac{\rho}{\epsilon_0} \,\mathrm{d}V 
    \Longrightarrow  \int_{\partial V} \mathbf{E} \cdot\mathrm{d}\mathbf{S} =\frac{1}{\epsilon_0} \int_{V} \rho \,\mathrm{d}V = \frac{Q}{\epsilon_0} .
\]
\subsubsection*{Integration of (2)}
\[
    \int_{\partial V} \mathbf{B} \cdot\mathrm{d}\mathbf{S} = 0.
\]
There is no net magnetic flux over any closed surface $ \partial V $. That is to say, there are no magnetic monopoles
\subsubsection*{Integration of (3)}
Integrate over a surface $S$ and use Stokes theorem
\[
    \oint_{\partial S} \mathbf{E}\cdot\mathrm{d}\mathbf{x} = - \int_{S} \frac{\partial \mathbf{B}}{\partial t}  \cdot\mathrm{d}\mathbf{S} = -\frac{\mathrm{d}}{\mathrm{d}t} \int_{S} \mathbf{B} \cdot\mathrm{d}\mathbf{S} .
\]
So the rate of change of magnetic flux across the surface $S$ produces a circulation of the electric field $\mathbf{E}$ about $\partial S$. In other words, a changing magnetic field will induce a current.

\subsubsection*{Integration of (4)}
Integrate over $S$ and use Stokes' theorem 
\[
    \oint_{\partial S} \mathbf{B} \cdot\mathrm{d}\mathbf{x} = \mu \cdot \int_{S} \mathbf{J} \cdot\mathrm{d}\mathbf{S} + \mu_0 \epsilon_0 \frac{\mathrm{d}}{\mathrm{d}t}\int_{S} \mathbf{E} \cdot\mathrm{d}\mathbf{S} .
\]
We see that electric current produces circulation of the magnetic field about the axis of direction of the current. So if current flows along a wire, a magnetic field will be produced that circulated the wire in the positive sense.

\subsection{Electromagnetic waves}
In \textit{empty space}, $ \rho=0,\mathbf{J}=\mathbf{0} $, so the Maxwell's equations become
\begin{align*}
    \div \mathbf{E} &= 0\\ 
    \div\mathbf{B}&= 0\\ 
    \curl\mathbf{E}+\frac{\partial \mathbf{B}}{\partial t}&=\mathbf{0} \\ 
    \curl\mathbf{B}-\mu_0 \epsilon_0 \frac{\partial \mathbf{E}}{\partial t}&=\mathbf{0}
\end{align*}

Recall that the Laplacian of a vector field is defined by
\[
    \nabla^{2} \mathbf{E}=\nabla(\nabla \cdot \mathbf{E})-\nabla \times(\nabla \times \mathbf{E}).
\]
So from Maxwell's equations (1), (3) and (4) we get
\[
    \nabla^{2} \mathbf{E}=-\nabla \times(\nabla \times \mathbf{E})=\nabla \times \frac{\partial \mathbf{B}}{\partial t}=\frac{\partial}{\partial t}(\nabla \times \mathbf{B})=\mu_{0} \epsilon_{0} \frac{\partial^{2} \mathbf{E}}{\partial t^{2}}.
\]
If we set $\mu_{0} \epsilon_{0}=1 / c^{2},$ we get the \textit{wave equation} for the electric field
\[
    \boxed{\nabla^{2} \mathbf{E}-\frac{1}{c^{2}} \frac{\partial^{2} \mathbf{E}}{\partial t^{2}}=0}
\]
This tells us that, in a vacuum, the electric field travels at the speed of light. Similarly, using Maxwell's equations (2), (3) and (4) we find
\[
    \nabla^{2} \mathbf{B}=-\nabla \times(\nabla \times \mathbf{B})=-\mu_{0} \epsilon_{0} \nabla \times \frac{\partial \mathbf{E}}{\partial t}=-\mu_{0} \epsilon_{0} \frac{\partial}{\partial t}(\nabla \times \mathbf{E})=\mu_{0} \epsilon_{0} \frac{\partial^{2} \mathbf{B}}{\partial t^{2}}.
\]
So again we get the wave equation, but this time for the magnetic field
\[
    \boxed{\nabla^{2} \mathbf{B}-\frac{1}{c^{2}} \frac{\partial^{2} \mathbf{B}}{\partial t^{2}}=0}
\]

\subsection{Electrostatics and magnetostatics}
Suppose all fields and source terms are independent of $t$. Then Maxwell's equations are decoupled into 
\[
    (\text{A})\left\{ \begin{aligned}
         &\div\mathbf{E} = \frac{\rho}{\epsilon_0},\\ 
         &\curl\mathbf{E} = \mathbf{0},
    \end{aligned} \right. ~~~~~~~~ (\text{B})\left\{ \begin{aligned}
        &\div\mathbf{B} = 0,\\[3pt]
        &\curl\mathbf{B} = \mu_0\mathbf{J}.
   \end{aligned} \right.
\]
If we are working on $ \mathbb{R}^{3} $ which is 2-connected, then $ \curl\mathbf{E} = \mathbf{0},\div\mathbf{B}=0 $ implies 
\[
    \mathbf{E} = - \nabla\phi\quad \text{and}\quad \mathbf{B} = \curl\mathbf{A}.
\]
Here $ \phi $ is called the \textbf{electric potential} for $\mathbf{E}$ and $ \mathbf{A} $ is called the \textbf{magnetic potential} for $ \mathbf{B} $. Then (A), (B) become
\[
    -\nabla^{2} \phi=\frac{\rho}{\epsilon_{0}}, \quad \nabla \times(\nabla \times \mathbf{A})=\mu_{0} \mathbf{J}.
\]
The first is called \textbf{Poisson's equation}. See Section 7.

\subsection{Gauge invariance}
Let us return to the case of the magnetic potential. Since we always have from (2) that
$$
    \nabla \cdot \mathbf{B}=0
$$
we can always write $\mathbf{B}=\nabla \times \mathbf{A}$ for some vector potential $\mathbf{A}$, assuming we work on all of $\mathbf{R}^{3}$. Important observation: this definition does not define $\mathbf{A}$ uniquely, because we can always change $ \mathbf{A} \mapsto \mathbf{A} + \nabla_\chi $, where $\chi$ is a scalar function. Doing so does not change the value of $\mathbf{B}=\nabla \times \mathbf{A}$ since $\nabla \times \nabla \chi=\mathbf{0}$. This is called \textbf{gauge invariance}. We have a certain amount of freedom in how we define $\mathbf{A}$. This invariance encapsulates the precise symmetry that generates conservation of charge, via Noether's theorem. Using $\mathbf{B}=\nabla \times \mathbf{A}$ in (3) we see
$$
    \nabla \times\left(\mathbf{E}+\frac{\partial \mathbf{A}}{\partial t}\right)=\mathbf{0}.
$$
This implies the existence of a scalar potential $\phi$ such that
$$
    \mathbf{E}=-\nabla \phi-\frac{\partial \mathbf{A}}{\partial t}.
$$
So we can reduce Maxwell's equations down to
$$
    (1) \Longrightarrow -\nabla^{2} \phi-\frac{\partial}{\partial t}(\nabla \cdot \mathbf{A})=\frac{\rho}{\varepsilon_{0}},
$$
and
$$
    (4) \Longrightarrow \nabla \times(\nabla \times \mathbf{A})+\mu_{0} \varepsilon_{0} \nabla\left(\frac{\partial \phi}{\partial t}\right)+\mu_{0} \varepsilon_{0} \frac{\partial^{2} \mathbf{A}}{\partial t^{2}}=\mu_{0} \mathbf{J}.
$$
Recalling that $\nabla \times(\nabla \times \mathbf{A})=\nabla(\nabla \cdot \mathbf{A})-\nabla^{2} \mathbf{A}$ and $\mu_{0} \varepsilon_{0}=1 / c^{2}$ we find that the second
of these equations can be rewritten in the form
$$
    -\left(\nabla^{2} \mathbf{A}-\frac{1}{c^{2}} \frac{\partial^{2} \mathbf{A}}{\partial t^{2}}\right)+\nabla\left(\nabla \cdot \mathbf{A}+\frac{1}{c^{2}} \frac{\partial \phi}{\partial t}\right)=\mu_{0} \mathbf{J}.
$$
Now we can exploit gauge freedom: without loss of generality we can assume that
\[
    \nabla \cdot \mathbf{A}+\frac{1}{c^{2}} \frac{\partial \phi}{\partial t}=0
\]
Indeed, by changing $\mathbf{A}$ to $\mathbf{A}+\nabla \chi$ we can make sure this term vanishes, using a suitable function $\chi$. Maxwell's equations are then reduced to
\[
    -\nabla^{2} \phi+\frac{1}{c^{2}} \frac{\partial^{2} \phi}{\partial t^{2}}=\frac{\rho}{\varepsilon_{0}} \quad \text { and } \quad-\nabla^{2} \mathbf{A}+\frac{1}{c^{2}} \frac{\partial^{2} \mathbf{A}}{\partial t^{2}}=\mu_{0} \mathbf{J}.
\]
These are called Maxwell's equations in Lorenz gauge. The electric and magnetic fields can be recovered from the formulas
\[
    \mathbf{E}=-\nabla \phi-\frac{\partial \mathbf{A}}{\partial t} \quad \text { and } \quad \mathbf{B}=\nabla \times \mathbf{A}.
\]
