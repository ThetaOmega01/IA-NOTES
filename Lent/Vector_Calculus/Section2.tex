\section{Coordinates, Differentials, Gradients}
\subsection{Differentials and first order changes}

Recall that for $ f(u_1,\dots,u_n) $ we define the \textbf{differential} of $f$ to be 
\[
    \rmd f = \frac{\partial f}{\partial u_i} \dd u_i, 
\]
assuming summation convention. Call $ \{\rmd u_i\} $ \textbf{differential forms}. They are defined to be \textit{linearly independent}, i.e. $ \alpha_i \dd u_i =0 \Rightarrow \alpha_i=0 $.
\begin{definition}[Differential of vectors]
    Let $\bfx = \bfx(u_1,\dots,u_n)$. Define
    \[
        \rmd \bfx= \frac{\partial \bfx}{\partial u_i} \dd u_i. 
    \]
\end{definition}

\begin{example}
    If $f(u,v,w)=u^2+w\sin v$, then $ \rmd f = 2u\dd u+w \cos v \dd v+ \sin v \dd w $. If 
    \[
        \mathbf{x}(u,v,w)=\begin{pmatrix}
            u^2-v^2 \\ w \\ e^v
        \end{pmatrix},
    \] 
    then 
    \[
        \rmd \mathbf{x} = \begin{pmatrix}
            2u \\ 0 \\ 0
        \end{pmatrix}\dd u+ \begin{pmatrix}
            -2v \\ 0 \\ e^v
        \end{pmatrix}\dd v+\begin{pmatrix}
            0 \\ 1 \\ 0
        \end{pmatrix}\dd w.
    \]
\end{example}

\subsubsection{Differentials and the chain rule}
Let $ f: \mathbb{R}^{n}\to \mathbb{R} $, $\bfu=(u_1,\dots,u_n)$ and $ \delta\bfu=(\delta u_1,\dots,\delta u_n) $, then we have 
\[
    \delta f=f(\bfu+\delta\bfu)-f(\bfu)=\frac{\partial f}{\partial u_i}\delta u_i +o(\delta\bfu),
\]
where $ \frac{o(\delta\bfu)}{|\delta\bfu|}\to 0 $ as $ |\delta\bfu|\to 0 $.

Similarly for vector fields
\[
    \delta\bfx = \frac{\partial \bfx}{\partial u_i}\delta u_i+o(\delta\bfu). 
\]

Now consider a function $ F(u_1,\dots,u_n)=f(x_1,\dots,x_n) $, where $ x_i=x_i(u_1,\dots,u_n) $ are functions of $u_i$. Clearly $ \rmd F=\rmd f $, so 
\[
    \frac{\partial F}{\partial u_i} \dd u_i=\rmd F = \rmd f= \frac{\partial f}{\partial x_j}\dd x_j = \frac{\partial f}{\partial x_j}\frac{\partial x_j}{\partial u_i}\dd u_i.  
\]
Since $\rmd u_i$ are a basis, we have the \textbf{chain rule}
\[
    \frac{\partial F}{\partial u_i}=\frac{\partial f}{\partial x_j}\frac{\partial x_j}{\partial u_i}.
\]

\subsection{Coordinates and line elements}
Recall that the Cartesian coordinates $ (x,y) $ and the polar coordinates $ (r,\theta) $ are related by an invertible(not at $O$ but that does not cause problems) relation
\[
    x=r \cos \theta,\quad y= r\sin \theta.
\]
A general set of coordinates $ (u,v) $ on $\bbR^2$ can be specified by their relationship to $(x,y)$, i.e. specify smooth functions
\[
    x=x(u,v),\quad y=y(u,v),
\]
and inverted smooth functions 
\[
    u=u(x,y),\quad v=v(x,y).
\]

Similarly for $ \mathbb{R}^{3} $, $(u,v,w)$ is specified by 
\[
    x=x(u,v,w),\quad y=y(u,v,w),\quad z=z(u,v,w).
\]

\begin{example}[Cartesian coordinates]
    $ \bfx(x,y)=\begin{pmatrix}
        x \\ y
    \end{pmatrix}=x\bfe_x+y\bfe_y $ where $ \{\bfe_x,\bfe_y\} $ are orthonormal and \textit{fixed}. Said differently,
    \[
        \bfe_x = \frac{\frac{\partial }{\partial x}\bfx(x,y) }{\left| \frac{\partial }{\partial x}\bfx(x,y) \right| },\quad \bfe_y=\frac{\frac{\partial }{\partial y}\bfx(x,y) }{\left| \frac{\partial }{\partial y}\bfx(x,y) \right| }.
    \]
\end{example}

Note that 
\[
    \rmd \bfx=\frac{\partial \bfx}{\partial x}\dd x+\frac{\partial \bfx}{\partial y}\dd y=\rmd x\bfe_x+\rmd y\bfe_y,  
\]
i.e. change in coordinates $ x \mapsto x+\delta x $, then the vector changes(to the first order) by $ \bfx \mapsto \bfx+\delta x\bfe_x .$

\begin{definition}[Line element]
    Call $ \rmd \bfx $ the \textbf{line element}. This tells us how small changes in coordinates produce changes in position vectors.
\end{definition}

\begin{example}[Polar coordinates]
    $ \mathbf{x}(r,\theta)=\begin{pmatrix}
        r\cos \theta \\ r\sin\theta
    \end{pmatrix}\equiv r\bfe_r $. We use basis vectors $ \bfe_r,\bfe_\theta $ such that 
    \[
        \bfe_r=\begin{pmatrix}
            \cos\theta \\ \sin\theta
        \end{pmatrix},\quad \bfe_\theta = \begin{pmatrix}
            -\sin \theta  \\\cos \theta 
        \end{pmatrix}.
    \]
    Note that $ \{\bfe_r,\bfe_\theta\} $ are orthonormal at each $ (r,\theta) $, but \textit{not} the same for each $ (r,\theta) $. As before,
    \[
        \bfe_r = \frac{\frac{\partial }{\partial r}\bfx(r,\theta) }{\left| \frac{\partial }{\partial r}\bfx(r,\theta) \right| },\quad \bfe_\theta=\frac{\frac{\partial }{\partial \theta}\bfx(r,\theta) }{\left| \frac{\partial }{\partial \theta}\bfx(r,\theta) \right| }
    \]
    Since $ \{\bfe_r,\bfe_\theta\} $ are orthogonal, we call $ (r,\theta) $ \textbf{orthogonal curvilinear coordinates}. The line element is 
    \begin{align*}
        \rmd \bfx&= \frac{\partial \bfx}{\partial r}\dd r+\frac{\partial \bfx}{\partial \theta}\dd \theta = \frac{\partial r\bfe_r}{\partial r}\dd r+\frac{\partial r\bfe_r}{\partial \theta}\dd \theta\\ 
        &= \bfe_r \dd r+ r\bfe_\theta\dd \theta.
    \end{align*}
    A change $ \theta \mapsto \theta+ \delta \theta $ produces a (first order) change $ \mathbf{x}\mapsto \mathbf{x}+r\delta \theta\bfe_\theta  $.
\end{example}

\subsubsection{Orthogonal curvilinear coordinates}
In general we say that $(u, v, w)$ are a set of \textbf{orthogonal curvilinear coordinates} if the vectors
\[
    \bfe_u = \frac{\partial \bfx/\partial u }{\left| \partial \bfx/\partial u \right| },\quad \bfe_v=\frac{\partial \bfx/\partial v }{\left| \partial \bfx/\partial v \right| },\quad \bfe_w=\frac{\partial \bfx/\partial w }{\left| \partial \bfx/\partial w \right| }
\]
form a \textit{right-handed orthonormal} basis ($ \bfe_u\times\bfe_v=\bfe_w $ etc.). Note that $ \bfe_u,\bfe_v,\bfe_w $ need \textit{not} be fixed.

It is standard to write 
\[
    h_u = \left| \frac{\partial \bfx}{\partial u}  \right| ,\quad h_v=\left| \frac{\partial \bfx}{\partial v}  \right|,\quad h_w=\left| \frac{\partial \bfx}{\partial w}  \right|
\]
called \textbf{scale factors}. They are called so because 
\[
    \rmd x = \frac{\partial \bfx}{\partial u}\dd u+ \frac{\partial \bfx}{\partial v}\dd v+\frac{\partial \bfx}{\partial w}\dd w=h_u \dd u \bfe_u+h_v \dd v \bfe_v+h_w \dd w \bfe_w.
\]
They describe how small changes in $ (u,v,w) $ scale up (to 1st order) to changes in $\bfx$ wrt $ \bfe_u,\bfe_v,\bfe_w $.

\subsubsection{Cylindrical polar coordinates}
Cylindrical polar coordinates $ (\rho,\phi,z) $ are defined by 
\[
    \bfx(\rho,\phi,z)= \begin{pmatrix}
        \rho\cos \phi \\ \rho \sin \phi \\ z
    \end{pmatrix}, \quad\begin{matrix}
        0\le \rho< \infty, \\ 0\le \phi<2\pi,\\ -\infty <z<\infty,
    \end{matrix}
\]
with 
\[
    \bfe_\rho=\begin{pmatrix}
        \cos \phi \\ \sin \phi \\ 0
    \end{pmatrix},\quad
    \bfe_\phi=\begin{pmatrix}
        -\sin\phi \\ \cos \phi \\ 0
    \end{pmatrix},\quad
    \bfe_z=\begin{pmatrix}
        0 \\ 0 \\ 1
    \end{pmatrix},
\]
scale factors
\[
    h_\rho=1,\quad,h_\phi=\rho,\quad h_z=1,
\]
and line element
\[
    \rmd \bfx = \rmd \rho\, \bfe_\rho+\rho\dd \phi \, \bfe_\phi+\rmd z \, \bfe_z.
\]
\begin{note}
    \[
        \bfx=\begin{pmatrix}
            \rho\cos \phi \\ \rho \sin \phi \\ z
        \end{pmatrix}=\rho\begin{pmatrix}
            \cos \phi \\  \sin \phi \\ 0
        \end{pmatrix}+z \begin{pmatrix}
            0 \\ 0 \\ 1
        \end{pmatrix}=\rho\, \bfe_\rho+z\, \bfe_z.
    \]
\end{note}

See \href{http://jt775.user.srcf.net/IA-Lent/handouts/vc_handout1.pdf}{handout 1} for detailed diagrams.

\subsubsection{Spherical polar coordinates}
Define $ (r,\theta,\phi) $ by 
\[
    \bfx(r,\theta,\phi) = \begin{pmatrix}
        r \cos \phi \sin \theta \\ r \sin \phi \sin \theta \\ r \cos \theta 
    \end{pmatrix},\quad \begin{matrix}
        0\le r\le \infty, \\ 0\le \theta<\pi,\\ 0 \le \phi< 2\pi,
    \end{matrix}
\]
with 
\[
    \bfe_r = \begin{pmatrix}
        \cos \phi \sin \theta \\ \sin \phi \sin \theta\\ \cos \theta 
    \end{pmatrix},\quad \bfe_\theta = \begin{pmatrix}
        \cos \phi \cos \theta \\ \sin \phi \cos \theta \\ -\sin \theta 
    \end{pmatrix},\quad \bfe_\phi = \begin{pmatrix}
        - \sin \phi \\ \cos \phi \\ 0
    \end{pmatrix},
\]
and scale factors $ h_r=1,h_\theta=r,h_\phi=r \sin \theta $. Line element is 
\[
    \rmd \bfx = \rmd r\,\bfe_r+ r\dd \theta\,\bfe_\theta+r \sin \theta\dd \phi\,\bfe_\phi,
\]
and $ \bfx=r\bfe_r $. See \href{http://jt775.user.srcf.net/IA-Lent/handouts/vc_handout1.pdf}{handout 1} for detailed diagrams.

\subsection{The gradient operator}
\begin{definition}[Gradient]
    For $ f: \mathbb{R}^{3}\to \mathbb{R} $, define \textbf{gradient} of $f$ by 
    \begin{equation}\label{eq:gradient}\tag{$ * $}
        \nabla f : f(\bfx+\bfh) = f(\bfx)+ \nabla f(\bfx) \cdot \bfh+o(\bfh)\quad (|\bfh|\to 0).
    \end{equation}
\end{definition}
\begin{definition}[Directional derivative]
    The \textbf{directional derivative} of $f$ in direction $\bfv$, denoted by $ D_\bfv f $ or $ \partial f/\partial \bfv  $ is defined by 
    \[
        D_\bfv f(\bfx) = \lim_{t \to 0} \frac{f(\bfx+t\bfv)-f(\bfv)}{t}, 
    \]
    i.e.
    \begin{equation}\label{eq:direc deriv}\tag{$ ** $}
        f(\bfx+t\bfv)=f(\bfx)+t D_\bfv f(\bfx)+o(t),\quad t\to 0. 
    \end{equation}
\end{definition}

Set $ \bfh=t\bfv $ in (\ref{eq:gradient}), we get 
\[
    f(\bfx+t\bfv) = f(\bfx)+t \nabla f(\bfx) \cdot \bfv + o(t),\quad t\to 0.
\]
Comparing to (\ref{eq:direc deriv}), we get 
\[
    D_\bfv f = \bfv \cdot \nabla f.
\]
By Cauchy-Schwarz, $ \nabla f $ points in direction of greatest increase of $f$. Similarly $ - \nabla f $ points in direction of greatest decrease of $f$.

\begin{example}
    Suppose $ f(\bfx) = \frac{1}{2}|\bfx|^2 $. Then 
    \begin{align*}
        f(\bfx+\bfh) &= \frac{1}{2}(\bfx+\bfh)\cdot (\bfx+\bfh)\\ 
        &= \frac{1}{2}|\bfx|^2+ \bfx \cdot \bfh+\frac{1}{2}|\bfh|^2\\ 
        &= \frac{1}{2}|\bfx|^2+\bfx \cdot \bfh+o(\bfh),\quad |\bfh|\to 0,
    \end{align*}
    so $ \nabla f =\bfx $.
\end{example}

\subsubsection{Gradient along a curve}
Suppose we have a curve $ t \mapsto \bfx(t) $. How does $f$ change as we move along this curve?

Write $ F(t) = f(\bfx(t)) $, $ \delta\bfx = \bfx(t+\delta t)-\bfx(t) $.
\begin{align*}
    F(t+\delta t) &= f(\bfx(t+\delta t)) = f(\bfx(t)+\delta \bfx)\\ 
    &= f(\bfx(t))+\nabla f(\bfx(t)) \cdot \delta\bfx +o(\delta \bfx),\quad |\delta\bfx| \to 0.
\end{align*}
Since $ \delta\bfx = \delta t\, \bfx'(t)+o(\delta t) $, 
\begin{align*}
    &F(t+\delta t)= F(t)+\bfx'(t) \cdot \nabla f(\bfx(t)) \delta t+o(\delta t)\\ 
    \Longrightarrow & \boxed{\frac{\mathrm{d}F}{\mathrm{d}t} = \frac{\mathrm{d}}{\mathrm{d}t}f(\bfx(t)) = \frac{\mathrm{d}\bfx}{\mathrm{d}t} \cdot \nabla f(\bfx(t))}.
\end{align*}

\subsubsection{Gradient as the normal of a surface}
Suppose a surface $S$ is defined implicitly by 
\[
    S = \{\bfx\in \mathbb{R}^{3}: f(\bfx)=0\}.
\]
If $ t \mapsto \bfx(t) $ is \textit{any} curve \textit{in} $S$, then $ f(\bfx(t))=0 $ and 
\[
    0 = \frac{\mathrm{d}}{\mathrm{d}t}f(\bfx(t)) =  \frac{\mathrm{d}\bfx}{\mathrm{d}t} \cdot \nabla f(\bfx(t)),
\]
so $ \nabla f $ is \textit{orthogonal} to tangent vector of \textit{any curve} in $S$, and thus is \textit{normal} to $S$ at $ \bfx $.

\subsection{Computing the gradient}
\subsubsection{In Cartesian coordinates}
To get $ \bfx \mapsto \bfx+\bfh $, simply let $ x \mapsto x+h_1 $ etc. and 
\begin{align*}
    f(\bfx+\bfh) &= f(x+h_1,y+h_2,z+h_3)\\ 
    &= f(\bfx) + \frac{\partial f}{\partial x}h_1+\frac{\partial f}{\partial y}h_2+\frac{\partial f}{\partial z}h_3 +o(\bfh)\\ 
    &= f(\bfx)+ \begin{pmatrix}
        \partial f/\partial x  \\ \partial f/\partial y  \\ \partial f/\partial z 
    \end{pmatrix}\cdot \bfh+o(\bfh),
\end{align*}
so 
\[
    \nabla f = \begin{pmatrix}
        \partial f/\partial x  \\ \partial f/\partial y  \\ \partial f/\partial z 
    \end{pmatrix} \Longleftrightarrow \nabla f = \bfe_i \frac{\partial f}{\partial x_i}.
\]
We see that $ \nabla $ is a ``vector differential operator'' and in Cartesian coordinates
\[
    \nabla = \bfe_i \frac{\partial }{\partial x_i} . 
\]

\begin{example}
    Let $ f=\frac{1}{2}|\bfx|^2 = \frac{1}{2}(x^2+y^2+z^2) $. Then $ \nabla f = (x,y,z) $.
\end{example}

\subsubsection{In orthogonal curvilinear coordinates}
Recall in Cartesian coordinates $ \rmd \bfx = \rmd x_i \bfe_i $ and $ \rmd f = (\partial f/\partial x_i ) x_i $. Notice that
\[
    \nabla f \cdot \rmd \bfx = \left( \bfe_i \frac{\partial f}{\partial x_i}  \right) \cdot \left( \bfe_j \dd x_j \right) = \frac{\partial f}{\partial x_i}(\bfe_i\cdot \bfe_j) \dd x_j = \frac{\partial f}{\partial x_i}\dd x_i = \rmd f.  
\]
Thus in \textit{any} coordinates, 
\[
    \boxed{\nabla f\cdot \rmd \bfx = \rmd f}
\] 

\begin{proposition}\label{prop:2.1}
    If $ (u,v,w) $ is OCC\footnote{orthogonal curvilinear coordinates} and $f=f(u,v,w)$, then 
    \[
        \nabla f = \frac{1}{h_u}\frac{\partial f}{\partial u}\bfe_u + \frac{1}{h_v}\frac{\partial f}{\partial v}\bfe_v + \frac{1}{h_w}\frac{\partial f}{\partial w}\bfe_w.
    \] 
\end{proposition}
\begin{proof}
    If $ f=f(u,v,w) $ and $ \bfx = \bfx(u,v,w) $, then 
    \[
        \rmd f = \frac{\partial f}{\partial u}\dd u +\frac{\partial f}{\partial v}\dd v +\frac{\partial f}{\partial w}\dd w,\ \rmd \bfx = h_u\dd u\bfe_u+h_v\dd v\bfe_v+h_w\dd w\bfe_w.
    \]
    Using $\nabla f\cdot \rmd \bfx = \rmd f$ and writing $ \nabla f = (\nabla f)_u\bfe_u+(\nabla f)_v\bfe_v+(\nabla f)_w\bfe_w $, we get 
    \[
        \frac{\partial f}{\partial u}\dd u +\frac{\partial f}{\partial v}\dd v +\frac{\partial f}{\partial w}\dd w = h_u(\nabla f)_u \dd u+h_v(\nabla f)_v \dd v+h_w(\nabla f)_w \dd w.
    \]
    Since $ \rmd u, \rmd v,\rmd w $ are linearly independent, comparing coefficients gives the result.
\end{proof}

Hence in cylindrical polars,
\[
    \nabla f = \frac{\partial f}{\partial \rho}\bfe_\rho+\frac{1}{\rho} \frac{\partial f}{\partial \phi}\bfe_\phi+\frac{\partial f}{\partial z}\bfe_z,  
\]
in spherical polars,
\[
    \nabla f = \frac{\partial f}{\partial r}\bfe_r+\frac{1}{r}\frac{\partial f}{\partial \theta}\bfe_\theta+\frac{1}{r\sin\theta}\frac{\partial f}{\partial \phi}\bfe_\phi.   
\]
\begin{example}
    Let $ f(\bfx)= \frac{1}{2}|\bfx|^2 $, then 
    \[
        f = \begin{cases}
        \frac{1}{2}(x^2+y^2+z^2) &\text{Cartesian}\\
        \frac{1}{2}(\rho^2+z^2) &\text{Cylindrical}\\
        \frac{1}{2}r^2 &\text{Spherical}
        \end{cases} 
    \]
    Can check $ \nabla f $ gives correct results.
\end{example}
Check \href{https://www.vle.cam.ac.uk/pluginfile.php/19798882/mod_resource/content/2/vc_notes1.pdf#page=17}{this page} for summary.