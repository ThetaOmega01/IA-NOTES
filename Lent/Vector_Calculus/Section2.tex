\section{Coordinates, Differentials, Gradients}
\subsection{Differentials and first order changes}

Recall that for $ f(u_1,\dots,u_n) $ we define the \textbf{differential} of $f$ to be 
\[
    \mathrm{d} f = \frac{\partial f}{\partial u_i} \dd u_i, 
\]
assuming summation convention. Call $ \{\mathrm{d} u_i\} $ \textbf{differential forms}. They are defined to be \textit{linearly independent}, i.e. $ \alpha_i \dd u_i =0 \Rightarrow \alpha_i=0 $.
\begin{definition}[Differential of vectors]
    Let $\mathbf{x} = \mathbf{x}(u_1,\dots,u_n)$. Define
    \[
        \mathrm{d} \mathbf{x}= \frac{\partial \mathbf{x}}{\partial u_i} \dd u_i. 
    \]
\end{definition}

\begin{example}
    If $f(u,v,w)=u^2+w\sin v$, then $ \mathrm{d} f = 2u\dd u+w \cos v \dd v+ \sin v \dd w $. If 
    \[
        \mathbf{x}(u,v,w)=\begin{pmatrix}
            u^2-v^2 \\ w \\ e^v
        \end{pmatrix},
    \] 
    then 
    \[
        \mathrm{d} \mathbf{x} = \begin{pmatrix}
            2u \\ 0 \\ 0
        \end{pmatrix}\dd u+ \begin{pmatrix}
            -2v \\ 0 \\ e^v
        \end{pmatrix}\dd v+\begin{pmatrix}
            0 \\ 1 \\ 0
        \end{pmatrix}\dd w.
    \]
\end{example}

\subsubsection*{Differentials and the chain rule}
Let $ f: \mathbb{R}^{n}\to \mathbb{R} $, $\mathbf{u}=(u_1,\dots,u_n)$ and $ \delta\mathbf{u}=(\delta u_1,\dots,\delta u_n) $, then we have 
\[
    \delta f=f(\mathbf{u}+\delta\mathbf{u})-f(\mathbf{u})=\frac{\partial f}{\partial u_i}\delta u_i +o(\delta\mathbf{u}),
\]
where $ \frac{o(\delta\mathbf{u})}{|\delta\mathbf{u}|}\to 0 $ as $ |\delta\mathbf{u}|\to 0 $.

Similarly for vector fields
\[
    \delta\mathbf{x} = \frac{\partial \mathbf{x}}{\partial u_i}\delta u_i+o(\delta\mathbf{u}). 
\]

Now consider a function $ F(u_1,\dots,u_n)=f(x_1,\dots,x_n) $, where $ x_i(u_1,\dots,u_n) $ are functions of $u_i$. Clearly $ \mathrm{d} F=\mathrm{d} f $, so 
\[
    \frac{\partial F}{\partial u_i} \dd u_i=\mathrm{d} F = \mathrm{d} f= \frac{\partial f}{\partial x_j}\dd x_j = \frac{\partial f}{\partial x_j}\frac{\partial x_j}{\partial u_i}\dd u_i.  
\]
Since $\mathrm{d} u_i$ are a basis, we have the \textbf{chain rule}
\[
    \frac{\partial F}{\partial u_i}=\frac{\partial f}{\partial x_j}\frac{\partial x_j}{\partial u_i}.
\]

\subsection{Coordinates and line elements}
Recall that the Cartesian coordinates $ (x,y) $ and the polar coordinates $ (r,\theta) $ are related by an invertible(not at $O$ but that does not cause problems) relation
\[
    x=r \cos \theta,\quad y= r\sin \theta.
\]
A general set of coordinates $ (u,v) $ on $\mathbb{R}^2$ can be specified by their relationship to $(x,y)$, i.e. specify smooth functions
\[
    x=x(u,v),\quad y=y(u,v),
\]
and inverted smooth functions 
\[
    u=u(x,y),\quad v=v(x,y).
\]

Similarly for $ \mathbb{R}^{3} $, $(u,v,w)$ is specified by 
\[
    x=x(u,v,w),\quad y=y(u,v,w),\quad z=z(u,v,w).
\]

\begin{example}[Cartesian coordinates]
    $ \mathbf{x}(x,y)=\begin{pmatrix}
        x \\ y
    \end{pmatrix}=x\mathbf{e}_x+y\mathbf{e}_y $ where $ \{\mathbf{e}_x,\mathbf{e}_y\} $ are orthonormal and \textit{fixed}. Said differently,
    \[
        \mathbf{e}_x = \frac{\frac{\partial }{\partial x}\mathbf{x}(x,y) }{\left| \frac{\partial }{\partial x}\mathbf{x}(x,y) \right| },\quad \mathbf{e}_y=\frac{\frac{\partial }{\partial y}\mathbf{x}(x,y) }{\left| \frac{\partial }{\partial y}\mathbf{x}(x,y) \right| }.
    \]
\end{example}

Note that 
\[
    \mathrm{d} \mathbf{x}=\frac{\partial \mathbf{x}}{\partial x}\dd x+\frac{\partial \mathbf{x}}{\partial y}\dd y=\mathrm{d} x\mathbf{e}_x+\mathrm{d} y\mathbf{e}_y,  
\]
i.e. change in coordinates $ x \mapsto x+\delta x $, then the vector changes(to the first order) by $ \mathbf{x} \mapsto \mathbf{x}+\delta x\mathbf{e}_x .$

\begin{definition}[Line element]
    Call $ \mathrm{d} \mathbf{x} $ the \textbf{line element}. This tells us how small changes in coordinates produce changes in position vectors.
\end{definition}

\begin{example}[Polar coordinates]
    $ \mathbf{x}(r,\theta)=\begin{pmatrix}
        r\cos \theta \\ r\sin\theta
    \end{pmatrix}\equiv r\mathbf{e}_r $. We use basis vectors $ \mathbf{e}_r,\mathbf{e}_\theta $ such that 
    \[
        \mathbf{e}_r=\begin{pmatrix}
            \cos\theta \\ \sin\theta
        \end{pmatrix},\quad \mathbf{e}_\theta = \begin{pmatrix}
            -\sin \theta  \\\cos \theta 
        \end{pmatrix}.
    \]
    Note that $ \{\mathbf{e}_r,\mathbf{e}_\theta\} $ are orthonormal at each $ (r,\theta) $, but \textit{not} the same for each $ (r,\theta) $. As before,
    \[
        \mathbf{e}_r = \frac{\frac{\partial }{\partial r}\mathbf{x}(r,\theta) }{\left| \frac{\partial }{\partial r}\mathbf{x}(r,\theta) \right| },\quad \mathbf{e}_\theta=\frac{\frac{\partial }{\partial \theta}\mathbf{x}(r,\theta) }{\left| \frac{\partial }{\partial \theta}\mathbf{x}(r,\theta) \right| }
    \]
    Since $ \{\mathbf{e}_r,\mathbf{e}_\theta\} $ are orthogonal, we call $ (r,\theta) $ \textbf{orthogonal curvilinear coordinates}. The line element is 
    \begin{align*}
        \mathrm{d} \mathbf{x}&= \frac{\partial \mathbf{x}}{\partial r}\dd r+\frac{\partial \mathbf{x}}{\partial \theta}\dd \theta = \frac{\partial r\mathbf{e}_r}{\partial r}\dd r+\frac{\partial r\mathbf{e}_r}{\partial \theta}\dd \theta\\ 
        &= \mathbf{e}_r \dd r+ r\mathbf{e}_\theta\dd \theta.
    \end{align*}
    A change $ \theta \mapsto \theta+ \delta \theta $ produces a (first order) change $ \mathbf{x}\mapsto \mathbf{x}+r\delta \theta\mathbf{e}_\theta  $.
\end{example}

\subsubsection*{Orthogonal curvilinear coordinates}
In general we say that $(u, v, w)$ are a set of \textbf{orthogonal curvilinear coordinates} if the vectors
\[
    \mathbf{e}_u = \frac{\partial \mathbf{x}/\partial u }{\left| \partial \mathbf{x}/\partial u \right| },\quad \mathbf{e}_v=\frac{\partial \mathbf{x}/\partial v }{\left| \partial \mathbf{x}/\partial v \right| },\quad \mathbf{e}_w=\frac{\partial \mathbf{x}/\partial w }{\left| \partial \mathbf{x}/\partial w \right| }
\]
form a \textit{right-handed orthonormal} basis ($ \mathbf{e}_u\times\mathbf{e}_v=\mathbf{e}_w $ etc.). Note that $ \mathbf{e}_u,\mathbf{e}_v,\mathbf{e}_w $ need \textit{not} be fixed.

It is standard to write 
\[
    h_u = \left| \frac{\partial \mathbf{x}}{\partial u}  \right| ,\quad h_v=\left| \frac{\partial \mathbf{x}}{\partial v}  \right|,\quad h_w=\left| \frac{\partial \mathbf{x}}{\partial w}  \right|
\]
called \textbf{scale factors}. They are called so because 
\[
    \mathrm{d} x = \frac{\partial \mathbf{x}}{\partial u}\dd u+ \frac{\partial \mathbf{x}}{\partial v}\dd v+\frac{\partial \mathbf{x}}{\partial w}\dd w=h_u \dd u \mathbf{e}_u+h_v \dd v \mathbf{e}_v+h_w \dd w \mathbf{e}_w.
\]
They describe how small changes in $ (u,v,w) $ scale up (to 1st order) to changes in $\mathbf{x}$ wrt $ \mathbf{e}_u,\mathbf{e}_v,\mathbf{e}_w $.

\subsubsection*{Cylindrical polar coordinates}
Cylindrical polar coordinates $ (\rho,\phi,z) $ are defined by 
\[
    \mathbf{x}(\rho,\phi,z)= \begin{pmatrix}
        \rho\cos \phi \\ \rho \sin \phi \\ z
    \end{pmatrix}, \quad\begin{matrix}
        0\le \rho< \infty, \\ 0\le \phi<2\pi,\\ -\infty <z<\infty,
    \end{matrix}
\]
with 
\[
    \mathbf{e}_\rho=\begin{pmatrix}
        \cos \phi \\ \sin \phi \\ 0
    \end{pmatrix},\quad
    \mathbf{e}_\phi=\begin{pmatrix}
        -\sin\phi \\ \cos \phi \\ 0
    \end{pmatrix},\quad
    \mathbf{e}_z=\begin{pmatrix}
        0 \\ 0 \\ 1
    \end{pmatrix},
\]
scale factors
\[
    h_\rho=1,\quad,h_\phi=\rho,\quad h_z=1,
\]
and line element
\[
    \mathrm{d} \mathbf{x} = \mathrm{d} \rho\, \mathbf{e}_\rho+\rho\dd \phi \, \mathbf{e}_\phi+\mathrm{d} z \, \mathbf{e}_z.
\]
\begin{note}
    \[
        \mathbf{x}=\begin{pmatrix}
            \rho\cos \phi \\ \rho \sin \phi \\ z
        \end{pmatrix}=\rho\begin{pmatrix}
            \cos \phi \\  \sin \phi \\ 0
        \end{pmatrix}+z \begin{pmatrix}
            0 \\ 0 \\ 1
        \end{pmatrix}=\rho\, \mathbf{e}_\rho+z\, \mathbf{e}_z.
    \]
\end{note}

See \href{http://jt775.user.srcf.net/IA-Lent/handouts/vc_handout1.pdf}{handout 1} for detailed diagrams.

\subsubsection*{Spherical polar coordinates}
Define $ (r,\theta,\phi) $ by 
\[
    \mathbf{x}(r,\theta,\phi) = \begin{pmatrix}
        r \cos \phi \sin \theta \\ r \sin \phi \sin \theta \\ r \cos \theta 
    \end{pmatrix},\quad \begin{matrix}
        0\le r\le \infty, \\ 0\le \theta<\pi,\\ 0 \le \phi< 2\pi,
    \end{matrix}
\]
with 
\[
    \mathbf{e}_r = \begin{pmatrix}
        \cos \phi \sin \theta \\ \sin \phi \sin \theta\\ \cos \theta 
    \end{pmatrix},\quad \mathbf{e}_\theta = \begin{pmatrix}
        \cos \phi \cos \theta \\ \sin \phi \cos \theta \\ -\sin \theta 
    \end{pmatrix},\quad \mathbf{e}_\phi = \begin{pmatrix}
        - \sin \phi \\ \cos \phi \\ 0
    \end{pmatrix},
\]
and scale factors $ h_r=1,h_\theta=r,h_\phi=r \sin \theta $. Line element is 
\[
    \mathrm{d} \mathbf{x} = \mathrm{d} r\,\mathbf{e}_r+ r\dd \theta\,\mathbf{e}_\theta+r \sin \theta\dd \phi\,\mathbf{e}_\phi,
\]
and $ \mathbf{x}=r\mathbf{e}_r $. See \href{http://jt775.user.srcf.net/IA-Lent/handouts/vc_handout1.pdf}{handout 1} for detailed diagrams.

\subsection{The gradient operator}
\begin{definition}[Gradient]
    For $ f: \mathbb{R}^{3}\to \mathbb{R} $, define \textbf{gradient} of $f$ by 
    \begin{equation}\label{eq:gradient}\tag{$ * $}
        \nabla f : f(\mathbf{x}+\mathbf{h}) = f(\mathbf{x})+ \nabla f(\mathbf{x}) \cdot \mathbf{h}+o(\mathbf{h})\quad (|\mathbf{h}|\to 0).
    \end{equation}
\end{definition}
\begin{definition}[Directional derivative]
    The \textbf{directional derivative} of $f$ in direction $\mathbf{v}$, denoted by $ D_\mathbf{v} f $ or $ \partial f/\partial \mathbf{v}  $ is defined by 
    \[
        D_\mathbf{v} f(\mathbf{x}) = \lim_{t \to 0} \frac{f(\mathbf{x}+t\mathbf{v})-f(\mathbf{x})}{t}, 
    \]
    i.e.
    \begin{equation}\label{eq:direc deriv}\tag{$ ** $}
        f(\mathbf{x}+t\mathbf{v})=f(\mathbf{x})+t D_\mathbf{v} f(\mathbf{x})+o(t),\quad t\to 0. 
    \end{equation}
\end{definition}

Set $ \mathbf{h}=t\mathbf{v} $ in (\ref{eq:gradient}), we get 
\[
    f(\mathbf{x}+t\mathbf{v}) = f(\mathbf{x})+t \nabla f(\mathbf{x}) \cdot \mathbf{v} + o(t),\quad t\to 0.
\]
Comparing to (\ref{eq:direc deriv}), we get 
\[
    D_\mathbf{v} f = \mathbf{v} \cdot \nabla f.
\]
By Cauchy-Schwarz, $ \nabla f $ points in direction of greatest increase of $f$. Similarly $ - \nabla f $ points in direction of greatest decrease of $f$.

\begin{example}
    Suppose $ f(\mathbf{x}) = \frac{1}{2}|\mathbf{x}|^2 $. Then 
    \begin{align*}
        f(\mathbf{x}+\mathbf{h}) &= \frac{1}{2}(\mathbf{x}+\mathbf{h})\cdot (\mathbf{x}+\mathbf{h})\\ 
        &= \frac{1}{2}|\mathbf{x}|^2+ \mathbf{x} \cdot \mathbf{h}+\frac{1}{2}|\mathbf{h}|^2\\ 
        &= \frac{1}{2}|\mathbf{x}|^2+\mathbf{x} \cdot \mathbf{h}+o(\mathbf{h}),\quad |\mathbf{h}|\to 0,
    \end{align*}
    so $ \nabla f =\mathbf{x} $.
\end{example}

\subsubsection*{Gradient along a curve}
Suppose we have a curve $ t \mapsto \mathbf{x}(t) $. How does $f$ change as we move along this curve?

Write $ F(t) = f(\mathbf{x}(t)) $, $ \delta\mathbf{x} = \mathbf{x}(t+\delta t)-\mathbf{x}(t) $.
\begin{align*}
    F(t+\delta t) &= f(\mathbf{x}(t+\delta t)) = f(\mathbf{x}(t)+\delta \mathbf{x})\\ 
    &= f(\mathbf{x}(t))+\nabla f(\mathbf{x}(t)) \cdot \delta\mathbf{x} +o(\delta \mathbf{x}),\quad |\delta\mathbf{x}| \to 0.
\end{align*}
Since $ \delta\mathbf{x} = \delta t\, \mathbf{x}'(t)+o(\delta t) $, 
\begin{align*}
    &F(t+\delta t)= F(t)+\mathbf{x}'(t) \cdot \nabla f(\mathbf{x}(t)) \delta t+o(\delta t)\\ 
    \Longrightarrow & \boxed{\frac{\mathrm{d}F}{\mathrm{d}t} = \frac{\mathrm{d}}{\mathrm{d}t}f(\mathbf{x}(t)) = \frac{\mathrm{d}\mathbf{x}}{\mathrm{d}t} \cdot \nabla f(\mathbf{x}(t))}.
\end{align*}

\subsubsection*{Gradient as the normal of a surface}
Suppose a surface $S$ is defined implicitly by 
\[
    S = \{\mathbf{x}\in \mathbb{R}^{3}: f(\mathbf{x})=0\}.
\]
If $ t \mapsto \mathbf{x}(t) $ is \textit{any} curve \textit{in} $S$, then $ f(\mathbf{x}(t))=0 $ and 
\[
    0 = \frac{\mathrm{d}}{\mathrm{d}t}f(\mathbf{x}(t)) =  \frac{\mathrm{d}\mathbf{x}}{\mathrm{d}t} \cdot \nabla f(\mathbf{x}(t)),
\]
so $ \nabla f $ is \textit{orthogonal} to tangent vector of \textit{any curve} in $S$, and thus is \textit{normal} to $S$ at $ \mathbf{x} $.

\subsection{Computing the gradient}
\subsubsection*{In Cartesian coordinates}
To get $ \mathbf{x} \mapsto \mathbf{x}+\mathbf{h} $, simply let $ x \mapsto x+h_1 $ etc. and 
\begin{align*}
    f(\mathbf{x}+\mathbf{h}) &= f(x+h_1,y+h_2,z+h_3)\\ 
    &= f(\mathbf{x}) + \frac{\partial f}{\partial x}h_1+\frac{\partial f}{\partial y}h_2+\frac{\partial f}{\partial z}h_3 +o(\mathbf{h})\\ 
    &= f(\mathbf{x})+ \begin{pmatrix}
        \partial f/\partial x  \\ \partial f/\partial y  \\ \partial f/\partial z 
    \end{pmatrix}\cdot \mathbf{h}+o(\mathbf{h}),
\end{align*}
so 
\[
    \nabla f = \begin{pmatrix}
        \partial f/\partial x  \\ \partial f/\partial y  \\ \partial f/\partial z 
    \end{pmatrix} \Longleftrightarrow \nabla f = \mathbf{e}_i \frac{\partial f}{\partial x_i}.
\]
We see that $ \nabla $ is a ``vector differential operator'' and in Cartesian coordinates
\[
    \nabla = \mathbf{e}_i \frac{\partial }{\partial x_i} . 
\]

\begin{example}
    Let $ f=\frac{1}{2}|\mathbf{x}|^2 = \frac{1}{2}(x^2+y^2+z^2) $. Then $ \nabla f = (x,y,z) $.
\end{example}

\subsubsection*{In orthogonal curvilinear coordinates}
Recall in Cartesian coordinates $ \mathrm{d} \mathbf{x} = \mathrm{d} x_i \mathbf{e}_i $ and $ \mathrm{d} f = (\partial f/\partial x_i )\dd x_i $. Notice that
\[
    \nabla f \cdot \mathrm{d} \mathbf{x} = \left( \mathbf{e}_i \frac{\partial f}{\partial x_i}  \right) \cdot \left( \mathbf{e}_j \dd x_j \right) = \frac{\partial f}{\partial x_i}(\mathbf{e}_i\cdot \mathbf{e}_j) \dd x_j = \frac{\partial f}{\partial x_i}\dd x_i = \mathrm{d} f.  
\]
Thus in \textit{any} coordinates, 
\[
    \boxed{\nabla f\cdot \mathrm{d} \mathbf{x} = \mathrm{d} f}
\] 

\begin{proposition}\label{prop:2.1}
    If $ (u,v,w) $ is OCC\footnote{orthogonal curvilinear coordinates} and $f=f(u,v,w)$, then 
    \[
        \nabla f = \frac{1}{h_u}\frac{\partial f}{\partial u}\mathbf{e}_u + \frac{1}{h_v}\frac{\partial f}{\partial v}\mathbf{e}_v + \frac{1}{h_w}\frac{\partial f}{\partial w}\mathbf{e}_w.
    \] 
\end{proposition}
\begin{proof}
    If $ f=f(u,v,w) $ and $ \mathbf{x} = \mathbf{x}(u,v,w) $, then 
    \[
        \mathrm{d} f = \frac{\partial f}{\partial u}\dd u +\frac{\partial f}{\partial v}\dd v +\frac{\partial f}{\partial w}\dd w,\ \mathrm{d} \mathbf{x} = h_u\dd u\mathbf{e}_u+h_v\dd v\mathbf{e}_v+h_w\dd w\mathbf{e}_w.
    \]
    Using $\nabla f\cdot \mathrm{d} \mathbf{x} = \mathrm{d} f$ and writing $ \nabla f = (\nabla f)_u\mathbf{e}_u+(\nabla f)_v\mathbf{e}_v+(\nabla f)_w\mathbf{e}_w $, we get 
    \[
        \frac{\partial f}{\partial u}\dd u +\frac{\partial f}{\partial v}\dd v +\frac{\partial f}{\partial w}\dd w = h_u(\nabla f)_u \dd u+h_v(\nabla f)_v \dd v+h_w(\nabla f)_w \dd w.
    \]
    Since $ \mathrm{d} u, \mathrm{d} v,\mathrm{d} w $ are linearly independent, comparing coefficients gives the result.
\end{proof}

Hence in cylindrical polars,
\[
    \nabla f = \frac{\partial f}{\partial \rho}\mathbf{e}_\rho+\frac{1}{\rho} \frac{\partial f}{\partial \phi}\mathbf{e}_\phi+\frac{\partial f}{\partial z}\mathbf{e}_z,  
\]
in spherical polars,
\[
    \nabla f = \frac{\partial f}{\partial r}\mathbf{e}_r+\frac{1}{r}\frac{\partial f}{\partial \theta}\mathbf{e}_\theta+\frac{1}{r\sin\theta}\frac{\partial f}{\partial \phi}\mathbf{e}_\phi.   
\]
\begin{example}
    Let $ f(\mathbf{x})= \frac{1}{2}|\mathbf{x}|^2 $, then 
    \[
        f = \begin{cases}
        \frac{1}{2}(x^2+y^2+z^2) &\text{Cartesian}\\
        \frac{1}{2}(\rho^2+z^2) &\text{Cylindrical}\\
        \frac{1}{2}r^2 &\text{Spherical}
        \end{cases} 
    \]
    Can check $ \nabla f $ gives correct results.
\end{example}
Check \href{https://www.vle.cam.ac.uk/pluginfile.php/19798882/mod_resource/content/2/vc_notes1.pdf#page=17}{this page} for summary.