\documentclass[10pt]{article}
\pdfoutput=1 
\usepackage{NotesTeX,lipsum}

%\usepackage{showframe}

\title{\begin{center}{\Huge \textit{Vectors and Matrices}}\\{{\itshape Based on Lectures and "Intro to Linear Algebra"}}\end{center}}
\author{$\theta\omega\theta$}
\affiliation{
Not in University of Cambridge\\
skipped some talks irrelevant to contents\\
}

\emailAdd{not telling you}

\begin{document}
	\maketitle
	\flushbottom
	\newpage
	\pagestyle{fancynotes}
	\part{Complex Numbers}
    \section{Definition}
    \begin{definition}
        Construct $\mathbb{C} $ from $ \mathbb{R}  $ by adding $i$ that $i^2 = -1$. Any $z\in \mathbb{C}$ is in the form
        \[
            z = x+iy, x = \Re z, y= \Im z, x,y\in \mathbb{R} . 
        \] 
        Addition and multiplication are defined by
        \[
            z_1+z_2 = (x_1+x_2) + i(y_1+y_2), z_1z_2 = (x_1+iy_1)(x_2+iy_2)
        .\]
        The \textit{conjugate} is defined by
        \[
            \bar{z} = z* = x-iy
        .\]
        The \textit{modulus} is defined by
        \[
            r = |z|, r\ge 0, r^2=|z|^2=z\bar{z}=x^2+y^2
        .\]
        The \textit{argument} is defined by
        \[
            z\neq 0: \theta=\arg(z)\in \mathbb{R}, z=r(\cos \theta+i\sin \theta)
        .\]
        The values of $ \theta $ in $ (-\pi, \pi] $ are called the \textit{principal values}.

        Complex numbers can be plotted on an \textit{Argand diagram}.
    \end{definition}
    \section{Basic Properties \& Consequences}
    \begin{enumerate}[(1)]
        \item $ +, \times $ are commutative and associative,
        
            $ \mathbb{C} $ under $+$ is an abelian group,
            
            $ \mathbb{C} $ under $\times$ is an abelian group,

            $ \mathbb{C} $ is a field.
        \item \textbf{Fundamental Theorem of Algebra}: A polynomial with deg $n$ with coefficients in $ \mathbb{C}  $ can be written as a product of $n$ linear factors, has at least one solution in $ \mathbb{C}  $ and has n solutions connected with multiplicity.
        \item Parallelogram constructions.
        \item \[
            \left| z_1 \right| \left| z_2 \right| = \left| z_1z_2 \right|, \left| z_1+z_2 \right| \le \left| z_1 \right| +\left| z_2 \right|
        .\]

        Alternative forms:
        \[
            \left| z_2-z_1 \right| \ge \left| z_2 \right| -\left| z_1 \right|, \left| z_2-z_1 \right| \ge \left| |z_2|-|z_1| \right| 
        .\]
        \item \textbf{De Moivre's Theorem}: $ z^n = r^n(\cos n\theta+i\sin n\theta) $.
    \end{enumerate}
\end{document}