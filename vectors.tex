\documentclass[10pt]{article}
\pdfoutput=1 
\usepackage{NotesTeX,lipsum}

%\usepackage{showframe}

\title{\begin{center}{\Huge \textit{Vectors and Matrices}}\\{{\itshape Based on Lectures and "Intro to Linear Algebra"}}\end{center}}
\author{$\theta\omega\theta$}
\affiliation{
Not in University of Cambridge\\
skipped some talks irrelevant to contents\\
}
\DeclareMathOperator{\spn}{span}

\emailAdd{not telling you}

\begin{document}
	\maketitle
	\flushbottom
	\newpage
	\pagestyle{fancynotes}
	\part{Complex Numbers}
    \section{Definition}
    \begin{definition}
        Construct $\mathbb{C} $ from $ \mathbb{R}  $ by adding $i$ that $i^2 = -1$. Any $z\in \mathbb{C}$ is in the form
        \[
            z = x+iy, x = \Re z, y= \Im z, x,y\in \mathbb{R} . 
        \] 
        Addition and multiplication are defined by
        \[
            z_1+z_2 = (x_1+x_2) + i(y_1+y_2), z_1z_2 = (x_1+iy_1)(x_2+iy_2)
        .\]
        The \textit{conjugate} is defined by
        \[
            \bar{z} = z* = x-iy
        .\]
        The \textit{modulus} is defined by
        \[
            r = |z|, r\ge 0, r^2=|z|^2=z\bar{z}=x^2+y^2
        .\]
        The \textit{argument} is defined by
        \[
            z\neq 0: \theta=\arg(z)\in \mathbb{R}, z=r(\cos \theta+i\sin \theta)
        .\]
        The values of $ \theta $ in $ (-\pi, \pi] $ are called the \textit{principal values}.

        Complex numbers can be plotted on an \textit{Argand diagram}.
    \end{definition}
    \section{Basic Properties \& Consequences}
    \begin{enumerate}[(1)]
        \item $ +, \times $ are commutative and associative,
        
            $ \mathbb{C} $ under $+$ is an abelian group,
            
            $ \mathbb{C} $ under $\times$ is an abelian group,

            $ \mathbb{C} $ is a field.
        \item \textbf{Fundamental Theorem of Algebra}: A polynomial with deg $n$ with coefficients in $ \mathbb{C}  $ can be written as a product of $n$ linear factors, has at least one solution in $ \mathbb{C}  $ and has n solutions connected with multiplicity.
        \item Parallelogram constructions.
        \item \[
            \left| z_1 \right| \left| z_2 \right| = \left| z_1z_2 \right|, \left| z_1+z_2 \right| \le \left| z_1 \right| +\left| z_2 \right|
        .\]

        Alternative forms:
        \[
            \left| z_2-z_1 \right| \ge \left| z_2 \right| -\left| z_1 \right|, \left| z_2-z_1 \right| \ge \left| |z_2|-|z_1| \right| 
        .\]
        \item \textbf{De Moivre's Theorem}: $ z^n = r^n(\cos n\theta+i\sin n\theta) $.
    \end{enumerate}
    \section{Exponential and Trigs in $\mathbb{C}$}
    \begin{definition}
        Define $ \exp, \cos, \sin $ on $ \mathbb{C} $ by
        \[
            \begin{aligned}
                 \exp(z)&= e^z = \sum_{n=0}^{\infty} \frac{1}{n!}z^n,\\
                 \cos(z)&= \frac{1}{2}(e^{iz}+e^{-iz}) = 1-\frac{1}{2!}z^2+\frac{1}{4!}z^4+\cdots,\\
                 \sin(z)&= \frac{1}{2i}(e^{iz}-e^{-iz}) = z-\frac{1}{3!}z^3+\frac{1}{5!}z^5+\cdots.\\
            \end{aligned}
        \]
        These series converge for all $ z\in \mathbb{C} $. Can be multiplied, rearranged, etc. Definitions reduce to familiar ones in the reals.
    \end{definition}
    \begin{proposition}\label{prop:exp multi}
        $ \forall z,w\in \mathbb{C}, e^ze^w=e^{z+w}; e^ze^{-z}=1, (e^z)^n =e^{nz}, n\in \mathbb{Z}$.
    \end{proposition}
    \begin{lemma}\label{lma:exp_arth}
        For $z=x+iy$:
        \begin{enumerate}[(1)]
            \item $ e^z = e^x(\cos y+i\sin y) $.
            \item $ \exp(z)\in \mathbb{C} \setminus \left\{ 0\right\} $.
            \item $ e^z = 1 \Leftrightarrow z = 2\pi n i, n\in \mathbb{Z} $.
        \end{enumerate}
    \end{lemma}
    \begin{definition}[Roots of unity]
        $z$ is an $ N $th root of unity if $ z^N=1 $.
    \end{definition}
    We have 
    \[
        z^N=r^Ne^{iN\theta} =1 \Longleftrightarrow r=1, N\theta = 2n\pi \Longleftrightarrow \theta = \frac{2n\pi}{N}
    ,\]
    which gives $N$ distinct solutions
    \[
        z = \frac{2n\pi}{N} = \omega^n, \quad, n=0,1,\dots, N-1
    .\]
    $\omega^n$ lie one the vertices of a regular $n$-gon on the unit circle.
    \section{Logarithms and Complex powers}
    \begin{definition}
        Define $ w = \log z, z\in \mathbb{C} \land z\neq 0 $ by $ e^w = e^{\log z}=z $. Note that since exp is many-to-one, $\log$ is multi-valued.
        \[
            \begin{aligned}
                 & z = re^{i\theta} =e^{\log r}e^{i\theta}=e^{\log r+ i\theta}\\
                 \Longrightarrow & \boxed{\log z = \log r+i\theta = \log |z|+i\arg(z)}
            \end{aligned}
        \]
        To make it single-valued, simply take the principal value.
    \end{definition}
    \begin{definition}
        Define \textit{complex power} by 
        \[
            z^\alpha = e^{\alpha\log z}, \quad z,\alpha\in \mathbb{C} , z\neq 0
        .\]
        Note that since $ \arg z \to \arg z + 2n\pi \Rightarrow z^\alpha \to z^\alpha e^{2n\pi} $, it is generally multi-valued. This also reduces to common powers when $ z,\alpha\in \mathbb{R}. $
    \end{definition}
    \begin{example}
        \[
            i^i = e^{i\log i} = e^{i(0+i(\frac{\pi}{2}+2n\pi))} = e^{-(\frac{\pi}{2}+2n\pi)}
        .\]
    \end{example}
    \section{Transformations, Lines, and Circles}
    \begin{itemize}
        \item We have five elementary transformations:
        \begin{enumerate}[(1)]
            \item $ z \mapsto z+a $,
            \item $ z \mapsto \lambda z $,
            \item $ z \mapsto e^{i\alpha} z $,
            \item $ z \mapsto \bar{z} $,
            \item $ z \mapsto \frac{1}{z} $.
        \end{enumerate}
        \item General point of a line in $\mathbb{C}$ through $z_0$ and parallel to $w$:
        \[
            z = z +\lambda w, \lambda\in \mathbb{R} \text{ or } \bar{w}z-w\bar{z}=\bar{w}z_0-w\bar{z_0}
        .\]
        \item General point of a circle in $\mathbb{C}$ with centre $ c $ and radius $ \rho $:
        \[
            z = c+ \rho e^{i\theta} \text{ or } \left| z-c \right| = \rho \text{ or } |z|^2-\bar{c}z-c\bar{z}=\rho^2-|c|^2
        .\]
        \item Stereographic projection.
    \end{itemize}
    \part{Vectors in 3 Dimensions}
    \section{Vector addition and scalar multiplication}
    \begin{definition}[scalar multiplication]
        Given $ \mathbf{a} $, and scalar $ \lambda\in \mathbb{R} $, define $ \lambda \mathbf{a} $ to be the position vector of $ A' $ on the line $OA$ with length $ |\lambda \mathbf{a}|=\left| \lambda \right| \left| \mathbf{a} \right|  $. Direction depends on the sign of $ \lambda $.
    \end{definition}
    Define $ \spn\left\{ \mathbf{a}\right\} = \left\{ \lambda \mathbf{a}: \lambda\in \mathbb{R} \right\} $. If $ \mathbf{a}\neq 0 $, then $ \spn \{\mathbf{a}\} $ is the entire line through $O$ and $A$.

    Define $ \mathbf{a} \parallel \mathbf{b} $ if and only if either $ \mathbf{a} = \lambda \mathbf{b} $ or $ \mathbf{b} = \lambda \mathbf{a} $. Allow $ \lambda=0 $, so $\forall \mathbf{a}, \mathbf{0} \parallel \mathbf{a} $. Also allow $\lambda<0$.
    \begin{definition}[vector addition]
        Give $ \mathbf{a}, \mathbf{b} $, if $ \mathbf{a} \nparallel \mathbf{b} $, construct a parallelogram $OACB$ and define $ \mathbf{c} = \mathbf{a} + \mathbf{b} $.
    \end{definition}
    If $ a \parallel b $, then $ \mathbf{a} = \alpha \mathbf{u}, \mathbf{b}=\beta \mathbf{u} $, where $\mathbf{u}$ is a unit vector and $ \mathbf{a}+\mathbf{b}=(\alpha+\beta)\mathbf{u} $.

    Given $ \mathbf{a},\mathbf{b},\dots,\mathbf{c} $, we have a linear combination
    \[
        \alpha \mathbf{a}+ \beta \mathbf{b}+\cdots+\gamma \mathbf{c}
    \]
    for any $ \alpha,\beta,\dots, \gamma\in \mathbb{R}  $.

    Define $ \spn \left\{ \mathbf{a},\mathbf{b},\dots,\mathbf{c}\right\} = \left\{ \alpha \mathbf{a}+ \beta \mathbf{b}+\cdots+\gamma \mathbf{c}:\alpha,\beta,\dots, \gamma\in \mathbb{R} \right\} $. In 3d case, if $ \mathbf{a} \nparallel \mathbf{b} $, then $ \spn \{ \mathbf{a}, \mathbf{b} \} $ is a plane through $O,A,B$.
    
    Here are some properties:
    \begin{itemize}
        \item $ \forall \mathbf{a}, \mathbf{b}, \mathbf{c}, \mathbf{a}+\mathbf{0}=\mathbf{0}+\mathbf{a}=\mathbf{a} $, this says that $ \mathbf{0} $ is the identity for addition.
        \item $ \exists -\mathbf{a}, \mathbf{a}+(-\mathbf{a})=(-\mathbf{a})+\mathbf{a}=\mathbf{0} $. This says $ -\mathbf{a} $ is the inverse of $ \mathbf{a} $ under addition.
        \item $ \forall \mathbf{a},\mathbf{b}, \mathbf{a}+\mathbf{b}=\mathbf{b}+\mathbf{a} $, this says that vector addition is commutative.
        \item $ \forall \mathbf{a},\mathbf{b},\mathbf{c}, (\mathbf{a}+\mathbf{b})+\mathbf{c}=\mathbf{a}+(\mathbf{b}+\mathbf{c}) $, this says that vector addition is associative.
    \end{itemize}
    Hence, the set of vectors with addition form an abelian group.

    Relation with scalars:
    \begin{itemize}
        \item $ \lambda(\mathbf{a}+\mathbf{b})=\lambda \mathbf{a}+\lambda \mathbf{b} $.
        \item $ (\lambda+\mu)\mathbf{a}=\lambda \mathbf{a}+ \mu \mathbf{a} $.
        \item $ (\lambda \mu)\mathbf{a}=\lambda(\mu \mathbf{a}) $.
    \end{itemize}
    \section{Dot product}
    \begin{definition}[dot product]
        Give $ \mathbf{a}, \mathbf{b} $, let $ \theta $ be the angle between them, define $ \mathbf{a}\cdot\mathbf{b}=|\mathbf{a}||\mathbf{b}|\cos \theta $. Note that $ \theta $ is defined unless $ \mathbf{a}=\mathbf{0} $, in which case we define $ \mathbf{a}\cdot \mathbf{b}=0 $.

        $ \mathbf{a}\perp\mathbf{b} \Leftrightarrow \mathbf{a}\cdot \mathbf{b}=0 \Leftrightarrow \theta=\frac{\pi}{2}\bmod \pi $ when $ \theta $ is defined. Allow $ \mathbf{a} $ or $ \mathbf{b}=0 $, so $ \mathbf{a} \parallel \mathbf{0} \land \mathbf{a} \perp \mathbf{0} $.
    \end{definition}
    For $ \mathbf{a}\neq \mathbf{0} $, $ |\mathbf{b}|\cos \theta $ is the component of $ \mathbf{b} $ along $ \mathbf{a} $.
    \[
        \left| \mathbf{b} \right| \cos \theta = \frac{\mathbf{a}\cdot \mathbf{b}}{|\mathbf{a}|}=\mathbf{u}\cdot \mathbf{b}
    .\]
    By resolving $ \mathbf{b} $ along and perpendicular to $ \mathbf{a} $, we get 
    \[
        \mathbf{b} = \mathbf{b}_{\parallel} + \mathbf{b}_{\perp }
    .\]
    Properties:
    \begin{itemize}
        \item $ \mathbf{a}\cdot \mathbf{b}=\mathbf{b}\cdot \mathbf{a}, \mathbf{a}\cdot \mathbf{a}=|\mathbf{a}|^2\ge 0 $, $ =0 $ iff $ \mathbf{a}=\mathbf{0} $.
        \item $ (\lambda \mathbf{a})\cdot \mathbf{b}=\lambda (\mathbf{a}\cdot \mathbf{b})=\mathbf{a}\cdot (\lambda \mathbf{b}) $.
        \item $ \mathbf{a}\cdot (\mathbf{b}+\mathbf{c})=\mathbf{a}\cdot \mathbf{b}+\mathbf{a}\cdot \mathbf{c} $.
    \end{itemize}
    \section{Vector cross product}
    \begin{definition}
        Given $ \mathbf{a},\mathbf{b} $, let $ \theta $ be the angle between them, wrt a unit vector $ \mathbf{n} $ normal to the plane they span. Define $ \mathbf{a} \wedge \mathbf{b} $ or $ \mathbf{a} \times \mathbf{b} $ as $ |\mathbf{a}||\mathbf{b}|\sin \theta \mathbf{n} $. $ \mathbf{0} $ case is similar.
    \end{definition}
    This is the \textit{vector area} of the parallelogram generated by $ \mathbf{a}, \mathbf{b} $. Note that $ \mathbf{a} \wedge \mathbf{b} = \mathbf{a} \wedge \mathbf{b}_{\perp } $.

    Properties:
    \begin{itemize}
        \item $ \mathbf{a} \wedge \mathbf{b}=\mathbf{b}\wedge \mathbf{a} $.
        \item $ (\lambda \mathbf{a})\wedge \mathbf{b}=\lambda(\mathbf{a}\wedge \mathbf{b})=\mathbf{a} \wedge (\lambda\mathbf{b}) $.
        \item $ \mathbf{a}\wedge (\mathbf{b}+\mathbf{c})=\mathbf{a}\wedge \mathbf{b}+\mathbf{a}\wedge \mathbf{c} $.
        \item $ \mathbf{a} \wedge \mathbf{b}=\mathbf{0} $ if and only if $ \mathbf{a} \parallel \mathbf{b} $.
        \item $ \mathbf{a}\wedge \mathbf{b} \perp \mathbf{a} \land \perp \mathbf{b} $.
    \end{itemize}
    %Lecture 4
    \section{Orthonormal Bases and Components}\marginnote{Lecture 4.}
    Choose $ \mathbf{e}_1, \mathbf{e}_2, \mathbf{e}_3 $ that are \textit{orthonormal}. That is, they are of unit lengths and $ \mathbf{e}_i\cdot \mathbf{e}_j =0, i\neq j\in \{1,2,3\}$, which is equivalent to choose cartesian axes along the directions. Then $ \left\{ \mathbf{e}_i\right\} $ is a basis and $ \forall \mathbf{a}\in \mathbb{R}^3 $, 
    \[
        \mathbf{a}=\sum_{i=1}^{3}a_i \mathbf{e}_i \marginnote{Each component $ a_i $ is uniquely determined by $ a_i = \mathbf{e}_i \cdot \mathbf{a} $.}
    .\]
    By this spirite, we can write 
    \[
        \mathbf{a} = (a_1,a_2,a_3) = \begin{pmatrix}
            a_1\\ 
            a_2\\ 
            a_3\\
            \end{pmatrix}
    .\]
    Scalar product in this form can be written as 
    \[
        \begin{pmatrix}
            a_1\\ 
            a_2\\ 
            a_3\\
            \end{pmatrix} \cdot \begin{pmatrix}
                b_1\\ 
                b_2\\ 
                b_3\\
                \end{pmatrix}
                =a_1b_1+a_2b_2+a_3b_3,\quad \left| \mathbf{a} \right| = a_1^2+a_2^2+a_3^2.
    .\]
    For vector products, choose this basis that it is also \textit{right-handed}:
    \[
        \mathbf{e}_i \times \mathbf{e}_{i+1}=\mathbf{e}_{i+2}
    .\]
    Then
    \[
        \begin{aligned}
            \mathbf{a}\times \mathbf{b} &= (a_1 \mathbf{e_1}+a_2 \mathbf{e}_2+a_3 \mathbf{e}_3)(b_1 \mathbf{e_1}+b_2 \mathbf{e}_2+b_3 \mathbf{e}_3)\\
            &= (a_3b_2-a_2b_3)\mathbf{e}_1+(a_3b_1-a_1b_3)\mathbf{e}_2+(a_1b_2-a_2b_1)\mathbf{e}_3.
        \end{aligned}
    \]
    \section{Triple products}
    \subsection{Scalar triple product}
    \begin{definition}
        Define scalar triple product by
        \[
            [\mathbf{a},\mathbf{b},\mathbf{c}]=\mathbf{a}\cdot (\mathbf{b} \times \mathbf{c})=\mathbf{b}\cdot (\mathbf{c} \times \mathbf{a})=\mathbf{c}\cdot (\mathbf{a}\times \mathbf{b})
        .\]
        This is the volumn of the parallelepiped with bases $ \mathbf{b}, \mathbf{c} $ and side $ \mathbf{a} .$
    \end{definition}
    \begin{remark}
        $ \mathbf{c}\cdot (\mathbf{a}\times \mathbf{b}) $ is a "signed" volumn. If $ \mathbf{c}\cdot (\mathbf{a}\times \mathbf{b})>0 $ then $ \left\{ \mathbf{a},\mathbf{b},\mathbf{c}\right\} $ is called a \textit{right-handed set}. $ \mathbf{c}\cdot (\mathbf{a}\times \mathbf{b}) $ if and only if $ \mathbf{a},\mathbf{b},\mathbf{c} $ are coplanar, e.g., $ \mathbf{c}=\alpha \mathbf{a}+\beta \mathbf{b} \in \spn\left\{ \mathbf{a},\mathbf{b}\right\}$.
    \end{remark}
    In components, 
    \[
        \begin{aligned}
            \mathbf{a}\cdot (\mathbf{b}\times \mathbf{c})&=a_1b2c_3-a_1b_3c_2 \\
            &+a_2b_3c_1-a_2b_1c_3\\
            &+a_3b_1c_2-a_3b_2c_1\\
        \end{aligned} =\begin{vmatrix}
            a_{1} & a_{2} & a_{3} \\
            b_{1} & b_{2} & b_{3} \\
            c_{1} & c_{2} & c_{3}
        \end{vmatrix}.
    \]
    \subsection{Vector triple product}
    \begin{definition}
        Define the vector triple product by $ \mathbf{a}\times (\mathbf{b}\times \mathbf{c}) $. Note that $ \mathbf{a}\times (\mathbf{b}\times \mathbf{c}) $ does not necessarily give the same result as $ (\mathbf{a}\times \mathbf{b})\times \mathbf{c} $.
        \[
            \mathbf{a}\times (\mathbf{b}\times \mathbf{c})=(\mathbf{a}\cdot \mathbf{c})\mathbf{b}-(\mathbf{a}\cdot \mathbf{b})\mathbf{c}
        .\]
    \end{definition}
    \section{Lines, Planes, and Vector equations}
    Vectors are defined as position vectors from $O$. But the definition of addition enables us to use them to describe displacements between points.
    \subsection{Lines}
    General point on a line through $ \mathbf{a} $ through $ \mathbf{u} $:
    \[
        \begin{aligned}
            &\mathbf{r} = \mathbf{a}+\lambda\mathbf{u}, &\lambda\in \mathbb{R}\quad &\text{The parametric form.}\\
            &\mathbf{u} \times \mathbf{r} = \mathbf{u} \times \mathbf{a},&  &\text{Cross form.}
        \end{aligned}
    \]
    \begin{proposition}\label{prop:line_equation_vec}
        Any vector equation of the form $ \mathbf{u}\times \mathbf{r}=\mathbf{c} $ represents a line.
    \end{proposition}
    \begin{proof}
        $ \mathbf{u}\times \mathbf{r}=\mathbf{c} \Rightarrow \mathbf{u}\cdot (\mathbf{u}\times \mathbf{r})=\mathbf{u}\cdot \mathbf{c} \Leftrightarrow \mathbf{u} \cdot \mathbf{c}=0 $. If $ \mathbf{u} \cdot \mathbf{c}\neq 0 $ then the equation is inconsistent. If $ \mathbf{u}\cdot \mathbf{c} =0$, then note that 
        \[
            \mathbf{u} \times (\mathbf{u} \times \mathbf{c})=(\mathbf{u}\cdot \mathbf{c})\mathbf{u}-(\mathbf{u}\cdot \mathbf{u})\mathbf{c} = -\left| \mathbf{u} \right|^2 \mathbf{c}
        .\]
        Hence $ \mathbf{a} = -(\mathbf{u} \times \mathbf{c})/|\mathbf{u}|^2 $ is a solution, and thus it represents a line.
    \end{proof}
    \subsection{Planes}
    General point on a plane through $ \mathbf{a} $ with directions $ \mathbf{u},\mathbf{v} $ in the plane($ \mathbf{u}\nparallel \mathbf{v} $):
    \[
        \begin{aligned}
             &\mathbf{r} = \mathbf{a}+\lambda \mathbf{u}+\mu \mathbf{v}, &\lambda, \mu\in \mathbb{R} &\quad \text{Parametric form,}\\
             & \mathbf{n} \cdot \mathbf{r} = k = \mathbf{n} \cdot \mathbf{a}, & \mathbf{n} = \mathbf{u}\times \mathbf{v} &\quad \text{Dot form.}
        \end{aligned}
    \]
    The component of $\mathbf{r}$ along $ \mathbf{n} $ is 
    \[
        \frac{\mathbf{n}\cdot \mathbf{r}}{|\mathbf{n}|}=\frac{k}{|\mathbf{n}|}
    .\]
    \subsection{Other vector equations}
    \begin{enumerate}[(1)]
        \item $ |\mathbf{r}|^2+\mathbf{r}\cdot \mathbf{a}=k \Leftrightarrow \left| \mathbf{r}+\frac{1}{2}\mathbf{a} \right|^2=k+\frac{1}{4}|\mathbf{a}|^2  $, a sphere with centre $ -\frac{1}{2}\mathbf{a} $ and radius $ \sqrt{k+\frac{1}{4}|\mathbf{a}|^2} $, provided $ k>-\frac{1}{4}|\mathbf{a}|^2 $.
        \item $ \mathbf{r}+\mathbf{a} \times (\mathbf{b}\times \mathbf{r}) =\mathbf{c} \Leftrightarrow \mathbf{r}+(\mathbf{a}\cdot \mathbf{r})\mathbf{b}-(\mathbf{a}\cdot \mathbf{b})\mathbf{r}=\mathbf{c}$. Dot with $\mathbf{a}$:
        \[
            \mathbf{a}\cdot \mathbf{r}= \mathbf{a}\cdot \mathbf{c} \Longrightarrow (1-\mathbf{a}\cdot \mathbf{b})\mathbf{r}=\mathbf{c}-(\mathbf{a}\cdot \mathbf{c})\mathbf{b}
        .\]
        If $ \mathbf{a}\cdot \mathbf{b}\neq 1 $, then there is a unique solution
        \[
            \mathbf{r} = \frac{1}{1-\mathbf{a}\cdot \mathbf{b}}(\mathbf{c}-(\mathbf{a}\cdot \mathbf{c})\mathbf{b})
        ,\]
        which is a point.

        If $ \mathbf{a}\cdot \mathbf{b}=1 $ and RHS$\neq 0$, then it is inconsistent.

        If $ \mathbf{a}\cdot \mathbf{b} $ and $\mathbf{c}-(\mathbf{a}\cdot \mathbf{c})\mathbf{b}=\mathbf{0}$, then 
        \[
            (\mathbf{a}\cdot \mathbf{r}-\mathbf{a}\cdot \mathbf{c})\mathbf{b}=\mathbf{0}
        .\] 
        Hence it is a plane.
    \end{enumerate}
\end{document}