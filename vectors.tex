\documentclass[10pt]{article}
\pdfoutput=1 
\usepackage{NotesTeX,lipsum}

%\usepackage{showframe}

\title{\begin{center}{\Huge \textit{Vectors and Matrices}}\\{{\itshape Based on Lectures and "Intro to Linear Algebra"}}\end{center}}
\author{$\theta\omega\theta$}
\affiliation{
Not in University of Cambridge\\
skipped some talks irrelevant to contents\\
}

\emailAdd{not telling you}

\begin{document}
	\maketitle
	\flushbottom
	\newpage
	\pagestyle{fancynotes}
	\part{Complex Numbers}
    \section{Definition}
    \begin{definition}
        Construct $\mathbb{C} $ from $ \mathbb{R}  $ by adding $i$ that $i^2 = -1$. Any $z\in \mathbb{C}$ is in the form
        \[
            z = x+iy, x = \Re z, y= \Im z, x,y\in \mathbb{R} . 
        \] 
        Addition and multiplication are defined by
        \[
            z_1+z_2 = (x_1+x_2) + i(y_1+y_2), z_1z_2 = (x_1+iy_1)(x_2+iy_2)
        .\]
        The \textit{conjugate} is defined by
        \[
            \bar{z} = z* = x-iy
        .\]
        The \textit{modulus} is defined by
        \[
            r = |z|, r\ge 0, r^2=|z|^2=z\bar{z}=x^2+y^2
        .\]
        The \textit{argument} is defined by
        \[
            z\neq 0: \theta=\arg(z)\in \mathbb{R}, z=r(\cos \theta+i\sin \theta)
        .\]
        The values of $ \theta $ in $ (-\pi, \pi] $ are called the \textit{principal values}.

        Complex numbers can be plotted on an \textit{Argand diagram}.
    \end{definition}
    \section{Basic Properties \& Consequences}
    \begin{enumerate}[(1)]
        \item $ +, \times $ are commutative and associative,
        
            $ \mathbb{C} $ under $+$ is an abelian group,
            
            $ \mathbb{C} $ under $\times$ is an abelian group,

            $ \mathbb{C} $ is a field.
        \item \textbf{Fundamental Theorem of Algebra}: A polynomial with deg $n$ with coefficients in $ \mathbb{C}  $ can be written as a product of $n$ linear factors, has at least one solution in $ \mathbb{C}  $ and has n solutions connected with multiplicity.
        \item Parallelogram constructions.
        \item \[
            \left| z_1 \right| \left| z_2 \right| = \left| z_1z_2 \right|, \left| z_1+z_2 \right| \le \left| z_1 \right| +\left| z_2 \right|
        .\]

        Alternative forms:
        \[
            \left| z_2-z_1 \right| \ge \left| z_2 \right| -\left| z_1 \right|, \left| z_2-z_1 \right| \ge \left| |z_2|-|z_1| \right| 
        .\]
        \item \textbf{De Moivre's Theorem}: $ z^n = r^n(\cos n\theta+i\sin n\theta) $.
    \end{enumerate}
    \section{Exponential and Trigs in $\mathbb{C}$}
    \begin{definition}
        Define $ \exp, \cos, \sin $ on $ \mathbb{C} $ by
        \[
            \begin{aligned}
                 \exp(z)&= e^z = \sum_{n=0}^{\infty} \frac{1}{n!}z^n,\\
                 \cos(z)&= \frac{1}{2}(e^{iz}+e^{-iz}) = 1-\frac{1}{2!}z^2+\frac{1}{4!}z^4+\cdots,\\
                 \sin(z)&= \frac{1}{2i}(e^{iz}-e^{-iz}) = z-\frac{1}{3!}z^3+\frac{1}{5!}z^5+\cdots.\\
            \end{aligned}
        \]
        These series converge for all $ z\in \mathbb{C} $. Can be multiplied, rearranged, etc. Definitions reduce to familiar ones in the reals.
    \end{definition}
    \begin{proposition}\label{prop:exp multi}
        $ \forall z,w\in \mathbb{C}, e^ze^w=e^{z+w}; e^ze^{-z}=1, (e^z)^n =e^{nz}, n\in \mathbb{Z}$.
    \end{proposition}
    \begin{lemma}\label{lma:exp_arth}
        For $z=x+iy$:
        \begin{enumerate}[(1)]
            \item $ e^z = e^x(\cos y+i\sin y) $.
            \item $ \exp(z)\in \mathbb{C} \setminus \left\{ 0\right\} $.
            \item $ e^z = 1 \Leftrightarrow z = 2\pi n i, n\in \mathbb{Z} $.
        \end{enumerate}
    \end{lemma}
    \begin{definition}[Roots of unity]
        $z$ is an $ N $th root of unity if $ z^N=1 $.
    \end{definition}
    We have 
    \[
        z^N=r^Ne^{iN\theta} =1 \Longleftrightarrow r=1, N\theta = 2n\pi \Longleftrightarrow \theta = \frac{2n\pi}{N}
    ,\]
    which gives $N$ distinct solutions
    \[
        z = \frac{2n\pi}{N} = \omega^n, \quad, n=0,1,\dots, N-1
    .\]
    $\omega^n$ lie one the vertices of a regular $n$-gon on the unit circle.
    \section{Logarithms and Complex powers}
    \begin{definition}
        Define $ w = \log z, z\in \mathbb{C} \land z\neq 0 $ by $ e^w = e^{\log z}=z $. Note that since exp is many-to-one, $\log$ is multi-valued.
        \[
            \begin{aligned}
                 & z = re^{i\theta} =e^{\log r}e^{i\theta}=e^{\log r+ i\theta}\\
                 \Longrightarrow & \boxed{\log z = \log r+i\theta = \log |z|+i\arg(z)}
            \end{aligned}
        \]
        To make it single-valued, simply take the principal value.
    \end{definition}
    \begin{definition}
        Define \textit{complex power} by 
        \[
            z^\alpha = e^{\alpha\log z}, \quad z,\alpha\in \mathbb{C} , z\neq 0
        .\]
        Note that since $ \arg z \to \arg z + 2n\pi \Rightarrow z^\alpha \to z^\alpha e^{2n\pi} $, it is generally multi-valued. This also reduces to common powers when $ z,\alpha\in \mathbb{R}. $
    \end{definition}
    \begin{example}
        \[
            i^i = e^{i\log i} = e^{i(0+i(\frac{\pi}{2}+2n\pi))} = e^{-(\frac{\pi}{2}+2n\pi)}
        .\]
    \end{example}
    \section{Transformations, Lines, and Circles}
    \begin{itemize}
        \item We have five elementary transformations:
        \begin{enumerate}[(1)]
            \item $ z \mapsto z+a $,
            \item $ z \mapsto \lambda z $,
            \item $ z \mapsto e^{i\alpha} z $,
            \item $ z \mapsto \bar{z} $,
            \item $ z \mapsto \frac{1}{z} $.
        \end{enumerate}
        \item General point of a line in $\mathbb{C}$ through $z_0$ and parallel to $w$:
        \[
            z = z +\lambda w, \lambda\in \mathbb{R} \text{ or } \bar{w}z-w\bar{z}=\bar{w}z_0-w\bar{z_0}
        .\]
        \item General point of a circle in $\mathbb{C}$ with centre $ c $ and radius $ \rho $:
        \[
            z = c+ \rho e^{i\theta} \text{ or } \left| z-c \right| = \rho \text{ or } |z|^2-\bar{c}z-c\bar{z}=\rho^2-|c|^2
        .\]
        \item Stereographic projection.
    \end{itemize}
\end{document}